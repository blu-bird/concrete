%%%%%%% Lecture Notes Style for Concrete Mathematics %%%%%%%%%%%%%
%%%%%%% Taught by Patrick White, TJHSST %%%%%%%%%%%%%%%%%%%%%%%%%%

\documentclass[11pt,twosided]{article}
%%%%%%%%%%%%%%%%%%%%%%%%%%%%%%%%%%%%%%%%%%%%%%%%%%%%%%%
%%%% HEADER FILE for CONCRETE MATH LECTURE NOTES %%%%%%
%%%%%%%%%%%%%%%%%%%%%%%%%%%%%%%%%%%%%%%%%%%%%%%%%%%%%%%

%%%%%%%%%%%%%%%%%%%%%%%%%%%%%%%%%%%%%%%%%%%%%%%%%%%%%%%%%%%%%%%%%%%
%% This file is included at the top of every lecture notes file %%%
%% It should ONLY be changed by the instructor %%
%%%%%%%%%%%%%%%%%%%%%%%%%%%%%%%%%%%%%%%%%%%%%%%%%%%%%%%%%%%%%%%%%%%

%%%%%%%%%%%%%%%%%%%%%% package inclusions %%%%%%%%%%%%%%%%%%%%
%\usepackage[
%top=2cm,
%bottom=2cm,
%left=3cm,
%right=2cm,
%headheight=17pt, % as per the warning by fancyhdr
%includehead,includefoot,
%heightrounded, % to avoid spurious underfull messages
%]{geometry} 


\usepackage{amsmath,amssymb,amsfonts}
\usepackage{longtable}
\usepackage{mathtools}
\usepackage{amsthm}
\usepackage{enumerate}
\usepackage{fancyhdr}
\pagestyle{fancy}


%%%%%%%%%%%%%%% theorem style definitions %%%%%%%%%%%%%%%%%%%%%%
\newtheorem{theorem}{Theorem}[section]
%\newtheorem{corollary}{Corollary}[theorem]
%\newtheorem{lemma}[theorem]{Lemma}
\newtheorem*{remark}{Remark}
%\newtheorem{example}{Example}[section]
\newtheorem*{definition}{Definition}


%%%%%%%%%%%%%%%% custom function definitions %%%%%%%%%%%%%%%%%%%%%%%%%
%%%%%%%%%%% as class progresses, this section may be enhanced %%%%%%%%

\def\multiset#1#2{\ensuremath{\left(\kern-.3em\left(\genfrac{}{}{0pt}{}{#1}{#2}\right)
		\kern-.3em\right)}} %%% multiset notation
\newcommand\rf[2]{{#1}^{\overline{#2}}} %%%% rising factorial \rf{x}{m}
\newcommand\ff[2]{{#1}^{\underline{#2}}} %%%%% falling factorial \ff{x}{m}
\newcommand{\Perm}[2]{{}^{#1}\!P_{#2}} % permutation
\newcommand{\half}{\ensuremath{\frac{1}{2}}}
\newcommand{\braces}[1]{\left\{#1\right\}}
\newcommand{\set}[1]{\braces{#1}}
\newcommand{\snb}[2]{\ensuremath{\left(\kern-.3em\left(\genfrac{}{}{0pt}{}{#1}{#2}\right)\kern-.3em\right)}}
\newcommand{\fallingfactorial}[1]{^{\underline{#1}}}
\newcommand{\fallfac}[2]{{#1}^{\underline{#2}}}
\newcommand{\stirlingone}[2]{\genfrac[]{0pt}{1}{#1}{#2}}
\newcommand{\stiri}[2]{\stirlingone{#1}{#2}}
\newcommand{\dstirlingone}[2]{\genfrac[]{0pt}{0}{#1}{#2}}
\newcommand{\dstiri}[2]{\dstirlingone{#1}{#2}}
\newcommand{\stirlingtwo}[2]{\genfrac\{\}{0pt}{1}{#1}{#2}}
\newcommand{\stirii}[2]{\stirlingtwo{#1}{#2}}
\newcommand{\dstirlingtwo}[2]{\genfrac\{\}{0pt}{0}{#1}{#2}}
\newcommand{\dstirii}[2]{\dstirlingtwo{#1}{#2}}
\newcommand{\ol}[1]{\overline{#1}}
%%%%%%%
% book stuff
%%%%%%%%%

\usepackage[twoside,outer=1.5in,inner=2in,bottom=1.5in,top=1.5in,marginpar=2in]{geometry} 
\usepackage{scrextend}
\usepackage{palatino}
\usepackage{multicol}
\usepackage{float}
\usepackage[hidelinks]{hyperref}
\usepackage[nameinlink, capitalise, noabbrev]{cleveref}
%\usepackage{background}
\usepackage{graphicx}
\usepackage{listliketab}

\usepackage{lipsum}
\usepackage{caption}
\usepackage{marginnote}

\reversemarginpar


\usepackage[dvipsnames]{xcolor}
\usepackage{amsthm}
\usepackage{shadethm}

\setlength{\parindent}{0pt}

\newcommand*\Note{%
	\marginnote[\textcolor{blue}{\raggedright{\LARGE ?}\ Note }]{}%
}
\newcommand*\Warn{%
	\marginnote[\textcolor{red}{\raggedright{\LARGE !}\ Warning }]{}%
}
\reversemarginpar

\newcommand{\N}{\mathbb{N}}
\newcommand{\Z}{\mathbb{Z}}
\newcommand{\I}{\mathbb{I}}
\newcommand{\R}{\mathbb{R}}
\newcommand{\Q}{\mathbb{Q}}
\renewcommand{\qed}{\hfill$\blacksquare$}
\let\newproof\proof
\renewenvironment{proof}{\begin{addmargin}[1em]{0em}\begin{newproof}}{\end{newproof}\end{addmargin}\qed}
% \newcommand{\expl}[1]{\text{\hfill[#1]}$}


\newenvironment{lemma}[2][Lemma]{\begin{trivlist}
		\item[\hskip \labelsep {\bfseries #1}\hskip \labelsep {\bfseries #2.}]}{\end{trivlist}}
\newenvironment{problem}[2][Problem]{\begin{trivlist}
		\item[\hskip \labelsep {\bfseries #1}\hskip \labelsep {\bfseries #2.}]}{\end{trivlist}}
\newenvironment{example}[2][Example]{\begin{trivlist}
		\item[\hskip \labelsep {\bfseries #1}\hskip \labelsep{\bfseries #2.}]}{\end{trivlist}}
\newenvironment{solution}{{\noindent \bfseries Solution\hskip 2ex}}{\qed}
\newenvironment{exercise}[2][Exercise]{\begin{trivlist}
		\item[\hskip \labelsep {\bfseries #1}\hskip \labelsep {\bfseries #2.}]}{\end{trivlist}}
\newenvironment{reflection}[2][Reflection]{\begin{trivlist}
		\item[\hskip \labelsep {\bfseries #1}\hskip \labelsep {\bfseries #2.}]}{\end{trivlist}}
\newenvironment{proposition}[2][Proposition]{\begin{trivlist}
		\item[\hskip \labelsep {\bfseries #1}\hskip \labelsep {\bfseries #2.}]}{\end{trivlist}}
\newenvironment{corollary}[2][Corollary]{\begin{trivlist}
		\item[\hskip \labelsep {\bfseries #1}\hskip \labelsep {\bfseries #2.}]}{\end{trivlist}}

%%%% add package inclusions here, if any

%%%%% Define your custom functions and macros here, if any

%%%%%%%%%%%%%%%%%%%% Define the variables below %%%%%%%%%%%%%%%%%%%%%%%%%%%%%%
\def\titlestring{The Lecture Title}
\def\scribestring{Your Name}
\def\datestring{Day, Mon, Date Year}


%%%%%%%%%%%%%%%%%%% Page Headers -- Do Not Change %%%%%%%%%%%%%%%%%%%%%%%%%%%
\lhead{\titlestring}
\rhead{Page \thepage}
\cfoot{Concrete Math -- White -- TJHSST}
\renewcommand{\headrulewidth}{0.4pt}
\renewcommand{\footrulewidth}{0.4pt}

%%%%%%%%%%%%%%%%% Begin the document %%%%%%%%%%%%%%%%%%%%%%%%%
\begin{document}
\thispagestyle{plain}  %% no headers on this page

%%%% Do not change these lines %%%%%
\noindent
{\LARGE \textbf{\titlestring}}\\\\
%
{\Large Scribe: \scribestring}\\ \\
{\textbf{Date}: \datestring}


%%%%%%%%%%%%%%%%%%%%%%%%%% YOUR CONTENT GOES HERE %%%%%%%%%%%%%%%%%%%%%%%%%%
\noindent

\section{Recursive Problems and Non-Homogeneous Difference Equations}
\begin{problem}{1}
Draw $n$ lines in a plane. What is the maximum number of subregions created?
\end{problem}

\begin{problem}{2}
How many sequences of $\{ 0, 1, 2, 3 \}^n$ contain an even number of zeroes? 
\end{problem}

\begin{solution}{1}
We can do this using recursion. Let $a_n$ be the maximum number of subregions created. Notice that for $n=1$, $a_n = 2$. In the general case, in order to maximize the number of regions, each new added line (the $n$th line) must intersect the other $n-1$ lines already present. This creates $n$ new regions, the same as the number of "spaces" between the lines.

Altogether, this gives the recurrence 
\[
	a_n = a_{n-1} + n
\]
with our base case, can be solved to yield 
\[
	a_n = \frac{n(n+1)}{2} + 1
\]
\end{solution}

\begin{solution}
	Notice that if we truncate the last digit from a valid string, we can either have a string of length $n-1$ with an even number of zeroes or a string with an odd number of zeroes. Let $o_n$ be the number of strings of length $n$ with an odd number of zeroes, and 
	

\end{solution}


How do we solve these recurrences explicitly (without silly tricks)? 

Given $\sum_{i=0}^n c_i x_i = f$, a constant-coefficient linear recurrence equation that is non-homogeneous, because it has the "forcing function" $f$. We will solve this by looking at the particular and homogeneous equations. 
The \textbf{particular} solution looks like the following: 
\[
	(p): \quad \sum_{i=0}^n c_i a_i^{(p)} = f
\]
and the \textbf{homogeneous} solution looks like the following: 
\[
	(h): \quad \sum_{i=0}^n c_i a_i^{(h)} = 0
\]
We will find both $a_i^{(p)}$ and $a_i^{(h)}$ and finally give the answer $a_i = a_i^{(h)} + a_i^{(p)}$ and solve for initial conditions. Notice that 
$\sum c_i(a_i^{(h)} + a_i^{(p)}) = f$, as $\sum c_ia_i^{(h)} = 0$ - we just solve this with the initial conditions to yield the correct solution. 

\begin{problem}{3}
	Solve $a_n + 2a{n-1} = n+3$, $a_0 = 3$. 
\end{problem}

\begin{solution}
	First, we solve the homogeneous equation: 
\[
	a_n + 2a_{n-1} = 0 \implies x + 2 = 0, \quad x = -2
\]		
This gives $a_n^{(h)} = c(-2)^n$. 

We now solve the particular equation using undetermined coefficients
\[
	a_n + 2a_{n-1} = n + 3
\]
We will ansatz that $a_n = Bn + D$. If we do this, we then must have $a_{n-1} = Bn + D - B$. When plugging in we get: 
\[
	Bn + D + 2Bn + 2D - 2B = n+3
\]
Equating like coefficients, we have
\[
	3Bn = n \quad 3D - 2B = 3
\]
This implies that $B = \frac{1}{3}$, $D = \frac{11}{9}$, so $a_n^{(p)} = \frac{1}{3}n + \frac{11}{9}$. 
Altogether, we must have $a_n = \frac{1}{3}n + \frac{11}{9} + C (-2)^n$. Notice that this \textbf{always} satisfies the total recurrence, as this both satisfies the homogeneous equation and the particular equation. When we plug in $n = 0$, we get $C = \frac{16}{9}$. Therefore, 
\[
	a_n = \frac{1}{3}n + \frac{11}{9} + \frac{16}{9} (-2)^n
\]

\end{solution}

How do we know what to try? In general, choose something of the same type. 

Solve these nonhomogeneous linear recurrences with this method: 

\begin{solution}
$a_n = 4^{n-1} + 2a_{n-1}$, $a_1 = 3$. 

\textit{Homogeneous Solution: }
\[
	a_n - 2a_{n-1} = 0 \implies x - 2 = 0, x = 2
\]
\[
	\implies a_n^{(h)} = C2^n 
\]
\textit{Particular Solution: }
\[
	a_n - 2a_{n-1} = 4^{n-1} 
\]
Ansatz $a_n = A4^{n}$. 
\[
	A4^n - 2A4^{n-1} = 4^{n-1} \rightarrow 2A = 1 \rightarrow A = \frac{1}{2}
\]
\[
	a_n = C2^n + \frac{1}{2}4^n 
\]
\[
	a_1 = 3 = 2C + 2 \implies C = \frac{1}{2}
\]
\[
	a_n = \frac{1}{2}(4^n + 2^n)
\]
\end{solution}

\begin{solution}
$a_n = a_{n-1} + n$, $a_1 = 2$: 

\textit{Homogeneous Solution: }
\[
	a_n - a_{n-1} = 0 \implies x - 1 = 0, x = 1
\]
\[
	\implies a_n^{(h)} = C 
\]
\textit{Particular Solution: }
\[
	a_n = a_{n-1} + n 
\]
The ansatz $a_n = Cn + D$ does not work, but $a_n = Cn^2 + Dn + E$ works: 
\[
	Cn^2 + Dn + E = C(n-1)^2 + D(n-1) + E + n
\]
\[
	0 = -2Cn + C - D + n \rightarrow C = \frac{1}{2}, D = \frac{1}{2}
\]
\[
	a_n = \frac{1}{2}n^2 + \frac{1}{2}n + C
\]
\[
	a_1 = \frac{1}{2} + \frac{1}{2} + C = 2 \rightarrow C = 1
\]
\[
	a_n =  \frac{1}{2}n^2 + \frac{1}{2}n + 1 = \frac{n(n+1)}{2} + 1
\]
\end{solution}
\end{document}
