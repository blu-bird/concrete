%%%%%%% Lecture Notes Style for Concrete Mathematics %%%%%%%%%%%%%
%%%%%%% Taught by Patrick White, TJHSST %%%%%%%%%%%%%%%%%%%%%%%%%%

\documentclass[11pt,twosided]{article}
%%%%%%%%%%%%%%%%%%%%%%%%%%%%%%%%%%%%%%%%%%%%%%%%%%%%%%%
%%%% HEADER FILE for CONCRETE MATH LECTURE NOTES %%%%%%
%%%%%%%%%%%%%%%%%%%%%%%%%%%%%%%%%%%%%%%%%%%%%%%%%%%%%%%

%%%%%%%%%%%%%%%%%%%%%%%%%%%%%%%%%%%%%%%%%%%%%%%%%%%%%%%%%%%%%%%%%%%
%% This file is included at the top of every lecture notes file %%%
%% It should ONLY be changed by the instructor %%
%%%%%%%%%%%%%%%%%%%%%%%%%%%%%%%%%%%%%%%%%%%%%%%%%%%%%%%%%%%%%%%%%%%

%%%%%%%%%%%%%%%%%%%%%% package inclusions %%%%%%%%%%%%%%%%%%%%
%\usepackage[
%top=2cm,
%bottom=2cm,
%left=3cm,
%right=2cm,
%headheight=17pt, % as per the warning by fancyhdr
%includehead,includefoot,
%heightrounded, % to avoid spurious underfull messages
%]{geometry} 


\usepackage{amsmath,amssymb}
\usepackage{mathtools}
\usepackage{amsthm}

\usepackage{fancyhdr}
\pagestyle{fancy}


%%%%%%%%%%%%%%% theorem style definitions %%%%%%%%%%%%%%%%%%%%%%
\newtheorem{theorem}{Theorem}[section]
%\newtheorem{corollary}{Corollary}[theorem]
%\newtheorem{lemma}[theorem]{Lemma}
\newtheorem*{remark}{Remark}
%\newtheorem{example}{Example}[section]
\newtheorem*{definition}{Definition}


%%%%%%%%%%%%%%%% custom function definitions %%%%%%%%%%%%%%%%%%%%%%%%%
%%%%%%%%%%% as class progresses, this section may be enhanced %%%%%%%%

\def\multiset#1#2{\ensuremath{\left(\kern-.3em\left(\genfrac{}{}{0pt}{}{#1}{#2}\right)
		\kern-.3em\right)}} %%% multiset notation
\newcommand\rf[2]{{#1}^{\overline{#2}}} %%%% rising factorial \rf{x}{m}
\newcommand\ff[2]{{#1}^{\underline{#2}}} %%%%% falling factorial \ff{x}{m}


%%%%%%%
% book stuff
%%%%%%%%%

\usepackage[twoside,outer=1.5in,inner=2in,bottom=1.5in,top=1.5in,marginpar=2in]{geometry} 
\usepackage{amsmath,amsthm,amssymb,scrextend}
\usepackage{fancyhdr}
\usepackage{palatino}
\usepackage{multicol}
%\usepackage{background}
\usepackage{graphicx}
\usepackage{listliketab}

\usepackage{lipsum}
\usepackage{caption}
\usepackage{marginnote}

\reversemarginpar


\usepackage[dvipsnames]{xcolor}
\usepackage{amsthm}
\usepackage{shadethm}

\newcommand*\Note{%
	\marginnote[\textcolor{blue}{\raggedright{\LARGE ?}\ Note }]{}%
}
\newcommand*\Warn{%
	\marginnote[\textcolor{red}{\raggedright{\LARGE !}\ Warning }]{}%
}
\reversemarginpar

\newcommand{\N}{\mathbb{N}}
\newcommand{\Z}{\mathbb{Z}}
\newcommand{\I}{\mathbb{I}}
\newcommand{\R}{\mathbb{R}}
\newcommand{\Q}{\mathbb{Q}}
\renewcommand{\qed}{\hfill$\blacksquare$}
\let\newproof\proof
\renewenvironment{proof}{\begin{addmargin}[1em]{0em}\begin{newproof}}{\end{newproof}\end{addmargin}\qed}
% \newcommand{\expl}[1]{\text{\hfill[#1]}$}


\newenvironment{lemma}[2][Lemma]{\begin{trivlist}
		\item[\hskip \labelsep {\bfseries #1}\hskip \labelsep {\bfseries #2.}]}{\end{trivlist}}
\newenvironment{problem}[2][Problem]{\begin{trivlist}
		\item[\hskip \labelsep {\bfseries #1}\hskip \labelsep {\bfseries #2.}]}{\end{trivlist}}
\newenvironment{example}[2][Example]{\begin{trivlist}
		\item[\hskip \labelsep {\bfseries #1}\hskip \labelsep{\bfseries #2.}]}{\end{trivlist}}
\newenvironment{solution}{{\noindent \bfseries Solution\hskip 2ex}}{\qed}
\newenvironment{exercise}[2][Exercise]{\begin{trivlist}
		\item[\hskip \labelsep {\bfseries #1}\hskip \labelsep {\bfseries #2.}]}{\end{trivlist}}
\newenvironment{reflection}[2][Reflection]{\begin{trivlist}
		\item[\hskip \labelsep {\bfseries #1}\hskip \labelsep {\bfseries #2.}]}{\end{trivlist}}
\newenvironment{proposition}[2][Proposition]{\begin{trivlist}
		\item[\hskip \labelsep {\bfseries #1}\hskip \labelsep {\bfseries #2.}]}{\end{trivlist}}
\newenvironment{corollary}[2][Corollary]{\begin{trivlist}
		\item[\hskip \labelsep {\bfseries #1}\hskip \labelsep {\bfseries #2.}]}{\end{trivlist}}



%%%% add package inclusions here, if any

%%%%% Define your custom functions and macros here, if any
\newcommand{\fallfac}[2]{{#1}^{\underline{#2}}}

%%%%%%%%%%%%%%%%%%%% Define the variables below %%%%%%%%%%%%%%%%%%%%%%%%%%%%%%
\def\titlestring{Generating Functions - The Catalan Numbers}
\def\scribestring{Sophia Pentakalos and Sophia Wang}
\def\datestring{Thursday, March, 21st 2019}


%%%%%%%%%%%%%%%%%%% Page Headers -- Do Not Change %%%%%%%%%%%%%%%%%%%%%%%%%%%
\lhead{\titlestring}
\rhead{Page \thepage}
\cfoot{Concrete Math -- White -- TJHSST}
\renewcommand{\headrulewidth}{0.4pt}
\renewcommand{\footrulewidth}{0.4pt}

%%%%%%%%%%%%%%%%% Begin the document %%%%%%%%%%%%%%%%%%%%%%%%%
\begin{document}
\thispagestyle{plain}  %% no headers on this page

%%%% Do not change these lines %%%%%
\noindent
{\LARGE \textbf{\titlestring}}\\\\
%
{\Large Scribe: \scribestring}\\ \\
{\textbf{Date}: \datestring}


%%%%%%%%%%%%%%%%%%%%%%%%%% YOUR CONTENT GOES HERE %%%%%%%%%%%%%%%%%%%%%%%%%%
\noindent

\section{Unlabeled Probability}

\begin{problem}
1 You throw two marbles into two buckets. Find the probability that they both land in the same bucket: 

\begin{itemize}
    \item Unlabeled into unlabeled:  1/2 (2 scenarios: marbles in separate buckets or together)
    \item Labeled into unlabeled: 1/2 (2 scenarios: marbles in separate buckets or together)
    \item Labeled into labeled: 1/2 (4 scenarios - 2 of which include both marbles in the same bucket)
    
    \item Unlabeled into labeled: 
    
    Let's label buckets A and B. You have two indistinguishable marbles, and so there are three scenarios: marbles are in separate buckets, both marbles are in A, and both marbles in B. Therefore, you may incorrectly conclude that the probability is 2/3, when in reality it remains 1/2. The probability of marbles being in separate buckets is twice the odds that both marbles are in A. 
    \newline \newline
    This is because \textbf{probability problems are ALWAYS labeled.}
    \newline \newline
    If you label the marbles 1 and 2, there are two ways for the marbles to be in separate buckets, each as probable as any of the other scenarios: 1-A and 2-B or 1-B and 2-A. 
    
    When you are asked for the probability, always treat both the balls and urns, or marbles and buckets, as labeled into labeled. 

\end{itemize}

\end{problem}

\section{Generating Function for Catalan Numbers}

Recall:

$$C_n = \sum_{i=0}^{n-1} C_i C_{n-1-i} $$ $$ C_0 = C_1 = 1$$

Or in other words: $C_{n+1} = \sum_{i=0}^{n} C_i C_{n-1}$

\bigbreak

\bigbreak

\noindent
Let's find a mathematical, instead of combinatorial, formula for $C_n$:

\smallbreak

\noindent Define $f(x) = \sum_{n=0}^{\infty} C_n x^n$

\medbreak


\noindent Notice that $f(x) = C_0 + C_1x + C_2x^2 + C_3x^3 + ...$

\medbreak
${f(x)}^2 = C_0^2 + (C_0C_1 + C_1C_0)x + (C_0C_2 + C_1C_1 + C_2C_0) x^2 + ...$ 

\smallbreak
$= C_1 + C_2x + C_3x^2 + C_4x^3+ ... $

\medbreak

$x{f(x)}^2 = C_1x + C_2x^2 + C_3x^3 
+ C_4x^4 + ... $

\smallbreak

$x{f(x)}^2 = f(x) - C_0 = f(x) - 1$

\smallbreak

$x{f(x)}^2 - f(x) + 1 = 0$

\smallbreak
\noindent Now, use the quadratic formula to get: 

$f(x) = \frac{1\pm \sqrt{1-4x}}{2x}$

\noindent 
However, we need to express this in series notation, in order to directly know the nth coefficient.

\medbreak

\noindent Recall from p.20 of Unit 1, Section 3.2:

\smallbreak

\noindent $(1-x)^\frac{1}{2} = \frac{\sum \fallfac{\frac{1}{2}}{k}}{k!} x^k$

\medbreak
$= 1 - \frac{1}{2}\sum_{n=0}^{\infty}\binom{2n}{n} \frac{1}{n+1}\frac{1}{4^n} x^{n+1}$

\medbreak

\noindent So, $(1-4x)^{\frac{1}{2}} = 1 - \frac{1}{2}\sum_{n=0}^{\infty}\binom{2n}{n} \frac{1}{n+1}\frac{1}{4^n} x^{n+1}$

\smallbreak
$(1-4x)^{\frac{1}{2}} = 1 - 2 \sum_{n=0}^{\infty}\binom{2n}{n} \frac{1}{n+1} x^{n+1}$ 

\medbreak

\noindent
Based on what we previously found, $(1-4x)^{\frac{1}{2}} = 1 - 2 \sum_{n=0}^{\infty}\binom{2n}{n} \frac{1}{n+1} x^{n+1}$ , we know to choose the solution of $f(x) = \frac{1 - \sqrt{1-4x}}{2x}$.

\medbreak

\noindent Now, we simplify, to get that the generating function for the Catalan Numbers is:
$$\fbox{f(x) = \sum_{n=0}^{\infty} \binom{2n}{n} \frac{1}{n+1} x^n}$$



\end{document}
