%%%%%%% Lecture Notes Style for Concrete Mathematics %%%%%%%%%%%%%
%%%%%%% Taught by Patrick White, TJHSST %%%%%%%%%%%%%%%%%%%%%%%%%%

\documentclass[11pt,twosided]{article}
%%%%%%%%%%%%%%%%%%%%%%%%%%%%%%%%%%%%%%%%%%%%%%%%%%%%%%%
%%%% HEADER FILE for CONCRETE MATH LECTURE NOTES %%%%%%
%%%%%%%%%%%%%%%%%%%%%%%%%%%%%%%%%%%%%%%%%%%%%%%%%%%%%%%

%%%%%%%%%%%%%%%%%%%%%%%%%%%%%%%%%%%%%%%%%%%%%%%%%%%%%%%%%%%%%%%%%%%
%% This file is included at the top of every lecture notes file %%%
%% It should ONLY be changed by the instructor %%
%%%%%%%%%%%%%%%%%%%%%%%%%%%%%%%%%%%%%%%%%%%%%%%%%%%%%%%%%%%%%%%%%%%

%%%%%%%%%%%%%%%%%%%%%% package inclusions %%%%%%%%%%%%%%%%%%%%
%\usepackage[
%top=2cm,
%bottom=2cm,
%left=3cm,
%right=2cm,
%headheight=17pt, % as per the warning by fancyhdr
%includehead,includefoot,
%heightrounded, % to avoid spurious underfull messages
%]{geometry} 


\usepackage{amsmath,amssymb,amsfonts}
\usepackage{longtable}
\usepackage{mathtools}
\usepackage{amsthm}
\usepackage{enumerate}
\usepackage{fancyhdr}
\pagestyle{fancy}


%%%%%%%%%%%%%%% theorem style definitions %%%%%%%%%%%%%%%%%%%%%%
\newtheorem{theorem}{Theorem}[section]
%\newtheorem{corollary}{Corollary}[theorem]
%\newtheorem{lemma}[theorem]{Lemma}
\newtheorem*{remark}{Remark}
%\newtheorem{example}{Example}[section]
\newtheorem*{definition}{Definition}


%%%%%%%%%%%%%%%% custom function definitions %%%%%%%%%%%%%%%%%%%%%%%%%
%%%%%%%%%%% as class progresses, this section may be enhanced %%%%%%%%

\def\multiset#1#2{\ensuremath{\left(\kern-.3em\left(\genfrac{}{}{0pt}{}{#1}{#2}\right)
		\kern-.3em\right)}} %%% multiset notation
\newcommand\rf[2]{{#1}^{\overline{#2}}} %%%% rising factorial \rf{x}{m}
\newcommand\ff[2]{{#1}^{\underline{#2}}} %%%%% falling factorial \ff{x}{m}
\newcommand{\Perm}[2]{{}^{#1}\!P_{#2}} % permutation
\newcommand{\half}{\ensuremath{\frac{1}{2}}}
\newcommand{\braces}[1]{\left\{#1\right\}}
\newcommand{\set}[1]{\braces{#1}}
\newcommand{\snb}[2]{\ensuremath{\left(\kern-.3em\left(\genfrac{}{}{0pt}{}{#1}{#2}\right)\kern-.3em\right)}}
\newcommand{\fallingfactorial}[1]{^{\underline{#1}}}
\newcommand{\fallfac}[2]{{#1}^{\underline{#2}}}
\newcommand{\stirlingone}[2]{\genfrac[]{0pt}{1}{#1}{#2}}
\newcommand{\stiri}[2]{\stirlingone{#1}{#2}}
\newcommand{\dstirlingone}[2]{\genfrac[]{0pt}{0}{#1}{#2}}
\newcommand{\dstiri}[2]{\dstirlingone{#1}{#2}}
\newcommand{\stirlingtwo}[2]{\genfrac\{\}{0pt}{1}{#1}{#2}}
\newcommand{\stirii}[2]{\stirlingtwo{#1}{#2}}
\newcommand{\dstirlingtwo}[2]{\genfrac\{\}{0pt}{0}{#1}{#2}}
\newcommand{\dstirii}[2]{\dstirlingtwo{#1}{#2}}
\newcommand{\ol}[1]{\overline{#1}}
%%%%%%%
% book stuff
%%%%%%%%%

\usepackage[twoside,outer=1.5in,inner=2in,bottom=1.5in,top=1.5in,marginpar=2in]{geometry} 
\usepackage{scrextend}
\usepackage{palatino}
\usepackage{multicol}
\usepackage{float}
\usepackage[hidelinks]{hyperref}
\usepackage[nameinlink, capitalise, noabbrev]{cleveref}
%\usepackage{background}
\usepackage{graphicx}
\usepackage{listliketab}

\usepackage{lipsum}
\usepackage{caption}
\usepackage{marginnote}

\reversemarginpar


\usepackage[dvipsnames]{xcolor}
\usepackage{amsthm}
\usepackage{shadethm}

\setlength{\parindent}{0pt}

\newcommand*\Note{%
	\marginnote[\textcolor{blue}{\raggedright{\LARGE ?}\ Note }]{}%
}
\newcommand*\Warn{%
	\marginnote[\textcolor{red}{\raggedright{\LARGE !}\ Warning }]{}%
}
\reversemarginpar

\newcommand{\N}{\mathbb{N}}
\newcommand{\Z}{\mathbb{Z}}
\newcommand{\I}{\mathbb{I}}
\newcommand{\R}{\mathbb{R}}
\newcommand{\Q}{\mathbb{Q}}
\renewcommand{\qed}{\hfill$\blacksquare$}
\let\newproof\proof
\renewenvironment{proof}{\begin{addmargin}[1em]{0em}\begin{newproof}}{\end{newproof}\end{addmargin}\qed}
% \newcommand{\expl}[1]{\text{\hfill[#1]}$}


\newenvironment{lemma}[2][Lemma]{\begin{trivlist}
		\item[\hskip \labelsep {\bfseries #1}\hskip \labelsep {\bfseries #2.}]}{\end{trivlist}}
\newenvironment{problem}[2][Problem]{\begin{trivlist}
		\item[\hskip \labelsep {\bfseries #1}\hskip \labelsep {\bfseries #2.}]}{\end{trivlist}}
\newenvironment{example}[2][Example]{\begin{trivlist}
		\item[\hskip \labelsep {\bfseries #1}\hskip \labelsep{\bfseries #2.}]}{\end{trivlist}}
\newenvironment{solution}{{\noindent \bfseries Solution\hskip 2ex}}{\qed}
\newenvironment{exercise}[2][Exercise]{\begin{trivlist}
		\item[\hskip \labelsep {\bfseries #1}\hskip \labelsep {\bfseries #2.}]}{\end{trivlist}}
\newenvironment{reflection}[2][Reflection]{\begin{trivlist}
		\item[\hskip \labelsep {\bfseries #1}\hskip \labelsep {\bfseries #2.}]}{\end{trivlist}}
\newenvironment{proposition}[2][Proposition]{\begin{trivlist}
		\item[\hskip \labelsep {\bfseries #1}\hskip \labelsep {\bfseries #2.}]}{\end{trivlist}}
\newenvironment{corollary}[2][Corollary]{\begin{trivlist}
		\item[\hskip \labelsep {\bfseries #1}\hskip \labelsep {\bfseries #2.}]}{\end{trivlist}}

%%%% add package inclusions here, if any
\usepackage{ textcomp }
\usepackage{amsmath}
%%%%% Define your custom functions and macros here, if any

%%%%%%%%%%%%%%%%%%%% Define the variables below %%%%%%%%%%%%%%%%%%%%%%%%%%%%%%
\def\titlestring{Generating Functions \& Solving Recurrences}
\def\scribestring{Ritika Shrivastav, Anupama Jayaraman}
\def\datestring{19, March, 2019}


%%%%%%%%%%%%%%%%%%% Page Headers -- Do Not Change %%%%%%%%%%%%%%%%%%%%%%%%%%%
\lhead{\titlestring}
\rhead{Page \thepage}
\cfoot{Concrete Math -- White -- TJHSST}
\renewcommand{\headrulewidth}{0.4pt}
\renewcommand{\footrulewidth}{0.4pt}
\newcommand{\bracenom}{\genfrac{\lbrace}{\rbrace}{0pt}{}}
%%%%%%%%%%%%%%%%% Begin the document %%%%%%%%%%%%%%%%%%%%%%%%%
\begin{document}
\thispagestyle{plain}  %% no headers on this page

%%%% Do not change these lines %%%%%
\noindent
{\LARGE \textbf{\titlestring}}\\\\
%
{\Large Scribe: \scribestring}\\ \\
{\textbf{Date}: \datestring}


%%%%%%%%%%%%%%%%%%%%%%%%%% YOUR CONTENT GOES HERE %%%%%%%%%%%%%%%%%%%%%%%%%%
\noindent

\section{Unit 1 Proof}
%Do proof
\begin{problem} 1 Prove $\sum_{j = 0}^n \bracenom{n}{k}\begin{bmatrix} 
j \\
k
\end{bmatrix} = \delta(n,k)$ \end{problem}

\begin{solution}
We are given:

1. $x^n = \sum_{j = 0}^n \bracenom{n}{j}x^\underline{j}$
\nextline

2. $x^\underline{j}$   $= \sum_{k = 0}^j \begin{bmatrix} 
j \\
k
\end{bmatrix}x^k$ $=$ $ \sum_{k = 0}^n \begin{bmatrix} 
j \\
k
\end{bmatrix}x^k$ 
\newline
For this second part, we can increase the upper bound because all values for which $n > j$ are just equal to 0.
\newline 
Plugging Equation 2 into Equation 1, we get...
\newline
$x^n$ = $\sum_{j = 0}^n \bracenom{n}{j}$ $\sum_{k = 0}^n \begin{bmatrix} 
j \\
k
\end{bmatrix}x^k$
\newline
Because j is independent of the inner sum, we can actually switch the order and move both sums to encapsulate the entire expression.
\newline
$x^n$ = $\sum_{j = 0}^n \sum_{k = 0}^n \bracenom{n}{j}$ $ \begin{bmatrix} 
j \\
k
\end{bmatrix}x^k$
\newline
The sum basically represents adding the rows of a matrix vs adding the columns of a matrix, which basically proves the statement and we get...
$\sum_{j = 0}^n \bracenom{n}{k}\begin{bmatrix} 
j \\
k
\end{bmatrix} = \delta(n,k)$ 
\end{solution}
\section{Applications of Generating Functions}

\subsection{A food counting problem}
\begin{problem}1 Fill a bag with $n$ pieces of fruit \underline{GIVEN}: The number of apples is even, the number of bananas is a multiple of 5, the number of oranges is at most 4, and the number of pears is at most 1. How many ways can I do this? \end{problem}
\begin{solution}
\\To solve this problem, let's find find a generating function that represents the situation. To do so, let's breakdown the function into its apples, bananas, oranges, and pears terms:\\ \\
Apples: We want 0, 2, 4, ... apples. We can write this as $(x^0 + x^2 + x^4 + x^6 +...) = (1 + x^2 + x^4 + x^6 +...)$. This is also equal to $\frac{1}{1-x^2}$.\\\\
Bananas: We want 0, 5, 10, ... oranges. We can write this as $(1 + x^5 + x^10 + x^15 +...)$, or $\frac{1}{1-x^5}$.\\\\
Oranges: We want 0, 1, 2, 3, or 4 oranges. We can write this as $(1 + x + x^2 + x^3 + x^4)$. To write this as a fraction, let's make use of the formula for a truncated series: $(1-x^{n+1}) = (1-x)(1+x+x^2+...+x^n)$. From this formula, we can see that we can write the number oranges as $\frac{1-x^5}{1-x}$. \\\\
Pears: We want either 0 or 1 pear, or $(1+x)$.\\\\\
If we multiply all of the components together, we get: \\\\
$g(x) = (1 + x^2 + x^4 + x^6 +...)(1 + x^5 + x^10 + x^15 +...)(1 + x + x^2 + x^3 + x^4)(1+x) = \frac{1}{1-x^2}*\frac{1}{1-x^5}*\frac{1-x^5}{1-x}*(1+x) = \frac{1}{(1-x)^2}$ \\\\
$\frac{1}{(1-x)^2}$ has the form of the summation of a multiset: $\sum_{k = 0}^\infty \multiset{2}{k}x^k = \sum_{k = 0}^\infty (k+1)x^k$. To find the number of ways to fill the bag with fruit, we simply find the coefficient of $x^n$ in the expansion of the summation. In other words, there are:\\\\
\framebox{$(n+1)$ ways}.

\end{solution}

\subsection{Making Change for \$1}
\begin{problem}2 Write a generating function for making change given 1\textcent{}, 5\textcent{}, 10\textcent{}, 25\textcent{}, and 50\textcent{} coins. 
\end{problem}
\begin{solution} 
\\As with the fruit problem, let's create a generating function for the scenario. We will be working in terms of cents.\\\\
1\textcent{}: $(1+x+x^2+x^3+...)=\frac{1}{1-x}$ \\\\
5\textcent{}: $(1+x^5+x^10+...)=\frac{1}{1-x^5}$\\\\
10\textcent{}: $(1+x^10+x^20+...)=\frac{1}{1-x^{10}}$\\\\
25\textcent{}: $(1+x^25+x^50+...)=\frac{1}{1-x^{25}}$\\\\
50\textcent{}: $(1+x^50+x^100+...)=\frac{1}{1-x^{50}}$\\\\
By multiplying all of the terms together, we have the generating function: \\\\
$g(x)=\frac{1}{(1-x)(1-x^5)(1-x^{10})(1-x^{25})(1-x^{50})}$ \\\\
To find the number of ways we can make change for \$1, we find the coefficient of $x^{100}$. This is \framebox{293}.

What if there are coins available in every single denomination?
Then we have... \\\\
1\textcent{}: $(1+x+x^2+x^3+...)=\frac{1}{1-x}$ \\\\
2\textcent{}: $(1+x^2+x^4+...)=\frac{1}{1-x^2}$\\\\
3\textcent{}: $(1+x^3+x^6+...)=\frac{1}{1-x^3}$\\\\
4\textcent{}: $(1+x^4+x^8+...)=\frac{1}{1-x^4}$\\\\
n\textcent{}: $(1+x^n+x^2n0+...)=\frac{1}{1-x^n}$\\\\
By multiplying all of the terms together, we have the generating function: \\\\
$g(x)=\frac{1}{(1-x)(1-x^2)(1-x^3)(1-x^4)(1-x^n)}$ \\\\
This is also the generating function for integer partitions. \\\\
$g(x) = \prod_{k=0}^n \frac{1}{1-x^k} = \sum_{r>0} P(r)x^r $ \\\\
P(r) is the is the number of integer partitions of r.
\end{solution}

%extension --> every denomination
\\

\begin{theorem}The number of partitions of n into distinct parts $r$ is the number of partitions of $n$ into odd parts.\end{theorem}
\\\\
Distinct parts is when you split the number into parts but the sum doesn't include 2 of the same number. For example, $8 = 1 + 7$ is distinct because 1 and 7 are different. However $8 = 4 + 2 + 2$ is not because you have two 2's. \\\\
Odd parts are those that are made of a repeated sum of only odd numbers, that can be repeated as many times as one pleases. For example, $8 = 3 + 3 + 1 + 1$ is fine because it contains only odds. However, $8 = 3 + 3 + 2$ is not because 2 is an even number.\\\\
Generating Functions:\\\\
Distinct parts: $g(x) = (1+x)(1+x^2)(1+x^3)...$\\\\
Odd parts: $g(x) = (\frac{1}{1-x})(\frac{1}{1-x^3})(\frac{1}{1-x^5})...$\\\\
Claim they are identical. So set both generating functions equal and use the trick that $1+x = \frac{1-x^2}{1-x}$ \\\\
$(1+x)(1+x^2)(1+x^3)... = (\frac{1}{1-x})(\frac{1}{1-x^3})(\frac{1}{1-x^5})...$\\\\
$(\frac{1-x^2}{1-x})(\frac{1-x^4}{1-x^2})(\frac{1-x^6}{1-x^3})... = (\frac{1}{1-x})(\frac{1}{1-x^3})(\frac{1}{1-x^5})...$\\\\
We notice that term cancel neatly to provide that the two sides are directly equal and therefore this theorem has been proved.

\section{Solving Recurrence Equations}

\begin{problem} 3
Given $a_n = a_{n-1} + 8a_{n-2} - 12a_{n-3}$; $a_0 = 2, a_1 = 3$, $a_2 = 19$, find the generating function $g(x) = \sum_{n = 0}^\infty a_nx^n$ and formula for $a_n$.
\end{problem}
\begin{solution}

\noindent Let's start with $g(x) = \sum_{n = 0}^\infty a_nx^n$\\\\
\noindent
Since we know the first three terms from our initial condition, we can add up our base cases and the summation of $a_n$ from $n=3$ to $n=\infty$. \\

$g(x) = 2 + 3x + 19x^2 + \sum_{n = 3}^\infty a_nx^n$ \\

\noindent
We know the recurrence relation so we can substitute it into the equation we have above\\

$g(x) = 2 + 3x + 19x^2 + \sum_{n = 3}^\infty (a_{n-1} + 8a_{n-2} - 12a_{n-3})x^n$\\

\noindent
and expand it properly\\

$g(x) = 2 + 3x + 19x^2 + \sum_{n = 3}^\infty a_{n-1}x^n + \sum_{n = 3}^\infty 8a_{n-2}x^n - \sum_{n = 3}^\infty 12a_{n-3}x^n$ \\

\noindent
Now, we need to make some manipulations. First, we want the subscript of each of the $a_n$'s to match the power to which each x is being raised. To do so, let's factor out the coefficients and the appropriate number of $x$'s from each term.\\

$g(x) = 2 + 3x + 19x^2 + x\sum_{n = 3}^\infty a_{n-1}x^{n-1} + 8x^2\sum_{n = 3}^\infty a_{n-2}x^{n-2} - 12x^3\sum_{n = 3}^\infty a_{n-3}x^{n-3}$ \\

\noindent
The second manipulation involves shifting the indices such that the term within the summation is $a_nx^n$. \\

$g(x) = 2 + 3x + 19x^2 + x\sum_{n = 2}^\infty a_{n}x^{n} + 8x^2\sum_{n = 1}^\infty a_{n}x^{n} - 12x^3\sum_{n = 0}^\infty a_{n}x^{n}$ \\

\noindent
Now that we have the new summations in terms of $a_nx^n$, they can all be rewritten using the original $g(x)$ function minus the initial terms that are excluded from the summation:\\

$g(x) = 2 + 3x + 19x^2 + x(g(x) - a_0 - a_1x) + 8x^2(g(x) - a_0) - 12x^3g(x)$
\\ \indent \indent $ = 2 + 3x + 19x^2 + x(g(x) - 2 - 3x) + 8x^2(g(x) - 2) - 12x^3g(x)$ \\

\noindent
Given this equation, we can finally solve for $g(x)$ by moving terms around. After we do this, we can see that our final equation for $g(x)$ is \\

\framebox{\textbf{$g(x) = \frac{2 + x}{1 - x - 8x^2 - 12x^3}$}}\\

\noindent\normalsize
\textbf{**Notice}: \textit{If we move all of the terms in the original $a_n$ equation to one side, we obtain $a_n - a_{n-1} - 8a_{n-2} + 12a_{n-3} = 0$. This resembles the denominator of our final $g(x)$ equation.} 

\noindent
Now let's find the formula for $a_n$. With our equation for $g(x)$, we can factor the denominator into $(1 + 3x)(1 - 2x)^2$ to determine an equation for $a_n$. More specifically, we can use \textit{partial fraction decomposition} to breakdown the function into terms that will be more useful to us.\\

$g(x) = \frac{2 + x}{1 - x - 8x^2 - 12x^3} = \frac{A}{1 + 3x} + \frac{B}{1 - 2x} + \frac{C}{(1-2x)^2}$ \\

\noindent
To conduct partial fraction decomposition, we multiply the denominator of $g(x)$ on both sides of the equation.\\

$2 + x = A(1-2x)^2 + B(1 + 3x)(1 - 2x) + C(1 + 3x)$\\

\noindent
Once we solve this equation, by plugging in different values for $x$, we get 
$A = \frac{3}{5}, B = \frac{2}{5}, C = 1$. This allows as to rewrite $g(x)$ as a sum of fractions. \\

{$g(x) = \frac{3}{5}(\frac{1}{1 + 3x}) + \frac{2}{5}(\frac{1}{1 - 2x}) + \frac{1}{(1-2x)^2}$}\\

\noindent\normalsize
Looking closer, we notice that each fraction can be rewritten as a summation. We can convert the equations by using the following principles:\\

1) $\frac{1}{1 - x} = \sum_{n = 0}^\infty x^n$ and 
2) $\frac{1}{(1 - x)^k} = \sum_{n = 0}^\infty \multiset{n}{k}x^n$\\

\noindent
With these principles, we get the following equation:\\

$g(x) = \frac{3}{5}\sum_{n = 0}^\infty (-3x)^n + \frac{2}{5}\sum_{n = 0}^\infty 2x^n + \sum_{n = 0}^\infty (n + 1)(2x)^n$\\

\noindent
We can now use this equation to achieve our final equation for $a_n$. To do so, we simply remove the summations and the $x^n$'s from the equation.:\\

\framebox{\textbf{$a_n = \frac{3}{5}(-3)^n + \frac{2}{5}(2^n) + (n + 1)2^n$}} 

\end{solution} \\



\begin{problem} 4
Given $a_n = 6a_{n-1} - 9a_{n-2}$; $a_0 = 1, a_1 = 9$, find $a_n$.
\end{problem}
\begin{solution}
We find the generating function....\\\\
$g(x) = \sum_{n = 0}^\infty a_nx^n$\\\\
$g(x) = 1 + 9x + \sum_{n = 2}^\infty (6a_{n-1} - 9a_{n-2})x^n$\\\\
$g(x) = 1 + 9x + \sum_{n = 2}^\infty 6a_{n-1}x^n - \sum_{n = 2}^\infty 9a_{n-2}x^n$\\\\
$g(x) = 1 + 9x + 6x\sum_{n = 2}^\infty 6a_{n-1}x^{n-1} - 9x^2\sum_{n = 2}^\infty a_{n-2}x^{n-2}$\\\\
$g(x) = 1 + 9x + 6x\sum_{n = 1}^\infty 6a_{n}x^{n} - 9x^2\sum_{n = 0}^\infty a_{n}x^{n}$\\\\
$g(x) = 1 + 9x + 6x(g(x) - a_0) - 9x^2g(x)$\\\\\\
$g(x)(1 - 6x - 9x^2) = 1 + 9x - 6x$\\\\
$g(x) = \frac{1 + 3x}{1 - 6x - 9x^2}$\\\\
Then, we use partial fractions...\\\\
$\frac{1 + 3x}{1 - 6x - 9x^2} = \frac{A}{1 - 3x} + \frac{B}{(1 - 3x)^2}$\\\\
$1 + 3x = A(1 - 3x) + B$\\\\
If $x = 1/3$...$2 = B$\\
If $x = 0$...$1 = A + 2$, $A = -1$\\\\
This gives us an expression like so...\\\\
$g(x) = -\frac{1}{1 - 3x} + \frac{2}{(1 - 3x)^2}$\\\\
Using summation principles mentioned above, we get...\\\\
$g(x) = -1\sum_{n = 0}^\infty (3x)^n + 2\sum_{n = 0}^\infty (n + 1)(3x)^n$\\\\
$a_n = (-1)3^n + 2(n + 1)3^n$\\\\
$a_n = 3^n(-1 + 2n + 2)$\\\\
and\\\\
$a_n = (2n + 1)3^n$

\end{solution} \\


\begin{problem} 5
given $a_n = -3a_{n-1} + 10a_{n-2} + 3(2^n)$; $a_0 = 1, a_1 = 6$, find $a_n$.
\end{problem}
\begin{solution}
We will use the steps from above to find $a_n$. \\

\noindent
First let's find the generating function: \\\\
$g(x) = 0 + 6x + \sum_{n = 2}^\infty a_n x^n$ \\\\
$g(x) = 6x + \sum_{n = 2}^\infty(-3a_{n-1} + 10a_{n-2} + 3(2^n))x^n$\\\\
$g(x) = 6x + -3\sum_{n = 2}^\infty a_{n-1}x^n + 10\sum_{n = 2}^\infty a_{n-2}x^n + 3\sum_{n = 2}^\infty (2x)^n$\\\\
$g(x) =  6x - 3x\sum_{n = 2}^\infty a_{n-1}x^{n-1} + 10x^2\sum_{n = 2}^\infty a_{n-2}x^{n-2} + 3\sum_{n = 2}^\infty(2x)^n$\\\\
$g(x) =  6x - 3x\sum_{n = 1}^\infty a_{n}x^{n} + 10x^2\sum_{n = 0}^\infty a_{n}x^{n} + 3\sum_{n = 2}^\infty(2x)^n$\\\\
$g(x) =  6x - 3x(g(x) - 0) + 10x^2(g(x)) + 3(\frac{1}{1-2x}-1-2x)$\\\\
$g(x)[1+3x-10x^2]=6x+\frac{3}{1-2x}-3-6x=\frac{6x}{1-2x}$ \\\\
$g(x)=\frac{6x}{(1-2x)(1+3x-10x^2)}=\frac{6x}{(1-2x)^2(1+5x)}$ \\\\
\noindent \\
Now for $a_n$:\\\\
$\frac{6x}{(1-2x)^2(1+5x)} = \frac{A}{1-2x}+\frac{B}{(1-2x)^2}+\frac{C}{1+5x}$\\\\
\noindent
By solving this partial fraction equation, we get $A=\frac{-12}{49}$, $B=\frac{6}{7}$, and $C=\frac{-30}{49}$. Since these are messy numbers, we will hold off on substituting them into our equations until the end. \\\\
$g(x) =  A\sum_{n = 0}^\infty 2^n x^n + B\sum_{n = 0}^\infty (n+1)2^n x^{n} + C\sum_{n = 0}^\infty(-5)^n x^n$\\\\
$a_n = A*2^n + B(n+1)2^n + C(-5)^n$ \\\\
\noindent
Once we plug in the values of A, B, and C into the equation, we get our final equation for $a_n$: \\\\
\framebox{$a_n = \frac{-12}{49}2^n + \frac{6}{7}(n+1)2^n + \frac{-30}{49}(-5)^n$}

\end{solution} \\

\end{document}
