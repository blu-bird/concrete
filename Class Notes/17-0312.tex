%%%%%%% Lecture Notes Style for Concrete Mathematics %%%%%%%%%%%%%
%%%%%%% Taught by Patrick White, TJHSST %%%%%%%%%%%%%%%%%%%%%%%%%%

\documentclass[11pt,twosided]{article}
%%%%%%%%%%%%%%%%%%%%%%%%%%%%%%%%%%%%%%%%%%%%%%%%%%%%%%%
%%%% HEADER FILE for CONCRETE MATH LECTURE NOTES %%%%%%
%%%%%%%%%%%%%%%%%%%%%%%%%%%%%%%%%%%%%%%%%%%%%%%%%%%%%%%

%%%%%%%%%%%%%%%%%%%%%%%%%%%%%%%%%%%%%%%%%%%%%%%%%%%%%%%%%%%%%%%%%%%
%% This file is included at the top of every lecture notes file %%%
%% It should ONLY be changed by the instructor %%
%%%%%%%%%%%%%%%%%%%%%%%%%%%%%%%%%%%%%%%%%%%%%%%%%%%%%%%%%%%%%%%%%%%

%%%%%%%%%%%%%%%%%%%%%% package inclusions %%%%%%%%%%%%%%%%%%%%
%\usepackage[
%top=2cm,
%bottom=2cm,
%left=3cm,
%right=2cm,
%headheight=17pt, % as per the warning by fancyhdr
%includehead,includefoot,
%heightrounded, % to avoid spurious underfull messages
%]{geometry} 


\usepackage{amsmath,amssymb,amsfonts}
\usepackage{longtable}
\usepackage{mathtools}
\usepackage{amsthm}
\usepackage{enumerate}
\usepackage{fancyhdr}
\pagestyle{fancy}


%%%%%%%%%%%%%%% theorem style definitions %%%%%%%%%%%%%%%%%%%%%%
\newtheorem{theorem}{Theorem}[section]
%\newtheorem{corollary}{Corollary}[theorem]
%\newtheorem{lemma}[theorem]{Lemma}
\newtheorem*{remark}{Remark}
%\newtheorem{example}{Example}[section]
\newtheorem*{definition}{Definition}


%%%%%%%%%%%%%%%% custom function definitions %%%%%%%%%%%%%%%%%%%%%%%%%
%%%%%%%%%%% as class progresses, this section may be enhanced %%%%%%%%

\def\multiset#1#2{\ensuremath{\left(\kern-.3em\left(\genfrac{}{}{0pt}{}{#1}{#2}\right)
		\kern-.3em\right)}} %%% multiset notation
\newcommand\rf[2]{{#1}^{\overline{#2}}} %%%% rising factorial \rf{x}{m}
\newcommand\ff[2]{{#1}^{\underline{#2}}} %%%%% falling factorial \ff{x}{m}
\newcommand{\Perm}[2]{{}^{#1}\!P_{#2}} % permutation
\newcommand{\half}{\ensuremath{\frac{1}{2}}}
\newcommand{\braces}[1]{\left\{#1\right\}}
\newcommand{\set}[1]{\braces{#1}}
\newcommand{\snb}[2]{\ensuremath{\left(\kern-.3em\left(\genfrac{}{}{0pt}{}{#1}{#2}\right)\kern-.3em\right)}}
\newcommand{\fallingfactorial}[1]{^{\underline{#1}}}
\newcommand{\fallfac}[2]{{#1}^{\underline{#2}}}
\newcommand{\stirlingone}[2]{\genfrac[]{0pt}{1}{#1}{#2}}
\newcommand{\stiri}[2]{\stirlingone{#1}{#2}}
\newcommand{\dstirlingone}[2]{\genfrac[]{0pt}{0}{#1}{#2}}
\newcommand{\dstiri}[2]{\dstirlingone{#1}{#2}}
\newcommand{\stirlingtwo}[2]{\genfrac\{\}{0pt}{1}{#1}{#2}}
\newcommand{\stirii}[2]{\stirlingtwo{#1}{#2}}
\newcommand{\dstirlingtwo}[2]{\genfrac\{\}{0pt}{0}{#1}{#2}}
\newcommand{\dstirii}[2]{\dstirlingtwo{#1}{#2}}
\newcommand{\ol}[1]{\overline{#1}}
%%%%%%%
% book stuff
%%%%%%%%%

\usepackage[twoside,outer=1.5in,inner=2in,bottom=1.5in,top=1.5in,marginpar=2in]{geometry} 
\usepackage{scrextend}
\usepackage{palatino}
\usepackage{multicol}
\usepackage{float}
\usepackage[hidelinks]{hyperref}
\usepackage[nameinlink, capitalise, noabbrev]{cleveref}
%\usepackage{background}
\usepackage{graphicx}
\usepackage{listliketab}

\usepackage{lipsum}
\usepackage{caption}
\usepackage{marginnote}

\reversemarginpar


\usepackage[dvipsnames]{xcolor}
\usepackage{amsthm}
\usepackage{shadethm}

\setlength{\parindent}{0pt}

\newcommand*\Note{%
	\marginnote[\textcolor{blue}{\raggedright{\LARGE ?}\ Note }]{}%
}
\newcommand*\Warn{%
	\marginnote[\textcolor{red}{\raggedright{\LARGE !}\ Warning }]{}%
}
\reversemarginpar

\newcommand{\N}{\mathbb{N}}
\newcommand{\Z}{\mathbb{Z}}
\newcommand{\I}{\mathbb{I}}
\newcommand{\R}{\mathbb{R}}
\newcommand{\Q}{\mathbb{Q}}
\renewcommand{\qed}{\hfill$\blacksquare$}
\let\newproof\proof
\renewenvironment{proof}{\begin{addmargin}[1em]{0em}\begin{newproof}}{\end{newproof}\end{addmargin}\qed}
% \newcommand{\expl}[1]{\text{\hfill[#1]}$}


\newenvironment{lemma}[2][Lemma]{\begin{trivlist}
		\item[\hskip \labelsep {\bfseries #1}\hskip \labelsep {\bfseries #2.}]}{\end{trivlist}}
\newenvironment{problem}[2][Problem]{\begin{trivlist}
		\item[\hskip \labelsep {\bfseries #1}\hskip \labelsep {\bfseries #2.}]}{\end{trivlist}}
\newenvironment{example}[2][Example]{\begin{trivlist}
		\item[\hskip \labelsep {\bfseries #1}\hskip \labelsep{\bfseries #2.}]}{\end{trivlist}}
\newenvironment{solution}{{\noindent \bfseries Solution\hskip 2ex}}{\qed}
\newenvironment{exercise}[2][Exercise]{\begin{trivlist}
		\item[\hskip \labelsep {\bfseries #1}\hskip \labelsep {\bfseries #2.}]}{\end{trivlist}}
\newenvironment{reflection}[2][Reflection]{\begin{trivlist}
		\item[\hskip \labelsep {\bfseries #1}\hskip \labelsep {\bfseries #2.}]}{\end{trivlist}}
\newenvironment{proposition}[2][Proposition]{\begin{trivlist}
		\item[\hskip \labelsep {\bfseries #1}\hskip \labelsep {\bfseries #2.}]}{\end{trivlist}}
\newenvironment{corollary}[2][Corollary]{\begin{trivlist}
		\item[\hskip \labelsep {\bfseries #1}\hskip \labelsep {\bfseries #2.}]}{\end{trivlist}}

%%%% add package inclusions here, if any
\usepackage{amsfonts, amssymb, amsmath, enumitem}
%%%%% Define your custom functions and macros here, if any

%%%%%%%%%%%%%%%%%%%% Define the variables below %%%%%%%%%%%%%%%%%%%%%%%%%%%%%%
\def\titlestring{Basic Operations on Generating Functions}
\def\scribestring{Jongwan Kim and Jun Chong}
\def\datestring{Tuesday, March 12, 2019}
\def\multiset#1#2{\left(\!\!\left({#1\atopwithdelims..#2}\right)\!\!\right)}

\newcommand{\stirlingii}[2]{\genfrac{\{}{\}}{0pt}{}{#1}{#2}}
\newcommand{\stirlingi}[2]{\genfrac{[}{]}{0pt}{}{#1}{#2}}

%%%%%%%%%%%%%%%%%%% Page Headers -- Do Not Change %%%%%%%%%%%%%%%%%%%%%%%%%%%
\lhead{\titlestring}
\rhead{Page \thepage}
\cfoot{Concrete Math -- White -- TJHSST}
\renewcommand{\headrulewidth}{0.4pt}
\renewcommand{\footrulewidth}{0.4pt}

%%%%%%%%%%%%%%%%% Begin the document %%%%%%%%%%%%%%%%%%%%%%%%%
\begin{document}
\thispagestyle{plain}  %% no headers on this page

%%%% Do not change these lines %%%%%
\noindent
{\LARGE \textbf{\titlestring}}\\\\
%
{\Large Scribe: \scribestring}\\ \\
{\textbf{Date}: \datestring}

\begin{section}{Practice with Generating Functions}
Here's some practice turning sequences to generating functions:
\begin{enumerate}[a]
\item $\langle 1, 1, 1, 1 \ldots \rangle = \frac{1}{1-x}$
\item $\langle 2, 2, 2, 2 \ldots \rangle = \frac{2}{1-x}$ by doubling above equation
\item $\langle 1, -1, 1, -1 \ldots \rangle = \frac{1}{1+x}$ by noticing the alternation of signs, so we can substitute $-x$ for $x$
\item $\langle 1, 0, 1, 0 \ldots \rangle = \frac{1}{1-x^2}$ by noticing we only want the even powers, so we can substitute $x^2$ for $x$
\item $\langle 0, 1, 0, 1 \ldots \rangle = \frac{x}{1-x^2}$ by noticing we can multiply the solution the last problem by $x$ to shift the start of the sum.
\item $\langle 1, 2, 4, 8 \ldots \rangle = \frac{1}{1-2x}$ by noticing we can get powers of $2$ by replacing $x$ with $2x$
\item $\langle 1, \frac12, \frac16, \frac{1}{24}, \ldots \rangle = \frac{e^x - 1}{x}$ by noticing the similarity to the expansion of the exponential, but we have to shift it down by killing the first term and lowering the powers.
\item $\langle 1, 2, 3, 4, \ldots \rangle = \frac{d}{dx}\frac{1}{1-x} = \frac{1}{(1-x)^2}$ because taking a derivative multiplies each term by its index.
\item $\langle 2, 6, 12, 20, 30, \ldots \rangle = \frac{d^2}{dx^2}\frac{1}{1-x} = \frac{2}{(1-x)^3}$ by applying the above trick twice.
\item $\langle 0, 1, 4, 9, 16, \ldots \rangle = ?$ \\
Recall that 
$$k^2 = 1(k+2)(k+1) - 3(k+1) + 1$$ where the coefficients are Stirling numbers of the second kind. However, we already have expressions for the generating functions for the falling factorials: 
\begin{align*}
    (k+2)(k+1) &\rightarrow \frac{d^2}{dx^2}\frac{1}{1-x} = \frac{2}{(1-x)^3} \\
    (k+1) &\rightarrow \frac{d}{dx}\frac{1}{1-x} = \frac{1}{(1-x)^2} \\
    1 &\rightarrow \frac{1}{1-x}
\end{align*}
This gives us our answer $\frac{2}{(1-x)^3} - \frac{3}{(1-x)^2} + \frac{1}{1-x} = \frac{x + x^2}{(1-x)^3}$
\end{enumerate}
\end{section}
\begin{section}{Multiplication of Generating Functions}
Let
\begin{align*}
    A(x) &= a_0 x^0 + a_1 x^1 + a_2 x^2 + \ldots = \sum_{i\geq 0} a_i x_i \\
    B(x) &= b_0 x^0 + b_1 x^1 + b_2 x^2 + \ldots = \sum_{i\geq 0} b_i x_i
\end{align*}
Let $C(x) = A(x)\cdot B(x)$. What is it's corresponding sequence?\\
We have to take the \emph{Cauchy product}.
\begin{align*}
    C(x) &= a_0b_0 + (a_0 b_1 + a_1 b_0)x + (a_0 b_2 + a_1 b_1 + a_2 b_0)x^2 \ldots \\
    C_n &= \sum_{k=0}^n a_k b_{n-k}
\end{align*}
Consider the two functions
\begin{align*}
    A(x) &= \text{anything} \\
    B(x) &= 1 + x + x^2 + x^3 + \ldots \\
    C(x) = A(x) B(x) &= a_0 + (a_0 + a_1) x + \ldots = \sum_{i=0}^k a_i x^k
\end{align*}
Thus we arrive at the following identity:
\begin{theorem}
If $A(x)$ be the ordinary generating function for $\langle a_0, a_1, a_2 \ldots \rangle$. Then, $A(x)/(1-x)$ for $\langle a_0, a_0+a_1, a_0+a_1+a_2, \ldots \rangle$
\end{theorem}
Let us now consider $A(x) = B(x) = \frac{1}{1-x}$ Recall the following theorem:
$$\sum \frac{1}{(1-x)^n} = \sum_{k=0}^\infty \multiset{n}{k}x^k$$
\begin{align*}
    C(x) &= \langle 1, 2, 3, 4, 5, \ldots \rangle \\
    \frac{1}{(1-x)^2} &= \sum_{k=0}^\infty \multiset{2}{k} x^k \\
    &= \sum {{k+1}\choose k} x^k \\
    &= \sum (k+1) x^k
\end{align*}
Now let us consider $B(x)\cdot C(x)$
\begin{align*}
    \langle 1, 3, 6, \ldots \rangle &= \frac{1}{(1-x)^3} \\
    &= \sum_{k=0}^\infty \multiset{3}{k}x^k \\
    &= \sum_{k=0}^\infty {k + 2 \choose 2} x^k \\
    &= \sum_{k=1}^\infty \frac{k(k+1)}{2} x^{k-1}
\end{align*}
This gives us the formula for the triangular numbers. We can go further and get the formula for the tetrahedral numbers.
\begin{align*}
    D(x) &= \frac{x}{(1-x)^3} = 0 + x + 3x^2 + 6x^3 + 10x^4 + \ldots = \sum_{k = 0}^{n}\frac{k(k+1)}{2}x^k \\
    E(x) &= D(x)\cdot B(x) = \frac{x}{(1-x)^4} + x\cdot \sum_{k=0}^\infty \multiset{4}{k}x^k \\
    &= \sum_{k=0}^\infty {{k+3}\choose 3} x^{k+1} = \sum_{k=1}^\infty \frac{k(k+1)(k+2)}{6}x^k
\end{align*}
Now it seems that, by extending this process of computing generating functions for hypertetrahedral numbers and equation coefficients. 
$$\sum_{k=1}^n k^{\overline{m}} = \frac{n^{\overline{m+1}}}{m+1}$$
This is the power rule for $k^{\overline{m}}$ (finite integrals).
For instance, $1^{\overline{4}} + 2^{\overline{4}} + 3^{\overline{4}} + 4^{\overline{4}} + 5^{\overline{4}} + 6^{\overline{4}} = \frac{6^{\overline{5}}}{5}$.
\begin{problem}{1}
Find $\sum_{i=1}^n i^2$
\end{problem}
\begin{solution}
We can write $i^2 = i^{\overline{2}} - i^{\overline{1}}$, and we also know
$\sum_{i=1}^n i^{\overline{1}} = \frac{n(n+1)}{2}$ and $\sum_{i=1}^n i^{\overline{2}} = \frac{n(n+1)(n+2)}{3}$, so we can subtract and simplify, getting $\frac{n(n+1)(2n+1)}{6}$.
\end{solution} \\
\\
\begin{solution}
Start by expanding the rising factorial in terms of power.
$$\sum i^3 = \sum i^{\overline{3}} - 3\sum i^2 - 2\sum i = \frac{n^{\overline{4}}}{4} - \frac{n(n+1)(2n+1)}{2} - n(n+1)$$
\end{solution}

\end{section}
\end{document}
