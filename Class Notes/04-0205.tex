%%%%%%% Lecture Notes Style for Concrete Mathematics %%%%%%%%%%%%%
%%%%%%% Taught by Patrick White, TJHSST %%%%%%%%%%%%%%%%%%%%%%%%%%

\documentclass[11pt,twosided]{article}
%%%%%%%%%%%%%%%%%%%%%%%%%%%%%%%%%%%%%%%%%%%%%%%%%%%%%%%
%%%% HEADER FILE for CONCRETE MATH LECTURE NOTES %%%%%%
%%%%%%%%%%%%%%%%%%%%%%%%%%%%%%%%%%%%%%%%%%%%%%%%%%%%%%%

%%%%%%%%%%%%%%%%%%%%%%%%%%%%%%%%%%%%%%%%%%%%%%%%%%%%%%%%%%%%%%%%%%%
%% This file is included at the top of every lecture notes file %%%
%% It should ONLY be changed by the instructor %%
%%%%%%%%%%%%%%%%%%%%%%%%%%%%%%%%%%%%%%%%%%%%%%%%%%%%%%%%%%%%%%%%%%%

%%%%%%%%%%%%%%%%%%%%%% package inclusions %%%%%%%%%%%%%%%%%%%%
%\usepackage[
%top=2cm,
%bottom=2cm,
%left=3cm,
%right=2cm,
%headheight=17pt, % as per the warning by fancyhdr
%includehead,includefoot,
%heightrounded, % to avoid spurious underfull messages
%]{geometry} 


\usepackage{amsmath,amssymb,amsfonts}
\usepackage{longtable}
\usepackage{mathtools}
\usepackage{amsthm}
\usepackage{enumerate}
\usepackage{fancyhdr}
\pagestyle{fancy}


%%%%%%%%%%%%%%% theorem style definitions %%%%%%%%%%%%%%%%%%%%%%
\newtheorem{theorem}{Theorem}[section]
%\newtheorem{corollary}{Corollary}[theorem]
%\newtheorem{lemma}[theorem]{Lemma}
\newtheorem*{remark}{Remark}
%\newtheorem{example}{Example}[section]
\newtheorem*{definition}{Definition}


%%%%%%%%%%%%%%%% custom function definitions %%%%%%%%%%%%%%%%%%%%%%%%%
%%%%%%%%%%% as class progresses, this section may be enhanced %%%%%%%%

\def\multiset#1#2{\ensuremath{\left(\kern-.3em\left(\genfrac{}{}{0pt}{}{#1}{#2}\right)
		\kern-.3em\right)}} %%% multiset notation
\newcommand\rf[2]{{#1}^{\overline{#2}}} %%%% rising factorial \rf{x}{m}
\newcommand\ff[2]{{#1}^{\underline{#2}}} %%%%% falling factorial \ff{x}{m}
\newcommand{\Perm}[2]{{}^{#1}\!P_{#2}} % permutation
\newcommand{\half}{\ensuremath{\frac{1}{2}}}
\newcommand{\braces}[1]{\left\{#1\right\}}
\newcommand{\set}[1]{\braces{#1}}
\newcommand{\snb}[2]{\ensuremath{\left(\kern-.3em\left(\genfrac{}{}{0pt}{}{#1}{#2}\right)\kern-.3em\right)}}
\newcommand{\fallingfactorial}[1]{^{\underline{#1}}}
\newcommand{\fallfac}[2]{{#1}^{\underline{#2}}}
\newcommand{\stirlingone}[2]{\genfrac[]{0pt}{1}{#1}{#2}}
\newcommand{\stiri}[2]{\stirlingone{#1}{#2}}
\newcommand{\dstirlingone}[2]{\genfrac[]{0pt}{0}{#1}{#2}}
\newcommand{\dstiri}[2]{\dstirlingone{#1}{#2}}
\newcommand{\stirlingtwo}[2]{\genfrac\{\}{0pt}{1}{#1}{#2}}
\newcommand{\stirii}[2]{\stirlingtwo{#1}{#2}}
\newcommand{\dstirlingtwo}[2]{\genfrac\{\}{0pt}{0}{#1}{#2}}
\newcommand{\dstirii}[2]{\dstirlingtwo{#1}{#2}}
\newcommand{\ol}[1]{\overline{#1}}
%%%%%%%
% book stuff
%%%%%%%%%

\usepackage[twoside,outer=1.5in,inner=2in,bottom=1.5in,top=1.5in,marginpar=2in]{geometry} 
\usepackage{scrextend}
\usepackage{palatino}
\usepackage{multicol}
\usepackage{float}
\usepackage[hidelinks]{hyperref}
\usepackage[nameinlink, capitalise, noabbrev]{cleveref}
%\usepackage{background}
\usepackage{graphicx}
\usepackage{listliketab}

\usepackage{lipsum}
\usepackage{caption}
\usepackage{marginnote}

\reversemarginpar


\usepackage[dvipsnames]{xcolor}
\usepackage{amsthm}
\usepackage{shadethm}

\setlength{\parindent}{0pt}

\newcommand*\Note{%
	\marginnote[\textcolor{blue}{\raggedright{\LARGE ?}\ Note }]{}%
}
\newcommand*\Warn{%
	\marginnote[\textcolor{red}{\raggedright{\LARGE !}\ Warning }]{}%
}
\reversemarginpar

\newcommand{\N}{\mathbb{N}}
\newcommand{\Z}{\mathbb{Z}}
\newcommand{\I}{\mathbb{I}}
\newcommand{\R}{\mathbb{R}}
\newcommand{\Q}{\mathbb{Q}}
\renewcommand{\qed}{\hfill$\blacksquare$}
\let\newproof\proof
\renewenvironment{proof}{\begin{addmargin}[1em]{0em}\begin{newproof}}{\end{newproof}\end{addmargin}\qed}
% \newcommand{\expl}[1]{\text{\hfill[#1]}$}


\newenvironment{lemma}[2][Lemma]{\begin{trivlist}
		\item[\hskip \labelsep {\bfseries #1}\hskip \labelsep {\bfseries #2.}]}{\end{trivlist}}
\newenvironment{problem}[2][Problem]{\begin{trivlist}
		\item[\hskip \labelsep {\bfseries #1}\hskip \labelsep {\bfseries #2.}]}{\end{trivlist}}
\newenvironment{example}[2][Example]{\begin{trivlist}
		\item[\hskip \labelsep {\bfseries #1}\hskip \labelsep{\bfseries #2.}]}{\end{trivlist}}
\newenvironment{solution}{{\noindent \bfseries Solution\hskip 2ex}}{\qed}
\newenvironment{exercise}[2][Exercise]{\begin{trivlist}
		\item[\hskip \labelsep {\bfseries #1}\hskip \labelsep {\bfseries #2.}]}{\end{trivlist}}
\newenvironment{reflection}[2][Reflection]{\begin{trivlist}
		\item[\hskip \labelsep {\bfseries #1}\hskip \labelsep {\bfseries #2.}]}{\end{trivlist}}
\newenvironment{proposition}[2][Proposition]{\begin{trivlist}
		\item[\hskip \labelsep {\bfseries #1}\hskip \labelsep {\bfseries #2.}]}{\end{trivlist}}
\newenvironment{corollary}[2][Corollary]{\begin{trivlist}
		\item[\hskip \labelsep {\bfseries #1}\hskip \labelsep {\bfseries #2.}]}{\end{trivlist}}

%%%% add package inclusions here, if any
\usepackage{enumerate}

%%%%% Define your custom functions and macros here, if any

\newcommand{\Perm}[2]{{}^{#1}\!P_{#2}}%

%%%%%%%%%%%%%%%%%%%% Define the variables below %%%%%%%%%%%%%%%%%%%%%%%%%%%%%%
\def\titlestring{Day 4: More Counting}
\def\scribestring{Caroline Jin}
\def\datestring{Tuesday, February, 5 2019}


%%%%%%%%%%%%%%%%%%% Page Headers -- Do Not Change %%%%%%%%%%%%%%%%%%%%%%%%%%%
\lhead{\titlestring}
\rhead{Page \thepage}
\cfoot{Concrete Math -- White -- TJHSST}
\renewcommand{\headrulewidth}{0.4pt}
\renewcommand{\footrulewidth}{0.4pt}

%%%%%%%%%%%%%%%%% Begin the document %%%%%%%%%%%%%%%%%%%%%%%%%
\begin{document}
\thispagestyle{plain}  %% no headers on this page

%%%% Do not change these lines %%%%%
\noindent
{\LARGE \textbf{\titlestring}}\\\\
%
{\Large Scribe: \scribestring}\\ \\
{\textbf{Date}: \datestring}


%%%%%%%%%%%%%%%%%%%%%%%%%% YOUR CONTENT GOES HERE %%%%%%%%%%%%%%%%%%%%%%%%%%
\noindent

\section{Partitions}
    A \textit{partition} of a set $S$ is a subset of $S_1, ... S_k$ such that
\begin{enumerate}[(i)]
        \item $\bigcup\limits_{i=1}^{k} S_{i}$. The subsets `cover' the set $S$
        \item $S_i 	\cap S_j = \varnothing$. The subsets are pairwise disjoint.
        \item $S_i \neq \varnothing$
\end{enumerate}

\begin{problem}
    1 Let $S$ be the set of all integers composed of digits in $\{1, 3, 5, 7\}$ at most one.
    
    \begin{enumerate}[(i)]
        \item Find $|S|$
        \item $\sum\limits_{x \in S} x$
    \end{enumerate}
\end{problem}

\begin{solution}
    \begin{enumerate}[(i)]
        \item Let $S = S_1 \cup S_2 \cup S_3 \cup S_4$ where $S_1$ is the number of one digit numbers, $S_2$ is the number of two-digit numbers, and so on. \\
            \[|S_1| = \Perm{4}{1} = 4\]
            \[|S_2| = \Perm{4}{2} = 12\]
            \[|S_3| = \Perm{4}{3} = 24\]
            \[|S_4| = \Perm{4}{4} = 24\]
            \[|S| = |S_1| + |S_2| + |S_3| + |S_4| = \boxed{64}\]
        \item Let $\alpha = \alpha_1 + 10\alpha_2 + 100\alpha_3 + 1000\alpha_4$ where $\alpha_1$ is the sum of all units digits of all numbers in $S$, $\alpha_2$ is the sum of all the tens digits of all the numbers, and so on. We will find the value of $\alpha_1$ using the following: 
        \begin{align*}
            &S_1 \rightarrow s_1 = (1+3+5+7)\\
            &S_2 \rightarrow s_2 = (1+3+5+7)\times (3)\\
            &S_3 \rightarrow s_3 = (1+3+5+7)\times (3\times2)\\
            &S_4 \rightarrow s_4 = (1+3+5+7)\times (3\times2\times1)\\
            \hline
            &\alpha_1 = 16 \times (1+3+6+6) = 256
        \end{align*}
        Note that $\alpha_2$ is the sum of the same values, excluding $s_1$ as $S_1$ is the set of only one digit numbers. $\alpha_3$ is the sum of the same values as $\alpha_2$, excluding $s_2$ as $S_2$ is the set of only two digit numbers, and so on.
        \begin{align*}
            &\alpha_2 = \alpha_1 - s_1 = 240\\
            &\alpha_3 = \alpha_2 - s_2 = 192\\
            &\alpha_4 = \alpha_3 - s_3 = 96\\
        \end{align*}
        Thus, $\alpha = \alpha_1 + 10\alpha_2 + 100\alpha_3 + 1000\alpha_4 = \boxed{117,856}$
        \\\\
        An easier solution is the following:
        \begin{align*}
            &1 + 3 + 5 + 7 = (1 + 7) + (3 + 5) = 8(\frac{4}{2}) = 16\\
            &13 + ... + 75 = (13 + 75) + ... + (35 + 53) = 88(\frac{12}{2}) = 528\\
            &\text{etc}
        \end{align*}
        Since each $x \in S_i$ pairs with $ \bar{x} \in S_i$ to sum to $88...8$. We find $\alpha = 8\frac{|S_1|}{2} + 88\frac{|S_2|}{2} + 888\frac{|S_3|}{2} + 8888\frac{|S_4|}{2} = \boxed{117,856}$
    \end{enumerate}
    
\end{solution}

\section{Cyclic Permutation}

Consider the set $T$ of 3 permutations of $(s_1, s_2, s_3, s_4)$ or $(1, 2, 3, 4)$. We know that $T = \{ 123, 132, 234, 214, ... \}$ and $|T| = P(4, 3) = 24$.\\

We define $x \cong y \iff x + y $ are cyclically equivalently 

\begin{problem}
    1 Given $123 \cong x$, how many solutions are there for $x \in T$?
\end{problem}

\begin{solution}
    The $x$ values are 123, 231, 312, so there are \boxed{3} solutions. Thus, we see any sequence of length $n \cong n$ sequences
\end{solution}

\begin{theorem}
    If $Q(n, r)$ is the number of cyclic permutations of length $r$ from a set of $n$ elements, $Q(n, r) = \frac{P(n, r)}{r}$.
\end{theorem}

\begin{theorem}
    There are (n-1)! ways to seat n people around a round table.
\end{theorem}

\begin{proof}
    Each ordering $\cong n$ orderings. Thus, $\frac{n!}{n} = (n-1)!$
\end{proof}

\begin{problem}
    2 There are 5 boys and 3 girls seated around a round table.
    \begin{enumerate}[(i)]
        \item There are no restrictions.
        \item $B_1$ and $G_1$ are not adjacent
        \item No girls are adjacent to other girls
    \end{enumerate}
\end{problem}

\begin{solution}
    \begin{enumerate}[(i)]
        \item Using theorem 2.1, there are \boxed{7!} ways.
        \item We first place $B_1$ in any of the 7 seats and set $B_1$ as our reference point. There are then 5 places for $G_1$ to sit not adjacent to $B_1$ and 6! ways for the remaining 6 people to sit. The total number of ways if $\boxed{6! \cdot 5}$. We can also consider the number of ways for $B_1$ and $G_1$ to sit next to each other, which is $2 \cdot 6!$. Subtracting that from arranging without restrictions, the total number of ways is $\boxed{7! - 2 \cdot 6!}$ 
        \item We first arrange all the 5 boys, which is $4!$ ways. There are 5 spaces between each boy, so we can choose 3 of the seats and then arrange the 3 girls, $\binom{5}{3} \cdot 3!$. The total number of ways is $\boxed{4! \cdot \binom{5}{3} \cdot 3!}$
    \end{enumerate}
\end{solution}
\section{Recursion}
\begin{exercise}
    1 Find the recursive definition of $P(n,r)$.
\end{exercise}

\begin{solution}
    We know that the closed form of $P(n,r) = \frac{n!}{(n-r)!}$. Our goal is to define $P(n, r) = f(P(<n, <r))$.\\
    Let $S = \{s_1, ..., s_n\}$, r be given $0 \leq r \leq n$, and $T$ be the set of all r-permutations of $S$.
    \noindent
    We can partition $T$ into $T = T_1 \cup T_2$ where
    \begin{align*}
        &t = T_1 \Leftrightarrow s_1 \notin t \text{ (no $s_1$)}\\
        &t = T_2 \Leftrightarrow s_1 \in t \text{ (yes $s_1$)}
    \end{align*}
    We can find the $|T_1|$ in terms of $P(\leq n, \leq r)$
    \begin{align*}
        &|T_1| = P(n-1, r)\\
        &|T_2| = r \cdot P(n-1, r-1) 
    \end{align*}
    For $|T_2|$, we can order $r-1$ elements from $\{s_2, ..., s_n \}$ and place $s_1$ in any of the $r$ locations. Thus, $\boxed{P(n,r) = P(n-1,r) + r \cdot P(n-1, r-1)}$
\end{solution}

\begin{exercise}
    2 Find a recursive definition of $C(n, r)$
\end{exercise}

\begin{solution}
    Again, let $T$ be the subset of $S = \{s_1, ... , s_n\}$ of size $r$.\\
    To find $|T|$, let $T = T_1 \cup T_2$ where $T_1$ has no set containing $s_1$ and $T_2$ has every set containing $s_1$.
    \begin{align*}
        &|T_1| = C(n-1, r) \text{ we can choose $r$ from $s_2, ... s_n$}\\
        &|T_2| = C(n-1, r-1) \text{ we choose $r-1$ from $s_2, ... s_n$ and add in $s_1$}
    \end{align*}
    Thus, $\boxed{C(n, r) = C(n-1, r) + C(n-1, r-1)}$
\end{solution}

\begin{problem}
    1 Given $2n$ tennis players. How many ways are there to arrange $n$ games/pairings?
\end{problem}

\begin{solution}
    There are several solutions to this problem:
    \begin{enumerate}
        \item We can match $P_1$ with another $2n - 1$ players. For the next player $P_2$ who hasn't been matched, we can choose $2n - 3$ players, and so on. The solution is just $(2n-1)(2n-3)...(1) = \boxed{(2n-1)!!}$
        \item We can choose each pair and divide by $n!$ to remove the ordering of the pairs. There are $\boxed{\frac{\binom{2n}{n}\binom{2n-2}{2} ... \binom{2}{2}}{n!}}$ ways. 
        \item We can permute all $2n$ players and divide by $n!$ (the number of ways to order the game doesn't matter) and $2^n$ (the order of the partners doesn't matter). There are $\boxed{\frac{(2n)!}{n!2^n}}$
    \end{enumerate}

\end{solution}
\end{document}
