%%%%%%% Lecture Notes Style for Concrete Mathematics %%%%%%%%%%%%%
%%%%%%% Taught by Patrick White, TJHSST %%%%%%%%%%%%%%%%%%%%%%%%%%

\documentclass[11pt,twosided]{article}
%%%%%%%%%%%%%%%%%%%%%%%%%%%%%%%%%%%%%%%%%%%%%%%%%%%%%%%
%%%% HEADER FILE for CONCRETE MATH LECTURE NOTES %%%%%%
%%%%%%%%%%%%%%%%%%%%%%%%%%%%%%%%%%%%%%%%%%%%%%%%%%%%%%%

%%%%%%%%%%%%%%%%%%%%%%%%%%%%%%%%%%%%%%%%%%%%%%%%%%%%%%%%%%%%%%%%%%%
%% This file is included at the top of every lecture notes file %%%
%% It should ONLY be changed by the instructor %%
%%%%%%%%%%%%%%%%%%%%%%%%%%%%%%%%%%%%%%%%%%%%%%%%%%%%%%%%%%%%%%%%%%%

%%%%%%%%%%%%%%%%%%%%%% package inclusions %%%%%%%%%%%%%%%%%%%%
%\usepackage[
%top=2cm,
%bottom=2cm,
%left=3cm,
%right=2cm,
%headheight=17pt, % as per the warning by fancyhdr
%includehead,includefoot,
%heightrounded, % to avoid spurious underfull messages
%]{geometry} 


\usepackage{amsmath,amssymb}
\usepackage{mathtools}
\usepackage{amsthm}

\usepackage{fancyhdr}
\pagestyle{fancy}


%%%%%%%%%%%%%%% theorem style definitions %%%%%%%%%%%%%%%%%%%%%%
\newtheorem{theorem}{Theorem}[section]
%\newtheorem{corollary}{Corollary}[theorem]
%\newtheorem{lemma}[theorem]{Lemma}
\newtheorem*{remark}{Remark}
%\newtheorem{example}{Example}[section]
\newtheorem*{definition}{Definition}


%%%%%%%%%%%%%%%% custom function definitions %%%%%%%%%%%%%%%%%%%%%%%%%
%%%%%%%%%%% as class progresses, this section may be enhanced %%%%%%%%

\def\multiset#1#2{\ensuremath{\left(\kern-.3em\left(\genfrac{}{}{0pt}{}{#1}{#2}\right)
		\kern-.3em\right)}} %%% multiset notation
\newcommand\rf[2]{{#1}^{\overline{#2}}} %%%% rising factorial \rf{x}{m}
\newcommand\ff[2]{{#1}^{\underline{#2}}} %%%%% falling factorial \ff{x}{m}


%%%%%%%
% book stuff
%%%%%%%%%

\usepackage[twoside,outer=1.5in,inner=2in,bottom=1.5in,top=1.5in,marginpar=2in]{geometry} 
\usepackage{amsmath,amsthm,amssymb,scrextend}
\usepackage{fancyhdr}
\usepackage{palatino}
\usepackage{multicol}
%\usepackage{background}
\usepackage{graphicx}
\usepackage{listliketab}

\usepackage{lipsum}
\usepackage{caption}
\usepackage{marginnote}

\reversemarginpar


\usepackage[dvipsnames]{xcolor}
\usepackage{amsthm}
\usepackage{shadethm}

\newcommand*\Note{%
	\marginnote[\textcolor{blue}{\raggedright{\LARGE ?}\ Note }]{}%
}
\newcommand*\Warn{%
	\marginnote[\textcolor{red}{\raggedright{\LARGE !}\ Warning }]{}%
}
\reversemarginpar

\newcommand{\N}{\mathbb{N}}
\newcommand{\Z}{\mathbb{Z}}
\newcommand{\I}{\mathbb{I}}
\newcommand{\R}{\mathbb{R}}
\newcommand{\Q}{\mathbb{Q}}
\renewcommand{\qed}{\hfill$\blacksquare$}
\let\newproof\proof
\renewenvironment{proof}{\begin{addmargin}[1em]{0em}\begin{newproof}}{\end{newproof}\end{addmargin}\qed}
% \newcommand{\expl}[1]{\text{\hfill[#1]}$}


\newenvironment{lemma}[2][Lemma]{\begin{trivlist}
		\item[\hskip \labelsep {\bfseries #1}\hskip \labelsep {\bfseries #2.}]}{\end{trivlist}}
\newenvironment{problem}[2][Problem]{\begin{trivlist}
		\item[\hskip \labelsep {\bfseries #1}\hskip \labelsep {\bfseries #2.}]}{\end{trivlist}}
\newenvironment{example}[2][Example]{\begin{trivlist}
		\item[\hskip \labelsep {\bfseries #1}\hskip \labelsep{\bfseries #2.}]}{\end{trivlist}}
\newenvironment{solution}{{\noindent \bfseries Solution\hskip 2ex}}{\qed}
\newenvironment{exercise}[2][Exercise]{\begin{trivlist}
		\item[\hskip \labelsep {\bfseries #1}\hskip \labelsep {\bfseries #2.}]}{\end{trivlist}}
\newenvironment{reflection}[2][Reflection]{\begin{trivlist}
		\item[\hskip \labelsep {\bfseries #1}\hskip \labelsep {\bfseries #2.}]}{\end{trivlist}}
\newenvironment{proposition}[2][Proposition]{\begin{trivlist}
		\item[\hskip \labelsep {\bfseries #1}\hskip \labelsep {\bfseries #2.}]}{\end{trivlist}}
\newenvironment{corollary}[2][Corollary]{\begin{trivlist}
		\item[\hskip \labelsep {\bfseries #1}\hskip \labelsep {\bfseries #2.}]}{\end{trivlist}}



%%%% add package inclusions here, if any
\usepackage{blubase}
%%%%% Define your custom functions and macros here, if any

%%%%%%%%%%%%%%%%%%%% Define the variables below %%%%%%%%%%%%%%%%%%%%%%%%%%%%%%
\def\titlestring{The Lecture Title}
\def\scribestring{Your Name}
\def\datestring{Day, Mon, Date Year}


%%%%%%%%%%%%%%%%%%% Page Headers -- Do Not Change %%%%%%%%%%%%%%%%%%%%%%%%%%%
\lhead{\titlestring}
\rhead{Page \thepage}
\cfoot{Concrete Math -- White -- TJHSST}
\renewcommand{\headrulewidth}{0.4pt}
\renewcommand{\footrulewidth}{0.4pt}

%%%%%%%%%%%%%%%%% Begin the document %%%%%%%%%%%%%%%%%%%%%%%%%
\begin{document}
\thispagestyle{plain}  %% no headers on this page

%%%% Do not change these lines %%%%%
\noindent
{\LARGE \textbf{\titlestring}}\\\\
%
{\Large Scribe: \scribestring}\\ \\
{\textbf{Date}: \datestring}


%%%%%%%%%%%%%%%%%%%%%%%%%% YOUR CONTENT GOES HERE %%%%%%%%%%%%%%%%%%%%%%%%%%
\noindent

\section{Generating Function for Catalan Numbers}
Recall the recursive definition of the Catalan numbers with base cases $C_0 = C_1 = 1$, and recursion
\[
	C_n = \sum_{k=0}^{n-1} C_i C_{n-1-i}
\]
Let's find an explicit form for $C_n$ by finding the generating function. Let $f(x) = \sum_{n=0}^\infty C_n$ - or in expanded form, 
\[
	f(x) = C_0 + C_1x + C_2x^2 + C_3x^3 + \ldots
\]
Let's go ahead and try to square this, collecting the terms of the same degree by looking at the Cauchy products of Catalan numbers: 
\[
	f(x)^2 = C_0^2 + (C_0C_1 + C_1C_2)x + (C_0C_2 + C_1C_1 + C_2C_0)x^2 + \ldots
\]
Applying the Catalan recursion, we have: 
\[
	f(x)^2 = C_1 + C_2 x + C_3 x^2 + \ldots \implies x f(x)^2 = C_1 x + C_2x^2 + C_3x^3 + \ldots 
\]
This right hand side is almost $f(x)$! We now have the functional equation
\[
	xf(x)^2  = f(x) - 1
\]
By the quadratic formula, we can find that 
\[
	f(x) = \frac{1 \pm \sqrt{1- 4x}}{2x}
\]
We will eventually pick the form that satisfies these base cases above. For now, let us expand this binomially to see which case we would like. Notice first we have to expand $(1-4x)^\frac{1}{2}$, which we have partially done before: 
\begin{align*}
(1-4x)^\frac{1}{2} &= \sum_{n=0}^\infty \frac{\binom{\frac{1}{2}}{k}}{k!} x^k \\
&= 1 - \frac{1}{2} \sum_{n=0}^\infty\binom{2n}{n}\frac{1}{n+1} \frac{1}{4^n} (4x)^{n+1} \\
&= 1 - 2 \sum_{n=0}^\infty\binom{2n}{n}\frac{1}{n+1}x^{n+1} 
\end{align*}
Notice that in order to get the $1$ cancelling out in the numerator, we have to pick the negative solution to avoid a divergence. Thus, we have 
\begin{align*}
f(x) &= \frac{1}{2x}\left(1 - \left(1 - 2 \sum_{n=0}^\infty\binom{2n}{n}\frac{1}{n+1}x^{n+1}  \right)  \right) \\
&= \frac{1}{2x}\left( \right) 
\end{align*}

















\end{document}

