%%%%%%% Lecture Notes Style for Concrete Mathematics %%%%%%%%%%%%%
%%%%%%% Taught by Patrick White, TJHSST %%%%%%%%%%%%%%%%%%%%%%%%%%

\documentclass[11pt,twosided]{article}
%%%%%%%%%%%%%%%%%%%%%%%%%%%%%%%%%%%%%%%%%%%%%%%%%%%%%%%
%%%% HEADER FILE for CONCRETE MATH LECTURE NOTES %%%%%%
%%%%%%%%%%%%%%%%%%%%%%%%%%%%%%%%%%%%%%%%%%%%%%%%%%%%%%%

%%%%%%%%%%%%%%%%%%%%%%%%%%%%%%%%%%%%%%%%%%%%%%%%%%%%%%%%%%%%%%%%%%%
%% This file is included at the top of every lecture notes file %%%
%% It should ONLY be changed by the instructor %%
%%%%%%%%%%%%%%%%%%%%%%%%%%%%%%%%%%%%%%%%%%%%%%%%%%%%%%%%%%%%%%%%%%%

%%%%%%%%%%%%%%%%%%%%%% package inclusions %%%%%%%%%%%%%%%%%%%%
%\usepackage[
%top=2cm,
%bottom=2cm,
%left=3cm,
%right=2cm,
%headheight=17pt, % as per the warning by fancyhdr
%includehead,includefoot,
%heightrounded, % to avoid spurious underfull messages
%]{geometry} 


\usepackage{amsmath,amssymb,amsfonts}
\usepackage{longtable}
\usepackage{mathtools}
\usepackage{amsthm}
\usepackage{enumerate}
\usepackage{fancyhdr}
\pagestyle{fancy}


%%%%%%%%%%%%%%% theorem style definitions %%%%%%%%%%%%%%%%%%%%%%
\newtheorem{theorem}{Theorem}[section]
%\newtheorem{corollary}{Corollary}[theorem]
%\newtheorem{lemma}[theorem]{Lemma}
\newtheorem*{remark}{Remark}
%\newtheorem{example}{Example}[section]
\newtheorem*{definition}{Definition}


%%%%%%%%%%%%%%%% custom function definitions %%%%%%%%%%%%%%%%%%%%%%%%%
%%%%%%%%%%% as class progresses, this section may be enhanced %%%%%%%%

\def\multiset#1#2{\ensuremath{\left(\kern-.3em\left(\genfrac{}{}{0pt}{}{#1}{#2}\right)
		\kern-.3em\right)}} %%% multiset notation
\newcommand\rf[2]{{#1}^{\overline{#2}}} %%%% rising factorial \rf{x}{m}
\newcommand\ff[2]{{#1}^{\underline{#2}}} %%%%% falling factorial \ff{x}{m}
\newcommand{\Perm}[2]{{}^{#1}\!P_{#2}} % permutation
\newcommand{\half}{\ensuremath{\frac{1}{2}}}
\newcommand{\braces}[1]{\left\{#1\right\}}
\newcommand{\set}[1]{\braces{#1}}
\newcommand{\snb}[2]{\ensuremath{\left(\kern-.3em\left(\genfrac{}{}{0pt}{}{#1}{#2}\right)\kern-.3em\right)}}
\newcommand{\fallingfactorial}[1]{^{\underline{#1}}}
\newcommand{\fallfac}[2]{{#1}^{\underline{#2}}}
\newcommand{\stirlingone}[2]{\genfrac[]{0pt}{1}{#1}{#2}}
\newcommand{\stiri}[2]{\stirlingone{#1}{#2}}
\newcommand{\dstirlingone}[2]{\genfrac[]{0pt}{0}{#1}{#2}}
\newcommand{\dstiri}[2]{\dstirlingone{#1}{#2}}
\newcommand{\stirlingtwo}[2]{\genfrac\{\}{0pt}{1}{#1}{#2}}
\newcommand{\stirii}[2]{\stirlingtwo{#1}{#2}}
\newcommand{\dstirlingtwo}[2]{\genfrac\{\}{0pt}{0}{#1}{#2}}
\newcommand{\dstirii}[2]{\dstirlingtwo{#1}{#2}}
\newcommand{\ol}[1]{\overline{#1}}
%%%%%%%
% book stuff
%%%%%%%%%

\usepackage[twoside,outer=1.5in,inner=2in,bottom=1.5in,top=1.5in,marginpar=2in]{geometry} 
\usepackage{scrextend}
\usepackage{palatino}
\usepackage{multicol}
\usepackage{float}
\usepackage[hidelinks]{hyperref}
\usepackage[nameinlink, capitalise, noabbrev]{cleveref}
%\usepackage{background}
\usepackage{graphicx}
\usepackage{listliketab}

\usepackage{lipsum}
\usepackage{caption}
\usepackage{marginnote}

\reversemarginpar


\usepackage[dvipsnames]{xcolor}
\usepackage{amsthm}
\usepackage{shadethm}

\setlength{\parindent}{0pt}

\newcommand*\Note{%
	\marginnote[\textcolor{blue}{\raggedright{\LARGE ?}\ Note }]{}%
}
\newcommand*\Warn{%
	\marginnote[\textcolor{red}{\raggedright{\LARGE !}\ Warning }]{}%
}
\reversemarginpar

\newcommand{\N}{\mathbb{N}}
\newcommand{\Z}{\mathbb{Z}}
\newcommand{\I}{\mathbb{I}}
\newcommand{\R}{\mathbb{R}}
\newcommand{\Q}{\mathbb{Q}}
\renewcommand{\qed}{\hfill$\blacksquare$}
\let\newproof\proof
\renewenvironment{proof}{\begin{addmargin}[1em]{0em}\begin{newproof}}{\end{newproof}\end{addmargin}\qed}
% \newcommand{\expl}[1]{\text{\hfill[#1]}$}


\newenvironment{lemma}[2][Lemma]{\begin{trivlist}
		\item[\hskip \labelsep {\bfseries #1}\hskip \labelsep {\bfseries #2.}]}{\end{trivlist}}
\newenvironment{problem}[2][Problem]{\begin{trivlist}
		\item[\hskip \labelsep {\bfseries #1}\hskip \labelsep {\bfseries #2.}]}{\end{trivlist}}
\newenvironment{example}[2][Example]{\begin{trivlist}
		\item[\hskip \labelsep {\bfseries #1}\hskip \labelsep{\bfseries #2.}]}{\end{trivlist}}
\newenvironment{solution}{{\noindent \bfseries Solution\hskip 2ex}}{\qed}
\newenvironment{exercise}[2][Exercise]{\begin{trivlist}
		\item[\hskip \labelsep {\bfseries #1}\hskip \labelsep {\bfseries #2.}]}{\end{trivlist}}
\newenvironment{reflection}[2][Reflection]{\begin{trivlist}
		\item[\hskip \labelsep {\bfseries #1}\hskip \labelsep {\bfseries #2.}]}{\end{trivlist}}
\newenvironment{proposition}[2][Proposition]{\begin{trivlist}
		\item[\hskip \labelsep {\bfseries #1}\hskip \labelsep {\bfseries #2.}]}{\end{trivlist}}
\newenvironment{corollary}[2][Corollary]{\begin{trivlist}
		\item[\hskip \labelsep {\bfseries #1}\hskip \labelsep {\bfseries #2.}]}{\end{trivlist}}

%%%% add package inclusions here, if any
\usepackage{blubase}
%%%%% Define your custom functions and macros here, if any

%%%%%%%%%%%%%%%%%%%% Define the variables below %%%%%%%%%%%%%%%%%%%%%%%%%%%%%%
\def\titlestring{Partition Numbers}
\def\scribestring{Bryan Lu}
\def\datestring{Monday, April 8, 2019}


%%%%%%%%%%%%%%%%%%% Page Headers -- Do Not Change %%%%%%%%%%%%%%%%%%%%%%%%%%%
\lhead{\titlestring}
\rhead{Page \thepage}
\cfoot{Concrete Math -- White -- TJHSST}
\renewcommand{\headrulewidth}{0.4pt}
\renewcommand{\footrulewidth}{0.4pt}

%%%%%%%%%%%%%%%%% Begin the document %%%%%%%%%%%%%%%%%%%%%%%%%
\begin{document}
\thispagestyle{plain}  %% no headers on this page

%%%% Do not change these lines %%%%%
\noindent
{\LARGE \textbf{\titlestring}}\\\\
%
{\Large Scribe: \scribestring}\\ \\
{\textbf{Date}: \datestring}


%%%%%%%%%%%%%%%%%%%%%%%%%% YOUR CONTENT GOES HERE %%%%%%%%%%%%%%%%%%%%%%%%%%
\noindent

\section{Partition Numbers}
The last combinatorial numbers that we have yet to look at in our twelve-fold table are the partition numbers. We define the \textbf{partition number} $p(n)$ as the number of ways to write $n$ as a sum of positive integers. For example, $p(5) = 7$ by enumeration: 
\begin{align*}
	5 &= 1 + 1 + 1 + 1 + 1 = 1 + 1 + 1 + 2 = 1 + 2 + 2 = 1 + 1 + 3 \\
	&= 1 + 4 = 2 + 3 
\end{align*} 


Note the distinction between partitions and compositions of numbers - when doing compositions, we are counting the partitions $3, 1, 1$, $1, 3, 1$, and $1, 1, 3$ as independent ways to make compositions of $5$, while these are all the same partition of $5$.  

The explicit formula for $p(n)$ is given by 
\[
	p(n) = \frac{1}{\pi\sqrt{2}} \sum_{k=1}^\infty \sqrt{k} A_k(n) \dv{}{n}\left( \frac{1}{\sqrt{n - \frac{1}{24}}} \sinh \left[\frac{\pi}{k} \sqrt{\frac{2}{3} \left(n - \frac{1}{24}\right)} \right]\right)
\]
where $A_k$ is given by
\[
	A_k = \sum_{\gcd(m, k) = 1, m < k}e^{\pi i (S(m, k)- 2nm/k}
\]	
derived by Ramanujan.

We can still state some results regarding partition numbers by comparing this to compositions. First, let $p_k(n)$ be the number of ways to partition $n$ into $k$ parts. Notice that the number of solutions to 
\[
	x_1 + x_2 + \ldots x_k = n
\]
is given by $\tsnb{k}{n-k}$, using the multiset notation, or also $\binom{n-1}{k-1}$. Each composition contributes at most $k!$ compositions, so the number of partitions on $k$ elements, $p_k(n)$, multiplied by the number of ways to arrange $k$ distinct elements is: 
\[
	k! p_n(k) \geq \binom{n-1}{k-1}
\]
so we have that we can bound $p_k(n)$ from below by 
\[
	p_k(n) \geq \frac{1}{k!}\binom{n-1}{k-1}
\]

\section{Ferrers Graphs/Young Tableaux}
We can write partitions of positive numbers in Ferrers graphs or Young tableaux - for example, here is a partition of $7$ into 4, 2, and 1:
\begin{center}
\begin{tabular}{c c c c}
$\circ$ & $\circ$ & $\circ$ & $\circ$ \\
$\circ$ & $\circ$ &  &  \\
$\circ$ &  & & \\

\end{tabular}
\end{center}

The transpose of each of these tableaux is still a partition, $\lambda^T$. Here is the transpose of the tableaux we saw above: 
\begin{center}
\begin{tabular}{c c c}
$\circ$ & $\circ$ & $\circ$ \\
$\circ$ & $\circ$ & \\
$\circ$ &  &\\
$\circ$ &  &  \\
\end{tabular}
\end{center}

If a partition $\lambda$ is \textit{self-conjugate}, then $\lambda^T = \lambda$. Here's a theorem that we've kind of seen before: 
\begin{theorem}
The number of self-congruent partitions $\lambda$ of $n$ is the number of partitions of $n$ into \textit{distinct} odd-sized parts.
\end{theorem} 
\begin{proof}
Note that if we draw the Young tableaux for a self-conjugate partition, we can construct a partition of $n$ into distinct odd numbers by taking the "outer" row/column (ie. the left most and the topmost row) and repeating on the remaining tableaux. If we repeat this process, this gives us a partition of $n$ into odd numbers. 

To go backwards, we can construct a self-conjugate partition by constructing the Young tableaux from the inside out, starting from the smallest odd number. 

\end{proof}

Another theorem of this nature: 
\begin{theorem}
The number of partitions of $n$ into even parts is the same as the number of partitions of $n$ into even multiplicity. 
\end{theorem}
\begin{proof}
Draw out a Young tableaux representing a partition of even multiplicity. Note that the columns are even-height, so the transpose of this partition will have even-length rows.  This is easily reversible, so we have a bijection. 

\end{proof}

\begin{theorem}
\[
	p_4(n) = p_4(3n)
\]
\end{theorem}
\begin{proof}
Consider a Young tableaux in a $4 \times n$ rectangle. The area taken up by the tableaux is $n$, and the area of the rectangle is $4n$, so the remaining area, when flipped upside down, is another Young tableaux for $3n$. This completes the proof. 

\end{proof}

In general, for integers $k, n$: 
\[
	p_k(n) = p_k((k-1)n)
\]

Finally, a recursive formula: 
\[
	p_k(n) = \sum_{i=1}^k p_i(n-k)
\]

\begin{proof}
Consider chopping off the left-hand column of a Ferrers graph of $n$ that has $k$ rows, and then taking the transpose of the rows gives the recurrence for partitions of $n$ over all possibilities. 

\end{proof}














\end{document}
