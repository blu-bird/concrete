%%%%%%% Lecture Notes Style for Concrete Mathematics %%%%%%%%%%%%%
%%%%%%% Taught by Patrick White, TJHSST %%%%%%%%%%%%%%%%%%%%%%%%%%

\documentclass[11pt,twosided]{article}
%%%%%%%%%%%%%%%%%%%%%%%%%%%%%%%%%%%%%%%%%%%%%%%%%%%%%%%
%%%% HEADER FILE for CONCRETE MATH LECTURE NOTES %%%%%%
%%%%%%%%%%%%%%%%%%%%%%%%%%%%%%%%%%%%%%%%%%%%%%%%%%%%%%%

%%%%%%%%%%%%%%%%%%%%%%%%%%%%%%%%%%%%%%%%%%%%%%%%%%%%%%%%%%%%%%%%%%%
%% This file is included at the top of every lecture notes file %%%
%% It should ONLY be changed by the instructor %%
%%%%%%%%%%%%%%%%%%%%%%%%%%%%%%%%%%%%%%%%%%%%%%%%%%%%%%%%%%%%%%%%%%%

%%%%%%%%%%%%%%%%%%%%%% package inclusions %%%%%%%%%%%%%%%%%%%%
%\usepackage[
%top=2cm,
%bottom=2cm,
%left=3cm,
%right=2cm,
%headheight=17pt, % as per the warning by fancyhdr
%includehead,includefoot,
%heightrounded, % to avoid spurious underfull messages
%]{geometry} 


\usepackage{amsmath,amssymb,amsfonts}
\usepackage{longtable}
\usepackage{mathtools}
\usepackage{amsthm}
\usepackage{enumerate}
\usepackage{fancyhdr}
\pagestyle{fancy}


%%%%%%%%%%%%%%% theorem style definitions %%%%%%%%%%%%%%%%%%%%%%
\newtheorem{theorem}{Theorem}[section]
%\newtheorem{corollary}{Corollary}[theorem]
%\newtheorem{lemma}[theorem]{Lemma}
\newtheorem*{remark}{Remark}
%\newtheorem{example}{Example}[section]
\newtheorem*{definition}{Definition}


%%%%%%%%%%%%%%%% custom function definitions %%%%%%%%%%%%%%%%%%%%%%%%%
%%%%%%%%%%% as class progresses, this section may be enhanced %%%%%%%%

\def\multiset#1#2{\ensuremath{\left(\kern-.3em\left(\genfrac{}{}{0pt}{}{#1}{#2}\right)
		\kern-.3em\right)}} %%% multiset notation
\newcommand\rf[2]{{#1}^{\overline{#2}}} %%%% rising factorial \rf{x}{m}
\newcommand\ff[2]{{#1}^{\underline{#2}}} %%%%% falling factorial \ff{x}{m}
\newcommand{\Perm}[2]{{}^{#1}\!P_{#2}} % permutation
\newcommand{\half}{\ensuremath{\frac{1}{2}}}
\newcommand{\braces}[1]{\left\{#1\right\}}
\newcommand{\set}[1]{\braces{#1}}
\newcommand{\snb}[2]{\ensuremath{\left(\kern-.3em\left(\genfrac{}{}{0pt}{}{#1}{#2}\right)\kern-.3em\right)}}
\newcommand{\fallingfactorial}[1]{^{\underline{#1}}}
\newcommand{\fallfac}[2]{{#1}^{\underline{#2}}}
\newcommand{\stirlingone}[2]{\genfrac[]{0pt}{1}{#1}{#2}}
\newcommand{\stiri}[2]{\stirlingone{#1}{#2}}
\newcommand{\dstirlingone}[2]{\genfrac[]{0pt}{0}{#1}{#2}}
\newcommand{\dstiri}[2]{\dstirlingone{#1}{#2}}
\newcommand{\stirlingtwo}[2]{\genfrac\{\}{0pt}{1}{#1}{#2}}
\newcommand{\stirii}[2]{\stirlingtwo{#1}{#2}}
\newcommand{\dstirlingtwo}[2]{\genfrac\{\}{0pt}{0}{#1}{#2}}
\newcommand{\dstirii}[2]{\dstirlingtwo{#1}{#2}}
\newcommand{\ol}[1]{\overline{#1}}
%%%%%%%
% book stuff
%%%%%%%%%

\usepackage[twoside,outer=1.5in,inner=2in,bottom=1.5in,top=1.5in,marginpar=2in]{geometry} 
\usepackage{scrextend}
\usepackage{palatino}
\usepackage{multicol}
\usepackage{float}
\usepackage[hidelinks]{hyperref}
\usepackage[nameinlink, capitalise, noabbrev]{cleveref}
%\usepackage{background}
\usepackage{graphicx}
\usepackage{listliketab}

\usepackage{lipsum}
\usepackage{caption}
\usepackage{marginnote}

\reversemarginpar


\usepackage[dvipsnames]{xcolor}
\usepackage{amsthm}
\usepackage{shadethm}

\setlength{\parindent}{0pt}

\newcommand*\Note{%
	\marginnote[\textcolor{blue}{\raggedright{\LARGE ?}\ Note }]{}%
}
\newcommand*\Warn{%
	\marginnote[\textcolor{red}{\raggedright{\LARGE !}\ Warning }]{}%
}
\reversemarginpar

\newcommand{\N}{\mathbb{N}}
\newcommand{\Z}{\mathbb{Z}}
\newcommand{\I}{\mathbb{I}}
\newcommand{\R}{\mathbb{R}}
\newcommand{\Q}{\mathbb{Q}}
\renewcommand{\qed}{\hfill$\blacksquare$}
\let\newproof\proof
\renewenvironment{proof}{\begin{addmargin}[1em]{0em}\begin{newproof}}{\end{newproof}\end{addmargin}\qed}
% \newcommand{\expl}[1]{\text{\hfill[#1]}$}


\newenvironment{lemma}[2][Lemma]{\begin{trivlist}
		\item[\hskip \labelsep {\bfseries #1}\hskip \labelsep {\bfseries #2.}]}{\end{trivlist}}
\newenvironment{problem}[2][Problem]{\begin{trivlist}
		\item[\hskip \labelsep {\bfseries #1}\hskip \labelsep {\bfseries #2.}]}{\end{trivlist}}
\newenvironment{example}[2][Example]{\begin{trivlist}
		\item[\hskip \labelsep {\bfseries #1}\hskip \labelsep{\bfseries #2.}]}{\end{trivlist}}
\newenvironment{solution}{{\noindent \bfseries Solution\hskip 2ex}}{\qed}
\newenvironment{exercise}[2][Exercise]{\begin{trivlist}
		\item[\hskip \labelsep {\bfseries #1}\hskip \labelsep {\bfseries #2.}]}{\end{trivlist}}
\newenvironment{reflection}[2][Reflection]{\begin{trivlist}
		\item[\hskip \labelsep {\bfseries #1}\hskip \labelsep {\bfseries #2.}]}{\end{trivlist}}
\newenvironment{proposition}[2][Proposition]{\begin{trivlist}
		\item[\hskip \labelsep {\bfseries #1}\hskip \labelsep {\bfseries #2.}]}{\end{trivlist}}
\newenvironment{corollary}[2][Corollary]{\begin{trivlist}
		\item[\hskip \labelsep {\bfseries #1}\hskip \labelsep {\bfseries #2.}]}{\end{trivlist}}

%%%% add package inclusions here, if any
\usepackage{blubase}
%%%%% Define your custom functions and macros here, if any

%%%%%%%%%%%%%%%%%%%% Define the variables below %%%%%%%%%%%%%%%%%%%%%%%%%%%%%%
\def\titlestring{Applications of Ordinary Generating Functions}
\def\scribestring{Your Name}
\def\datestring{Day, Mon, Date Year}


%%%%%%%%%%%%%%%%%%% Page Headers -- Do Not Change %%%%%%%%%%%%%%%%%%%%%%%%%%%
\lhead{\titlestring}
\rhead{Page \thepage}
\cfoot{Concrete Math -- White -- TJHSST}
\renewcommand{\headrulewidth}{0.4pt}
\renewcommand{\footrulewidth}{0.4pt}

%%%%%%%%%%%%%%%%% Begin the document %%%%%%%%%%%%%%%%%%%%%%%%%
\begin{document}
\thispagestyle{plain}  %% no headers on this page

%%%% Do not change these lines %%%%%
\noindent
{\LARGE \textbf{\titlestring}}\\\\
%
{\Large Scribe: \scribestring}\\ \\
{\textbf{Date}: \datestring}


%%%%%%%%%%%%%%%%%%%%%%%%%% YOUR CONTENT GOES HERE %%%%%%%%%%%%%%%%%%%%%%%%%%
\noindent

\section{Revisiting an Old Problem}
\begin{problem}{1}
	Given $\{ a, b, c, d, e \}$, compute the number of ways pick a multiset of size $3$ from these elements such that no item may appear more than 2 times. 
\end{problem}

We will do this using an ordinary generating function. Define an \textbf{ordinary generating function} $A(x)$ of a sequence $a_0, a_1, a_2, \ldots$ as 
\[
	A(x) = \sum_{k=0}^\infty a_k x^k 
\]

Consider the sum 
\[
	\binom{n}{0}^2 + \binom{n}{1}^2 + \binom{n}{2}^2 + \ldots + \binom{n}{n}^2
\]
We will evaluate this in two ways. First, consider the product $(1+x)^n \left( 1 + \frac{1}{x} \right)^n$ and expand these using the binomial theorem: 
\begin{align*}
(1+x)^n \left( 1 + \frac{1}{x} \right)^n &= \left( \sum_{k=0}^n \binom{n}{k} x^k \right) \left( \sum_{j=0}^n \binom{n}{j} x^{-j} \right) \\
&= \sum_{k=0}^n \sum_{j=0}^n \binom{n}{k} \binom{n}{j} x^{k-j} 
\end{align*}
Note that our desired sum is the coefficient of the constant term - ie. all the terms when $k=j$. 

However, notice also that this product can be rearranged as 
\[
	(1+x)^n \left( 1 + \frac{1}{x} \right)^n = (1+x)^{2n} x^{-n} = x^{-n} \sum_{k=0}^n \binom{2n}{k} x^n 
\]
so the constant term is $\binom{2n}{n}$. Therefore, this sum is $\binom{2n}{n}$. $\blacksquare$

We can also show this with a clever combinatorial argument. Suppose we have $n$ blue marbles and $n$ green marbles, and we wish to pick $n$ of these marbles, which can be done in $\binom{2n}{n}$ ways. However, we can also compute this by doing casework on the number of green marbles we pick. We could pick $g$ green marbles, where $g$ can range between $0$ and $n$. 

\section{}
Generating functions are based around the idea that algebra can do a lot of combinatorial problems for us. 
Consider first the \textit{enumerator} polynomial:
\[
	(1+ax)(1+bx)(1+cx) = 1 + (a+b+c) x + (ab + bc + ac)x^2 + abcx^3 
\]
Notice that if we have sets $A, B, C$ with sizes 
This isn't very useful for larger and larger sizes of sets - but what if we just wanted to count the number of possible subsets of size $k$ from a set of $n$ distinct elements? By our logic above, if we set $a = b = c$, we just get 
\[
	(1+x)^3 = 1 + 3x + 3x^2 + x^3 = \sum_{k=0} C_k x^k 
\]
where $C_k$ is the number of subsets of size $k$. This directly gives us the identity 
\[
	(1+x)^n = 
\]
Let's plug a few values of $x$ into this expression. If we plug $x = 1$, we have: 
\[
	2^n = \sum_{k=0}^{n} \binom{n}{k}
\]
If we plug in $x = -1$, we have: 
\[
	0 = \sum_{n=0}^{n} \binom{n}{k} (-1)^k 
\]
or, by parity: 
\[
	\sum_{k even} \binom{n}{k} = \sum_{k odd} \binom{n}{k} 
\]
If we take a derivative with respect to $x$ of the original expression: 
\[
	n(1+x)^{n-1} = \sum_{k=0}^{n} k \binom{n}{k} x^{k-1}
\]
So if we plug in $x = 1$: 
\[
	n2^{n-1} = \sum_{k=0}^{n} k \binom{n}{k}
\]
We can also consider this combinatorially - consider the number of ways to pick a team of any size that has a designated captain. 

What if we want to select $k$ items from $n$, with each element chosen at most twice? We now have to change the way that our original enumerating function worked - notice that we must introduce a quadratic term in the $1+ax$ in a factor, $a^2x^2$, to pick two of any element from the set $A$. Repeating the same logic as above, we have the generating function for this problem 
\[
	(1+x+x^2)^n.
\]
Let us expand it using multinomial coefficients: 
\begin{align*}
 (1+x+x^2)^n = \sum_{i+j+k=n} 1^i x^j (x^2)^k \binom{n}{i, j, k}
\end{align*}
where $\binom{n}{i, j, k} = \frac{n!}{i!j!k!}$, a \textit{multinomial} coefficient. Therefore, we have the coefficient of $x^{j+2k}$ as $\binom{n}{i, j, k}$, which, when written as $\sum C_rx^r$, is the number of ways to pick a multiset of size $r$ with each element used up to two times. This allows us to do the first problem again from a different perspective: 

\begin{problem}{1, again}
	Given $\{ a, b, c, d, e \}$, compute the number of ways pick a multiset of size $3$ from these elements such that no item may appear more than 2 times. 
\end{problem}

We can do this using a generating function - this equivalent to finding the coefficent of $x^3$ in the expansion of $(1+x+x^2)^5$. Notice that the only way to do this is to have $x^3 = (x)(x^2)^1$, or $x^3 = (x)^3$. This means that the number of ways to generate our multiset is $\binom{5}{3, 0, 2} + \binom{5}{1, 1, 3} = 10 + 20 = \boxed{30}$. 

\begin{problem}
	From 18 items, pick between 3 and 5 of each and select a multiset of size 79. 
\end{problem}
\begin{solution}
	This corresponds to finding the coefficient $x^79$ of $(x^3 + x^4 + x^5)^18$, which can be computed using some brute force to be %some large garbage
\end{solution}	

Given a set of $n$ elements, choose a multiset of size $r$ (with no restrictions). We know already that this is The corresponding generating function is 
\[
	(1 + x + x^2 + \ldots + x^r + \ldots)^n
\]
of which we want the coefficient of $x^r$. We don't have to have the coefficients after the $x^r$ term, but the fact that we have an infinitude of terms after the $x^r$ terms gives the following easy form for this generating function: 
\[
	\frac{1}{(1-x)^n}
\]
We can expand this binomially: 
\[
	(1-x)^{-n} = \sum_{k=0}^\infty \frac{\ffact{-n}{-k}}{k!} (-x)^k = \sum_{k=0}^\infty \frac{\rfact{n}{k}}{k!} x^k  
\]	
where $\rfact{n}{k}$ is the \textit{rising factorial}, $n(n+1)(n+2) \ldots (n+k-1)$. This can be rearranged again to the form we already have seen: 
\[
	 = \sum_{k=0}^\infty \binom{n+k-1}{k} x^k  = \sum_{k=0}^\infty \snb{n}{k} x^k
\]	
\begin{problem}{3}
	Given a set of $n$ elements, choose a multiset of size $r$, but each element must be used at least once. 
\end{problem}

\begin{solution}
	Instead of starting each of our factors at $1$, we must start each factor at $x$ as each element must be chosen at least once. This gives us the generating function
\[
	(x + x^2 + \ldots )^n = \left(\frac{x}{1-x}\right)^n 
\]
We can expand: 
\begin{align*}
	&= x^n \sum_{k=0}^\infty \frac{\ffact{-n}{-k}}{k!} (-x)^k \\
	&= \sum_{k=0}^\infty \frac{\rfact{n}{k}}{k!} x^{k+n} \\
	&= \sum_{r=n}^\infty \frac{\rfact{n}{r-n}}{(r-n)!} x^{r}\\
	&= \sum_{r=n}^\infty \binom{r-1}{n-1} x^{r} = \sum_{r=n}^\infty \snb{n}{r-n} x^{r}
\end{align*}
Therefore, we have the number of ways to do this as the coefficient of $x^r$, which is $\binom{r-1}{n-1} = \tsnb{n}{r-n}$. 
\end{solution}

\begin{problem}{4}
	Find the coefficient of $x^23$ in the expansion of $(1+x^5+x^9)^100$. 
\end{problem}
\begin{solution}
	The only way we can do this is by considering $23 = 2\cdot 9 + 5$, so then 

\end{solution}


\end{document}
