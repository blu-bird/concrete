%%%%%%% Lecture Notes Style for Concrete Mathematics %%%%%%%%%%%%%
%%%%%%% Taught by Patrick White, TJHSST %%%%%%%%%%%%%%%%%%%%%%%%%%

\documentclass[11pt,twosided]{article}
%%%%%%%%%%%%%%%%%%%%%%%%%%%%%%%%%%%%%%%%%%%%%%%%%%%%%%%
%%%% HEADER FILE for CONCRETE MATH LECTURE NOTES %%%%%%
%%%%%%%%%%%%%%%%%%%%%%%%%%%%%%%%%%%%%%%%%%%%%%%%%%%%%%%

%%%%%%%%%%%%%%%%%%%%%%%%%%%%%%%%%%%%%%%%%%%%%%%%%%%%%%%%%%%%%%%%%%%
%% This file is included at the top of every lecture notes file %%%
%% It should ONLY be changed by the instructor %%
%%%%%%%%%%%%%%%%%%%%%%%%%%%%%%%%%%%%%%%%%%%%%%%%%%%%%%%%%%%%%%%%%%%

%%%%%%%%%%%%%%%%%%%%%% package inclusions %%%%%%%%%%%%%%%%%%%%
%\usepackage[
%top=2cm,
%bottom=2cm,
%left=3cm,
%right=2cm,
%headheight=17pt, % as per the warning by fancyhdr
%includehead,includefoot,
%heightrounded, % to avoid spurious underfull messages
%]{geometry} 


\usepackage{amsmath,amssymb,amsfonts}
\usepackage{longtable}
\usepackage{mathtools}
\usepackage{amsthm}
\usepackage{enumerate}
\usepackage{fancyhdr}
\pagestyle{fancy}


%%%%%%%%%%%%%%% theorem style definitions %%%%%%%%%%%%%%%%%%%%%%
\newtheorem{theorem}{Theorem}[section]
%\newtheorem{corollary}{Corollary}[theorem]
%\newtheorem{lemma}[theorem]{Lemma}
\newtheorem*{remark}{Remark}
%\newtheorem{example}{Example}[section]
\newtheorem*{definition}{Definition}


%%%%%%%%%%%%%%%% custom function definitions %%%%%%%%%%%%%%%%%%%%%%%%%
%%%%%%%%%%% as class progresses, this section may be enhanced %%%%%%%%

\def\multiset#1#2{\ensuremath{\left(\kern-.3em\left(\genfrac{}{}{0pt}{}{#1}{#2}\right)
		\kern-.3em\right)}} %%% multiset notation
\newcommand\rf[2]{{#1}^{\overline{#2}}} %%%% rising factorial \rf{x}{m}
\newcommand\ff[2]{{#1}^{\underline{#2}}} %%%%% falling factorial \ff{x}{m}
\newcommand{\Perm}[2]{{}^{#1}\!P_{#2}} % permutation
\newcommand{\half}{\ensuremath{\frac{1}{2}}}
\newcommand{\braces}[1]{\left\{#1\right\}}
\newcommand{\set}[1]{\braces{#1}}
\newcommand{\snb}[2]{\ensuremath{\left(\kern-.3em\left(\genfrac{}{}{0pt}{}{#1}{#2}\right)\kern-.3em\right)}}
\newcommand{\fallingfactorial}[1]{^{\underline{#1}}}
\newcommand{\fallfac}[2]{{#1}^{\underline{#2}}}
\newcommand{\stirlingone}[2]{\genfrac[]{0pt}{1}{#1}{#2}}
\newcommand{\stiri}[2]{\stirlingone{#1}{#2}}
\newcommand{\dstirlingone}[2]{\genfrac[]{0pt}{0}{#1}{#2}}
\newcommand{\dstiri}[2]{\dstirlingone{#1}{#2}}
\newcommand{\stirlingtwo}[2]{\genfrac\{\}{0pt}{1}{#1}{#2}}
\newcommand{\stirii}[2]{\stirlingtwo{#1}{#2}}
\newcommand{\dstirlingtwo}[2]{\genfrac\{\}{0pt}{0}{#1}{#2}}
\newcommand{\dstirii}[2]{\dstirlingtwo{#1}{#2}}
\newcommand{\ol}[1]{\overline{#1}}
%%%%%%%
% book stuff
%%%%%%%%%

\usepackage[twoside,outer=1.5in,inner=2in,bottom=1.5in,top=1.5in,marginpar=2in]{geometry} 
\usepackage{scrextend}
\usepackage{palatino}
\usepackage{multicol}
\usepackage{float}
\usepackage[hidelinks]{hyperref}
\usepackage[nameinlink, capitalise, noabbrev]{cleveref}
%\usepackage{background}
\usepackage{graphicx}
\usepackage{listliketab}

\usepackage{lipsum}
\usepackage{caption}
\usepackage{marginnote}

\reversemarginpar


\usepackage[dvipsnames]{xcolor}
\usepackage{amsthm}
\usepackage{shadethm}

\setlength{\parindent}{0pt}

\newcommand*\Note{%
	\marginnote[\textcolor{blue}{\raggedright{\LARGE ?}\ Note }]{}%
}
\newcommand*\Warn{%
	\marginnote[\textcolor{red}{\raggedright{\LARGE !}\ Warning }]{}%
}
\reversemarginpar

\newcommand{\N}{\mathbb{N}}
\newcommand{\Z}{\mathbb{Z}}
\newcommand{\I}{\mathbb{I}}
\newcommand{\R}{\mathbb{R}}
\newcommand{\Q}{\mathbb{Q}}
\renewcommand{\qed}{\hfill$\blacksquare$}
\let\newproof\proof
\renewenvironment{proof}{\begin{addmargin}[1em]{0em}\begin{newproof}}{\end{newproof}\end{addmargin}\qed}
% \newcommand{\expl}[1]{\text{\hfill[#1]}$}


\newenvironment{lemma}[2][Lemma]{\begin{trivlist}
		\item[\hskip \labelsep {\bfseries #1}\hskip \labelsep {\bfseries #2.}]}{\end{trivlist}}
\newenvironment{problem}[2][Problem]{\begin{trivlist}
		\item[\hskip \labelsep {\bfseries #1}\hskip \labelsep {\bfseries #2.}]}{\end{trivlist}}
\newenvironment{example}[2][Example]{\begin{trivlist}
		\item[\hskip \labelsep {\bfseries #1}\hskip \labelsep{\bfseries #2.}]}{\end{trivlist}}
\newenvironment{solution}{{\noindent \bfseries Solution\hskip 2ex}}{\qed}
\newenvironment{exercise}[2][Exercise]{\begin{trivlist}
		\item[\hskip \labelsep {\bfseries #1}\hskip \labelsep {\bfseries #2.}]}{\end{trivlist}}
\newenvironment{reflection}[2][Reflection]{\begin{trivlist}
		\item[\hskip \labelsep {\bfseries #1}\hskip \labelsep {\bfseries #2.}]}{\end{trivlist}}
\newenvironment{proposition}[2][Proposition]{\begin{trivlist}
		\item[\hskip \labelsep {\bfseries #1}\hskip \labelsep {\bfseries #2.}]}{\end{trivlist}}
\newenvironment{corollary}[2][Corollary]{\begin{trivlist}
		\item[\hskip \labelsep {\bfseries #1}\hskip \labelsep {\bfseries #2.}]}{\end{trivlist}}

%%%% add package inclusions here, if any
\usepackage{blubase}
%%%%% Define your custom functions and macros here, if any

%%%%%%%%%%%%%%%%%%%% Define the variables below %%%%%%%%%%%%%%%%%%%%%%%%%%%%%%
\def\titlestring{The Lecture Title}
\def\scribestring{Your Name}
\def\datestring{Day, Mon, Date Year}


%%%%%%%%%%%%%%%%%%% Page Headers -- Do Not Change %%%%%%%%%%%%%%%%%%%%%%%%%%%
\lhead{\titlestring}
\rhead{Page \thepage}
\cfoot{Concrete Math -- White -- TJHSST}
\renewcommand{\headrulewidth}{0.4pt}
\renewcommand{\footrulewidth}{0.4pt}

%%%%%%%%%%%%%%%%% Begin the document %%%%%%%%%%%%%%%%%%%%%%%%%
\begin{document}
\thispagestyle{plain}  %% no headers on this page

%%%% Do not change these lines %%%%%
\noindent
{\LARGE \textbf{\titlestring}}\\\\
%
{\Large Scribe: \scribestring}\\ \\
{\textbf{Date}: \datestring}


%%%%%%%%%%%%%%%%%%%%%%%%%% YOUR CONTENT GOES HERE %%%%%%%%%%%%%%%%%%%%%%%%%%
\noindent

\section{Complex Roots and Bell Numbers}
Consider writing $\frac{1}{1+x^2}$ as a series: 
\[
	\frac{1}{1+x^2} = \sum (-x^2)^n = \sum (-1)^n x^{2n}
\]
Notice that if $n$ is odd, the coefficient of $x^n$ is odd - otherwise, the coefficient is $(-1)^n$. 

However, we could also decompose this into linear terms that are 

Consider the recurrence relation $a_n = -a_{n-2}$, $a_0 = 0$, $a_1 = 1$. If we apply the technique we learned last class, we can consider the characteristic polynomial: 
\[
	x^2 + 1 = 0 \implies x = \pm i
\]
Therefore, $a_n = Ai^n + B(-i)^n$. Plugging in our initial conditions, we have 
\[
	A + B = 0 \quad Ai - Bi = 1 \rightarrow A = -\frac{i}{2} \quad B = \frac{i}{2}
\]
Altogether, this gives $a_n = -\frac{1}{2}i\cdot i^n + \frac{1}{2} \cdot i(-i)^n$. This is a really complex way to express a clearly periodic sequence of period 4 - why would we ever need to use complex numbers if we have functions (ie. $\sin$ and $\cos$) that can do this periodicity for us? 

Let us backtrack and see if we can find a way to do this using trigonometric functions by applying DeMoivre's Theorem. Notice that the roots of the characteristic polynomial can be expressed as
\[
	r_1 = \cos \frac{\pi}{2} + i \sin \frac{\pi}{2} 	
\]
\[
	r_2 = \cos \frac{\pi}{2} - i \sin \frac{\pi}{2} 	
\]
Notice that if we raise to the $n$th power, we have
\[
	a_n = Ar_1^n + Br_2^n = A\left( \cos \frac{\pi n}{2} + i \sin \frac{\pi n}{2} \right) + B\left( \cos \frac{\pi n}{2} - i \sin \frac{\pi n}{2} 	\right) 
\]
If we combine like terms, we have
\[
	a_n = (A+B) \cos \left(\frac{n \pi}{2} \right) + (A-B) \sin \left(\frac{n \pi}{2} \right)
\]
Since all of the terms of the sequence are real, we must force $A$ and $B$ to have the same real part. In general, we can always just solve instead for the coefficients of $\sin$ and $\cos$ instead. 
\[
	a_n = C \cos \left(\frac{n \pi}{2} \right) + D \sin \left(\frac{n \pi}{2} \right)
\]
When we plug in: 
\[
	a_0 = 0 = C \cos 0 \implies C = 0
\]
\[
	a_1 = 1 = D \sin \frac{\pi}{2} \implies D = 1
\]
which gives 
\[
	a_n = \sin \frac{\pi n}{2} 
\]
We can use a similar method to find the roots of unity for $a_n = - a_{n-6}$. 

Let's do this for $a_n = a_{n-1} - a_{n-2}, a_1 = 1, a_2 = 0$. Notice that the characteristic polynomial is
\[
	x^2 - x + 1 = 0 \implies x = \frac{1 \pm \sqrt{3}i}{2}
\]	
This is a 6th root of unity, so we can note that we can simply write this as a the linear combination: 
\[
	a_n = A \cos \frac{n \pi}{3} + B \sin \frac{n \pi}{3}
\]
We plug in our initial conditions: 
\[
	a_1 = \frac{1}{2} A + \frac{3}{2} B = 1
\]
\[
	a_2 = -\frac{1}{2} A + \frac{3}{2} B = 0
\]
\[
	\implies A = 1, B = \frac{1}{\sqrt{3}}
\]
This gives $a_n = \cos \frac{n \pi}{3} + \frac{1}{\sqrt{3}} \sin \frac{n \pi}{3}$

\section{Bell Numbers}
Recall the \textbf{Bell number} $B_n$ is the number of non-empty partitions of a set of $n$ elements. By definition of the Stirling numbers, we have this is equal to 
\[
	B_n = \sum_{k=0}^n \stirii{n}{k}
\]
Recall also that $x^n = \sum_{k=0}^n \stirii{n}{k} \ffact{x}{k}$. This is kind of close to the Bell numbers, but we can use this to get to an explicit formula for the Bell numbers. 

Let $X$ be a random variable. Using linearity of expectation, we have that 
\[
	E[X] = \sum_{k=0}^n \stirii{n}{k} E[\ffact{X}{k}]
\]	
We are looking for a random variable $X$, then, such that $E[\ffact{X}{k}] = 1$ - this will give a computation for $B_n$.

It turns out that one exists! Let $X \sim \text{Poisson}(\lambda)$. The \textbf{Poisson distribution} is the limiting case of the binomial distribution. A diagram of the Poisson distribution is included below. 


The probability distribution function for a Poissson distribution - if $X \sim \text{Poisson}(\lambda)$, then $P(x = k) = \frac{\lambda^k}{k!}e^{-\lambda}$. To prove this is valid, we have to show that: 

$P(x = k) = \sum_{k=0}^\infty P(x=k) = \sum_{k=0}^\infty \frac{\lambda^k}{k!}e^{-\lambda} = e^{-\lambda} \sum_{k=0}^\infty \frac{\lambda^k}{k!} = e^{-\lambda} e^{\lambda} = 1$

We now compute $E[X]$ for the Poisson distribution: 
\[
E[X] = \sum_{k=0}^\infty k P(x=k) = e^{-\lambda} \sum_{k=1}^\infty \frac{\lambda^k}{(k-1)!} = \lambda e^{-\lambda} \sum_{k=1}^\infty \frac{\lambda^{k-1}}{(k-1)!} = \lambda 
\]

We can compute the expected values we want using a moment-generating function, but we can write these out explicitly: 
\[
E[X^n] = \sum_{k=0}^\infty k^n P(x=k) =  \sum_{k=0}^\infty k^n \frac{\lambda^k}{k!}e^{-\lambda}
\]
To find $E[\ffact{X}{k}]$, we can instead look start with $E[t^X]$ and take successive derivatives (this is the \textbf{factorial generating function}): 
\[
E[t^X] = \sum_{k=0}^\infty t^k P(x=k) =  \sum_{k=0}^\infty t^k \frac{\lambda^k}{k!}e^{-\lambda} = e^{-\lambda} \sum_{k=0}^\infty  \frac{(t\lambda)^k}{k!} = e^{-\lambda} (e^{t\lambda}) = e^{\lambda(t-1)}
\]
Notice that if we take successive derivatives with respect to $t$, we have: 
\[
	\dnv{}{t}{n} e^{\lambda(t-1)} = \lambda^n e^{\lambda(t-1)}
\]
Differentiating the sum, we have that this is also 
\[
	E[\ffact{X}{n} t^{X-n}] = e^{-\lambda} \sum_{k=0}^\infty  \frac{\ffact{k}{n} t^{k-n} \lambda^k}{k!} 
\]
When plugging in $t=1$ we obtain that 
\[
	E[\ffact{X}{n}] = \lambda^n
\]
If we let $X \sim \text{Poisson}(\lambda = 1)$, and thus $E[\ffact{X}{n}] = 1$, we can compute $B_n$: 
\[
	E[X^n] = \frac{1}{e}\sum_{k=0}^\infty \frac{k^n}{k!} =  \sum_{k=0}^n \dstirii{n}{k} = B_n
\]

We can also construct a recursive relation for the Bell numbers using a combinatorial argument. Notice that the element $n+1$ of the set $\{1, 2, \ldots n, n+1\}$ can be put in a subset of any choice of $k$ other elements, and then partitioning the remaining $B_{n-k}$ elements. This gives the recursiion relation
\[
	B_{n+1} = \sum_{k=0}^n \binom{n}{k} B_{n-k}
\]





\end{document}
