%%%%%%% Lecture Notes Style for Concrete Mathematics %%%%%%%%%%%%%
%%%%%%% Taught by Patrick White, TJHSST %%%%%%%%%%%%%%%%%%%%%%%%%%

\documentclass[11pt,twosided]{article}
%%%%%%%%%%%%%%%%%%%%%%%%%%%%%%%%%%%%%%%%%%%%%%%%%%%%%%%
%%%% HEADER FILE for CONCRETE MATH LECTURE NOTES %%%%%%
%%%%%%%%%%%%%%%%%%%%%%%%%%%%%%%%%%%%%%%%%%%%%%%%%%%%%%%

%%%%%%%%%%%%%%%%%%%%%%%%%%%%%%%%%%%%%%%%%%%%%%%%%%%%%%%%%%%%%%%%%%%
%% This file is included at the top of every lecture notes file %%%
%% It should ONLY be changed by the instructor %%
%%%%%%%%%%%%%%%%%%%%%%%%%%%%%%%%%%%%%%%%%%%%%%%%%%%%%%%%%%%%%%%%%%%

%%%%%%%%%%%%%%%%%%%%%% package inclusions %%%%%%%%%%%%%%%%%%%%
%\usepackage[
%top=2cm,
%bottom=2cm,
%left=3cm,
%right=2cm,
%headheight=17pt, % as per the warning by fancyhdr
%includehead,includefoot,
%heightrounded, % to avoid spurious underfull messages
%]{geometry} 


\usepackage{amsmath,amssymb,amsfonts}
\usepackage{longtable}
\usepackage{mathtools}
\usepackage{amsthm}
\usepackage{enumerate}
\usepackage{fancyhdr}
\pagestyle{fancy}


%%%%%%%%%%%%%%% theorem style definitions %%%%%%%%%%%%%%%%%%%%%%
\newtheorem{theorem}{Theorem}[section]
%\newtheorem{corollary}{Corollary}[theorem]
%\newtheorem{lemma}[theorem]{Lemma}
\newtheorem*{remark}{Remark}
%\newtheorem{example}{Example}[section]
\newtheorem*{definition}{Definition}


%%%%%%%%%%%%%%%% custom function definitions %%%%%%%%%%%%%%%%%%%%%%%%%
%%%%%%%%%%% as class progresses, this section may be enhanced %%%%%%%%

\def\multiset#1#2{\ensuremath{\left(\kern-.3em\left(\genfrac{}{}{0pt}{}{#1}{#2}\right)
		\kern-.3em\right)}} %%% multiset notation
\newcommand\rf[2]{{#1}^{\overline{#2}}} %%%% rising factorial \rf{x}{m}
\newcommand\ff[2]{{#1}^{\underline{#2}}} %%%%% falling factorial \ff{x}{m}
\newcommand{\Perm}[2]{{}^{#1}\!P_{#2}} % permutation
\newcommand{\half}{\ensuremath{\frac{1}{2}}}
\newcommand{\braces}[1]{\left\{#1\right\}}
\newcommand{\set}[1]{\braces{#1}}
\newcommand{\snb}[2]{\ensuremath{\left(\kern-.3em\left(\genfrac{}{}{0pt}{}{#1}{#2}\right)\kern-.3em\right)}}
\newcommand{\fallingfactorial}[1]{^{\underline{#1}}}
\newcommand{\fallfac}[2]{{#1}^{\underline{#2}}}
\newcommand{\stirlingone}[2]{\genfrac[]{0pt}{1}{#1}{#2}}
\newcommand{\stiri}[2]{\stirlingone{#1}{#2}}
\newcommand{\dstirlingone}[2]{\genfrac[]{0pt}{0}{#1}{#2}}
\newcommand{\dstiri}[2]{\dstirlingone{#1}{#2}}
\newcommand{\stirlingtwo}[2]{\genfrac\{\}{0pt}{1}{#1}{#2}}
\newcommand{\stirii}[2]{\stirlingtwo{#1}{#2}}
\newcommand{\dstirlingtwo}[2]{\genfrac\{\}{0pt}{0}{#1}{#2}}
\newcommand{\dstirii}[2]{\dstirlingtwo{#1}{#2}}
\newcommand{\ol}[1]{\overline{#1}}
%%%%%%%
% book stuff
%%%%%%%%%

\usepackage[twoside,outer=1.5in,inner=2in,bottom=1.5in,top=1.5in,marginpar=2in]{geometry} 
\usepackage{scrextend}
\usepackage{palatino}
\usepackage{multicol}
\usepackage{float}
\usepackage[hidelinks]{hyperref}
\usepackage[nameinlink, capitalise, noabbrev]{cleveref}
%\usepackage{background}
\usepackage{graphicx}
\usepackage{listliketab}

\usepackage{lipsum}
\usepackage{caption}
\usepackage{marginnote}

\reversemarginpar


\usepackage[dvipsnames]{xcolor}
\usepackage{amsthm}
\usepackage{shadethm}

\setlength{\parindent}{0pt}

\newcommand*\Note{%
	\marginnote[\textcolor{blue}{\raggedright{\LARGE ?}\ Note }]{}%
}
\newcommand*\Warn{%
	\marginnote[\textcolor{red}{\raggedright{\LARGE !}\ Warning }]{}%
}
\reversemarginpar

\newcommand{\N}{\mathbb{N}}
\newcommand{\Z}{\mathbb{Z}}
\newcommand{\I}{\mathbb{I}}
\newcommand{\R}{\mathbb{R}}
\newcommand{\Q}{\mathbb{Q}}
\renewcommand{\qed}{\hfill$\blacksquare$}
\let\newproof\proof
\renewenvironment{proof}{\begin{addmargin}[1em]{0em}\begin{newproof}}{\end{newproof}\end{addmargin}\qed}
% \newcommand{\expl}[1]{\text{\hfill[#1]}$}


\newenvironment{lemma}[2][Lemma]{\begin{trivlist}
		\item[\hskip \labelsep {\bfseries #1}\hskip \labelsep {\bfseries #2.}]}{\end{trivlist}}
\newenvironment{problem}[2][Problem]{\begin{trivlist}
		\item[\hskip \labelsep {\bfseries #1}\hskip \labelsep {\bfseries #2.}]}{\end{trivlist}}
\newenvironment{example}[2][Example]{\begin{trivlist}
		\item[\hskip \labelsep {\bfseries #1}\hskip \labelsep{\bfseries #2.}]}{\end{trivlist}}
\newenvironment{solution}{{\noindent \bfseries Solution\hskip 2ex}}{\qed}
\newenvironment{exercise}[2][Exercise]{\begin{trivlist}
		\item[\hskip \labelsep {\bfseries #1}\hskip \labelsep {\bfseries #2.}]}{\end{trivlist}}
\newenvironment{reflection}[2][Reflection]{\begin{trivlist}
		\item[\hskip \labelsep {\bfseries #1}\hskip \labelsep {\bfseries #2.}]}{\end{trivlist}}
\newenvironment{proposition}[2][Proposition]{\begin{trivlist}
		\item[\hskip \labelsep {\bfseries #1}\hskip \labelsep {\bfseries #2.}]}{\end{trivlist}}
\newenvironment{corollary}[2][Corollary]{\begin{trivlist}
		\item[\hskip \labelsep {\bfseries #1}\hskip \labelsep {\bfseries #2.}]}{\end{trivlist}}

%%%% add package inclusions here, if any
\usepackage{blubase}
%%%%% Define your custom functions and macros here, if any

%%%%%%%%%%%%%%%%%%%% Define the variables below %%%%%%%%%%%%%%%%%%%%%%%%%%%%%%
\def\titlestring{Properties of $n!$}
\def\scribestring{Your Name}
\def\datestring{February 21, 2019}


%%%%%%%%%%%%%%%%%%% Page Headers -- Do Not Change %%%%%%%%%%%%%%%%%%%%%%%%%%%
\lhead{\titlestring}
\rhead{Page \thepage}
\cfoot{Concrete Math -- White -- TJHSST}
\renewcommand{\headrulewidth}{0.4pt}
\renewcommand{\footrulewidth}{0.4pt}

%%%%%%%%%%%%%%%%% Begin the document %%%%%%%%%%%%%%%%%%%%%%%%%
\begin{document}
\thispagestyle{plain}  %% no headers on this page

%%%% Do not change these lines %%%%%
\noindent
{\LARGE \textbf{\titlestring}}\\\\
%
{\Large Scribe: \scribestring}\\ \\
{\textbf{Date}: \datestring}


%%%%%%%%%%%%%%%%%%%%%%%%%% YOUR CONTENT GOES HERE %%%%%%%%%%%%%%%%%%%%%%%%%%
\noindent

\section{Stirling's Approximation to $n$!}
\begin{theorem}[Stirling's Approximation]
\[
\lim_{n\to \infty} \frac{n!}{\left(\frac{n}{e} \right)^n \sqrt{2\pi n}} = 1
\]
\end{theorem}
This denominator, $\left(\frac{n}{e} \right)^n \sqrt{2\pi n}$, is \textbf{Stirling's approximation} for $n$! - we can write this as $n! \sim \left(\frac{n}{e} \right)^n \sqrt{2\pi n}$.  

We will show something similar - we will show that $n! = \left( \frac{n}{e} \right)^n \delta \sqrt{n}$, where $\delta \cong \sqrt{2\pi}$. In order to do this, we will instead use $\log(n!) = \sum_{k=1}^n \log k$ to handle the factorial. 

\begin{proof}[Proof (sort of):]
Consider the function $y = \ln x$, and we will find 2 Riemann-style approximations to $\int_1^n \ln x \, dx$. We will do this by looking at the graph of $y = \ln x$: 
% insert graph 

Our first approximation is using the Trapezoid Rule, with bases at the integers from $1$ to $n$. The area of the trapezoids is 
\[
\frac{1}{2} \sum_{k=1}^{n-1} (\ln i + \ln i+1)
\]
which can be expressed as 
\[
\frac{1}{2} \ln 1 + \sum_{k=2}^{n-1} (\ln k + \ln (k+1)) + \frac{1}{2} \ln n = \ln n! - \frac{1}{2} \ln n. 
\] 
This is an underestimate of the area, as all of these trapezoids lie beneath the curve. 

Our second approximation is also using the Trapezoid rule, but we will construct the bases at the half-integers, so the midpoint of the line connecting the ends of the bases are tangent to the curve. (This is with the exception of a rectangle with height $\ln n$ at the end.) This can be evaluated by the sum 
\[
	\sum_{k=2}^{n-1} \ln k + \frac{1}{2} \ln n = \ln n! - \frac{1}{2} \ln n
\]
which is exactly the same approximation. Note that this area is the same as before, but it's an overapproximation for the integral from $\frac{3}{2}$ to $n$, instead of $1$ to $n$. This actually gives us: 
\[
	\int_1^n \ln x \, dx > \ln n! - \frac{1}{2} \ln n > \int_{\frac{3}{2}}^n \ln x \, dx 
\]
This gives us the bounds on $\ln n!$: 
\[
\int_1^n \ln x \, dx + \frac{1}{2} \ln n > \ln n! > \int_{\frac{3}{2}}^n \ln x \, dx + \frac{1}{2} \ln n
\]
Now we compute the integrals (by computing $\int \ln x \, dx$): 
\[
n \ln n - n - (1 \ln 1 - 1) + \frac{1}{2} \ln n > \ln n! > n \ln n - n - \left(\frac{3}{2} \ln \frac{3}{2} - \frac{3}{2} \right) + \frac{1}{2} \ln n
\]
If we simplify both sides: 
\[
\left(n + \frac{1}{2} \right)\ln n - n + 1 > \ln n! > \left( n + \frac{1}{2} \right) \ln n - n - \frac{3}{2} \left( \ln \frac{3}{2} - 1 \right) 
\]
Notice these expressions are almost the same thing, except for the constant term. We can therefore write 
\[
	\ln n! = \left(n + \frac{1}{2} \right)\ln n - n + \delta_n
\]
where $\frac{3}{2} \left( 1 - \ln \frac{3}{2} \right) < \delta_n < 1$. These numerical values give $0.891802 < \delta < 1$. With this, we can now find bounds on $n!$: 
\[
	n! = n^n \sqrt{n} e^{-n} e^\delta = \left( \frac{n}{e} \right)^n \sqrt{n} e^{\delta}
\]
where $e^\delta \in [2.439, 2.718]$. The actual value we are missing, $\sqrt{2\pi}$, has a value of $2.50663$, which is a pretty good approximation. 
\end{proof}
\begin{problem}{1}
	With Stirling's Approximation, approximate $C_n$. 
\end{problem}
\begin{proof}[Solution:]
\[
	C_n = \frac{1}{n+1} \frac{(2n)!}{(n!)^2} = \frac{1}{n+1} \frac{\left(\frac{2n}{e} \right)^{2n} \sqrt{2n} \sqrt{2\pi}}{\left(\frac{n}{e} \right)^{2n} 2\pi n} = \frac{4^{n}}{\sqrt{\pi n}(n+1)}
\]
\end{proof}
As a general rule for asymptotic expansions, we will stick to 
\[
	\ln! = n \ln n - n
\]
which is good enough for our purposes. 

\section{Gamma Function}
\begin{definition} We define the \textbf{Gamma function} $\Gamma(z)$ such that
\[\Gamma(z+1) = \int_0^\infty x^z e^{-x} \, dx \]
\end{definition}
Let's just integrate this by parts: 
\begin{align*}
\int_0^\infty x^z e^{-x} \, dx &= -x^ze^{-x}\Big|_0^\infty + \int_0^\infty ze^{-x}x^{z-1} \, dx \\
&= z \int_0^\infty e^{-x}x^{z-1} \, dx = z \Gamma(z)
\end{align*}
Therefore, we have the multiplicative recursive property 
\[
	Gamma(z+1) = z Gamma(z)
\]
We have to establish a base case first - let's compute $\Gamma(1)$: 
\[
	\Gamma(1) = \int_0^\infty x^0 e^{-x} \, dx = \int_0^\infty e^{-x} \, dx = 1 
\]
This gives us the following: 
\begin{align*}
	\Gamma(1) &= 1 \\
	\Gamma(2) &= 1 \\
	\Gamma(3) &= 2 \\
	\Gamma(4) &= 6 \\
	\Gamma(5) &= 24 \\
	\vdots
\end{align*}
From this we see that $\Gamma(n+1) = n!$ for $n \in \ZZ$. However, we can feasibly extend the definition of the factorial to all real numbers with this function 

With this in mind, compute $\Gamma(\tfrac{1}{2})$. 
\begin{align*}
	\Gamma(\tfrac{1}{2}) &= \int_0^\infty x^{-\tfrac{1}{2}} e^{-x} \, dx \\
	&= \int_0^\infty \frac{e^{-x}}{\sqrt{x}} \, dx \\
	&= \int_0^\infty \frac{e^{-u^2}}{u} \, 2u \, du \\
	&= 2 \int_0^\infty e^{-u^2} \, du
\end{align*}
This lass integral is called the \textit{Gaussian Integral}, which does not have an elementary antiderivative, but has nice definite integrals. Let's call this particular integral $K$.
Then we consider $K^2$: 
\begin{align*}
K^2 &= \int_0^\infty e^{-u^2} \, du \int_0^\infty e^{-v^2} \, dv \\
&= \int_0^\infty \int_0^\infty e^{-u^2} \, du  e^{-v^2} \, dv \\
&= \int_0^{\frac{\pi}{2}} \, d\theta \int_0^\infty r\, dr \, e^{-r^2} \\
&= \frac{\pi}{2} \left(-\frac{1}{2} e^{-r^2} \right)\Big|_0^\infty = \frac{\pi}{4}
\end{align*}
Therefore, we have $K = \frac{\sqrt{\pi}}{2}$ and therefore $\Gamma(\tfrac{1}{2}) = \sqrt{\pi}$. 
	



\end{document}
