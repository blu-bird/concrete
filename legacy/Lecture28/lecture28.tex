%%%%%%% Lecture Notes Style for Concrete Mathematics %%%%%%%%%%%%%
%%%%%%% Taught by Patrick White, TJHSST %%%%%%%%%%%%%%%%%%%%%%%%%%

\documentclass[11pt,twosided]{article}
%%%%%%%%%%%%%%%%%%%%%%%%%%%%%%%%%%%%%%%%%%%%%%%%%%%%%%%
%%%% HEADER FILE for CONCRETE MATH LECTURE NOTES %%%%%%
%%%%%%%%%%%%%%%%%%%%%%%%%%%%%%%%%%%%%%%%%%%%%%%%%%%%%%%

%%%%%%%%%%%%%%%%%%%%%%%%%%%%%%%%%%%%%%%%%%%%%%%%%%%%%%%%%%%%%%%%%%%
%% This file is included at the top of every lecture notes file %%%
%% It should ONLY be changed by the instructor %%
%%%%%%%%%%%%%%%%%%%%%%%%%%%%%%%%%%%%%%%%%%%%%%%%%%%%%%%%%%%%%%%%%%%

%%%%%%%%%%%%%%%%%%%%%% package inclusions %%%%%%%%%%%%%%%%%%%%
%\usepackage[
%top=2cm,
%bottom=2cm,
%left=3cm,
%right=2cm,
%headheight=17pt, % as per the warning by fancyhdr
%includehead,includefoot,
%heightrounded, % to avoid spurious underfull messages
%]{geometry} 


\usepackage{amsmath,amssymb,amsfonts}
\usepackage{longtable}
\usepackage{mathtools}
\usepackage{amsthm}
\usepackage{enumerate}
\usepackage{fancyhdr}
\pagestyle{fancy}


%%%%%%%%%%%%%%% theorem style definitions %%%%%%%%%%%%%%%%%%%%%%
\newtheorem{theorem}{Theorem}[section]
%\newtheorem{corollary}{Corollary}[theorem]
%\newtheorem{lemma}[theorem]{Lemma}
\newtheorem*{remark}{Remark}
%\newtheorem{example}{Example}[section]
\newtheorem*{definition}{Definition}


%%%%%%%%%%%%%%%% custom function definitions %%%%%%%%%%%%%%%%%%%%%%%%%
%%%%%%%%%%% as class progresses, this section may be enhanced %%%%%%%%

\def\multiset#1#2{\ensuremath{\left(\kern-.3em\left(\genfrac{}{}{0pt}{}{#1}{#2}\right)
		\kern-.3em\right)}} %%% multiset notation
\newcommand\rf[2]{{#1}^{\overline{#2}}} %%%% rising factorial \rf{x}{m}
\newcommand\ff[2]{{#1}^{\underline{#2}}} %%%%% falling factorial \ff{x}{m}
\newcommand{\Perm}[2]{{}^{#1}\!P_{#2}} % permutation
\newcommand{\half}{\ensuremath{\frac{1}{2}}}
\newcommand{\braces}[1]{\left\{#1\right\}}
\newcommand{\set}[1]{\braces{#1}}
\newcommand{\snb}[2]{\ensuremath{\left(\kern-.3em\left(\genfrac{}{}{0pt}{}{#1}{#2}\right)\kern-.3em\right)}}
\newcommand{\fallingfactorial}[1]{^{\underline{#1}}}
\newcommand{\fallfac}[2]{{#1}^{\underline{#2}}}
\newcommand{\stirlingone}[2]{\genfrac[]{0pt}{1}{#1}{#2}}
\newcommand{\stiri}[2]{\stirlingone{#1}{#2}}
\newcommand{\dstirlingone}[2]{\genfrac[]{0pt}{0}{#1}{#2}}
\newcommand{\dstiri}[2]{\dstirlingone{#1}{#2}}
\newcommand{\stirlingtwo}[2]{\genfrac\{\}{0pt}{1}{#1}{#2}}
\newcommand{\stirii}[2]{\stirlingtwo{#1}{#2}}
\newcommand{\dstirlingtwo}[2]{\genfrac\{\}{0pt}{0}{#1}{#2}}
\newcommand{\dstirii}[2]{\dstirlingtwo{#1}{#2}}
\newcommand{\ol}[1]{\overline{#1}}
%%%%%%%
% book stuff
%%%%%%%%%

\usepackage[twoside,outer=1.5in,inner=2in,bottom=1.5in,top=1.5in,marginpar=2in]{geometry} 
\usepackage{scrextend}
\usepackage{palatino}
\usepackage{multicol}
\usepackage{float}
\usepackage[hidelinks]{hyperref}
\usepackage[nameinlink, capitalise, noabbrev]{cleveref}
%\usepackage{background}
\usepackage{graphicx}
\usepackage{listliketab}

\usepackage{lipsum}
\usepackage{caption}
\usepackage{marginnote}

\reversemarginpar


\usepackage[dvipsnames]{xcolor}
\usepackage{amsthm}
\usepackage{shadethm}

\setlength{\parindent}{0pt}

\newcommand*\Note{%
	\marginnote[\textcolor{blue}{\raggedright{\LARGE ?}\ Note }]{}%
}
\newcommand*\Warn{%
	\marginnote[\textcolor{red}{\raggedright{\LARGE !}\ Warning }]{}%
}
\reversemarginpar

\newcommand{\N}{\mathbb{N}}
\newcommand{\Z}{\mathbb{Z}}
\newcommand{\I}{\mathbb{I}}
\newcommand{\R}{\mathbb{R}}
\newcommand{\Q}{\mathbb{Q}}
\renewcommand{\qed}{\hfill$\blacksquare$}
\let\newproof\proof
\renewenvironment{proof}{\begin{addmargin}[1em]{0em}\begin{newproof}}{\end{newproof}\end{addmargin}\qed}
% \newcommand{\expl}[1]{\text{\hfill[#1]}$}


\newenvironment{lemma}[2][Lemma]{\begin{trivlist}
		\item[\hskip \labelsep {\bfseries #1}\hskip \labelsep {\bfseries #2.}]}{\end{trivlist}}
\newenvironment{problem}[2][Problem]{\begin{trivlist}
		\item[\hskip \labelsep {\bfseries #1}\hskip \labelsep {\bfseries #2.}]}{\end{trivlist}}
\newenvironment{example}[2][Example]{\begin{trivlist}
		\item[\hskip \labelsep {\bfseries #1}\hskip \labelsep{\bfseries #2.}]}{\end{trivlist}}
\newenvironment{solution}{{\noindent \bfseries Solution\hskip 2ex}}{\qed}
\newenvironment{exercise}[2][Exercise]{\begin{trivlist}
		\item[\hskip \labelsep {\bfseries #1}\hskip \labelsep {\bfseries #2.}]}{\end{trivlist}}
\newenvironment{reflection}[2][Reflection]{\begin{trivlist}
		\item[\hskip \labelsep {\bfseries #1}\hskip \labelsep {\bfseries #2.}]}{\end{trivlist}}
\newenvironment{proposition}[2][Proposition]{\begin{trivlist}
		\item[\hskip \labelsep {\bfseries #1}\hskip \labelsep {\bfseries #2.}]}{\end{trivlist}}
\newenvironment{corollary}[2][Corollary]{\begin{trivlist}
		\item[\hskip \labelsep {\bfseries #1}\hskip \labelsep {\bfseries #2.}]}{\end{trivlist}}

%%%% add package inclusions here, if any
\usepackage{blubase}
%%%%% Define your custom functions and macros here, if any

%%%%%%%%%%%%%%%%%%%% Define the variables below %%%%%%%%%%%%%%%%%%%%%%%%%%%%%%
\def\titlestring{The Lecture Title}
\def\scribestring{Your Name}
\def\datestring{Day, Mon, Date Year}


%%%%%%%%%%%%%%%%%%% Page Headers -- Do Not Change %%%%%%%%%%%%%%%%%%%%%%%%%%%
\lhead{\titlestring}
\rhead{Page \thepage}
\cfoot{Concrete Math -- White -- TJHSST}
\renewcommand{\headrulewidth}{0.4pt}
\renewcommand{\footrulewidth}{0.4pt}

%%%%%%%%%%%%%%%%% Begin the document %%%%%%%%%%%%%%%%%%%%%%%%%
\begin{document}
\thispagestyle{plain}  %% no headers on this page

%%%% Do not change these lines %%%%%
\noindent
{\LARGE \textbf{\titlestring}}\\\\
%
{\Large Scribe: \scribestring}\\ \\
{\textbf{Date}: \datestring}


%%%%%%%%%%%%%%%%%%%%%%%%%% YOUR CONTENT GOES HERE %%%%%%%%%%%%%%%%%%%%%%%%%%
\noindent

\section{Polya's Enumeration Theorem}
\textbf{Polya's Enumeration Theorem} is a generalization of Burnside's Lemma. 

We will first prove Burnside's Lemma. Recall the statement of Burnside's Lemma: for a set $S$ and a permutation group $G$, then the number of equivalence classes of $S$ under $G$ is given by 
\[
	\frac{1}{|G|} \sum_{\pi \in G} |fix(\pi)| 
\]
where $fix(\pi)$ is the set of elements that are fixed by $\pi$. 

Recall that we proved last time the Orbit Stabilizer Theorem: 
For any element $s \in S$ and the permutation group $G$, we have that 
\[
	|orbit(s)||stabilizer(s)| = |G|.
\]
Recall also that $orbit(s)$ is the set of all other elements in the set $S$ that permutations on $s$ can reach, and the stabilizer group $stabilizer(s)$ is the group of permutations acting on $s$ that leave it fixed. 


\begin{proof}[First Proof of Burnside's]
We want to count equivalence classes there are, so essentially we are looking at \textit{how many} orbits there are. 

Let's do this by considering the fixed points in the table $S \times G \rightarrow S$. We will do this in two ways. The first way is to simply do it naively, adding up the number of ways that each $\pi$ in $G$ fixes (down in columns). This gives the expression 
\[
	\sum_{\pi \in G} |fix(\pi)|.
\]

If we add across in rows, however, we similarly get the sum 
\[
	\sum_{s \in S} |stab(s)|.
\]
By the Orbit-Stabilizer Theorem, this is 
\[
	\sum_{s \in S} \frac{|G|}{|orbit(s)|}.
\]
We can now break up $S$ into distinct orbits and sum over the orbits $O_1, O_2, \ldots O_k$. We can rewrite this as a double sum: 
\[
	\sum_{O_i \text{ orbits}}\sum_{s \in O_i} \frac{|G|}{|orbit(s)|}.
\]
What is the size of $orbit(s)$ if $s \in O_i$? Well, it has to be $|O_i|$: 
\[
	|G| \sum_{O_i \text{ orbits}}\sum_{s \in O_i} \frac{1}{|O_i|}.
\]

The inner sum can be evaluated by noting that if we're adding up $\frac{1}{|O_i|}$ for $|O_i|$ elements, this has to be 1: 
\[
	|G| \sum_{O_i \text{ orbits}} 1
\]
This is therefore $|G| (\text{the \# of orbits})$, so the number of orbits or the number of equivalence classes is given by Burnside's statement, as desired. 

\end{proof}

Let's build up to Polya's Enumeration Theorem through a series of definitions. 

Let $X$ be a set (of vertices) and $S$ is a set of functions on $X$, $X \rightarrow C$, where $C$ is another set (of colors). Let $G$ be a group operating on $X$. We will say that if $f_i, f_j \in S$ are \textit{equivalent}, ie. $(f_i \cong f_2)$ if and only if 
$f_i(x) = f_j(\pi(x))$ for all $x \in X$ for some $\pi \in G$. 

This is fairly abstract, but our functions $f_i$ are essentially colorings of objects, and all we're saying that two colorings are the same if they transform to the same thing under an element in the group. 

Example: $c_{12}$ and $c_{15}$ in the table, and test it for all 
elements under the permutation $H$.

We define the \textit{weight} of an element $c \in C$ as $w(c)$, and the \textit{weight} of a function $f: X \rightarrow C$, $W(f)$, is the product of the weights in its range, namely
\[
	\prod_{x \in X} w(f(x))
\]
where the weight of both functions and elements map to an abelian group and a field. 

The \textit{inventory} of a set of functions is 
\[
	\sum_{f_i} W(f_i)
\]

To be concrete, suppose if we consider the two coloring of a square with red and green, and let $w(R) = r$ and $w(G) = g$, then from the table, we can see that coloring $c_1$ has weight $r^4$ and coloring $c_2$ has weight $rg^3$, etc. 

For example, what is the inventory of all the two-colorings of a square, $c_1, c_2, \ldots c_{16}$? By enumeration, this is $r^4 + 4r^3 g + 6r^2g^2 + 4 rg^3 + g^4 = (r + g)^4$. This makes sense from a generating functions perspective, as each of the four vertices can be assigned either the weight $r$ or $g$. 

Under the action of a group, \textbf{Polya's Enumeration Theorem} will tell us the correct coefficients of this generating function, ie. the correct number colorings that are independent of each other under a group. 

We have to introduce one last definition - let the \textit{cycle index (polynomial)} of a permutation group $G$ be 
\[
	P_G(x_1, x_2, \ldots, x_k, \ldots) = \frac{1}{|G|}\sum_{\pi\in G} x_1^{b_1} x_2^{b_2} \ldots x_k^{b_k} \ldots
\]
where $b_k$ is the number of cycles of length $k$ in $\pi$ and the subscript of $x_i$ is the size of the cycle. 

For example, in the dihedral group $D_4$, ie. rotations and reflections of a square, we have the following correspondences: 
\[
	e \rightarrow x_1^4 \quad r_{90} \rightarrow x_4^1 \quad r_{180} \rightarrow x_2^2 \quad r_{270} \rightarrow x_4^1
\]	
\[
	H \rightarrow x_2^2 \quad V \rightarrow x_2^2 \quad L \rightarrow x_1^2x_2 \quad R \rightarrow x_1^2x_2
\]	
Thus, the cycle index is 
\[
	\frac{1}{8}(x_1^4 + 3x_2^2 + 2x_1^2 x_2 + 2 x_4)
\]

Finally, we have Polya's Enumeration Theorem, wrapped in these definitions: 
\begin{theorem}
The inventory of the equivalence classes of functions $f: X \rightarrow C$ under a permutation group $G$ is given by 
\[
	P_G\left(\sum w(c), \sum w^2(c), \sum w^3(c), \ldots, \sum w^k(c), \ldots \right) 
\]	
where the sums are over $C$. 
\end{theorem}

If we apply this to the two-colorings, we get the desired 
\[
	r^4 + r^3 g + 2 r^2 g^2 + rg^3 + g^4.
\]

And a simple corollary:
The number of equivalence classes is just given by letting each of the weights be $1$, or letting each of the inputs in $P_G$ be $|C|$. 


As an example, let's consider the vertices in a cube... 

The cycle index is 
\[
\frac{}{24}
\]


This gives 10. 

Polya was notorious for using this in chemistry to count the number of molecules with differing functional groups that were different but rotationally, the same. One example - consider a methane center and the possibilities of a methyl group, ethyl group, a hydrogen, and a chlorine group. These give 4 "colors" for the 4 vertices of this tetrahedron, so the number of distinct molecules is the number of distinct 4-colorings of the vertices of a tetrahedron. 

To do this, we find the cycle index for a tetrahedron: 

This gives 36. 

What if we mandate that the molecule has to have one molecule: 

This gives 21. 
 

















\end{document}
