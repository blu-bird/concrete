%%%%%%% Lecture Notes Style for Concrete Mathematics %%%%%%%%%%%%%
%%%%%%% Taught by Patrick White, TJHSST %%%%%%%%%%%%%%%%%%%%%%%%%%

\documentclass[11pt,twosided]{article}
%%%%%%%%%%%%%%%%%%%%%%%%%%%%%%%%%%%%%%%%%%%%%%%%%%%%%%%
%%%% HEADER FILE for CONCRETE MATH LECTURE NOTES %%%%%%
%%%%%%%%%%%%%%%%%%%%%%%%%%%%%%%%%%%%%%%%%%%%%%%%%%%%%%%

%%%%%%%%%%%%%%%%%%%%%%%%%%%%%%%%%%%%%%%%%%%%%%%%%%%%%%%%%%%%%%%%%%%
%% This file is included at the top of every lecture notes file %%%
%% It should ONLY be changed by the instructor %%
%%%%%%%%%%%%%%%%%%%%%%%%%%%%%%%%%%%%%%%%%%%%%%%%%%%%%%%%%%%%%%%%%%%

%%%%%%%%%%%%%%%%%%%%%% package inclusions %%%%%%%%%%%%%%%%%%%%
%\usepackage[
%top=2cm,
%bottom=2cm,
%left=3cm,
%right=2cm,
%headheight=17pt, % as per the warning by fancyhdr
%includehead,includefoot,
%heightrounded, % to avoid spurious underfull messages
%]{geometry} 


\usepackage{amsmath,amssymb,amsfonts}
\usepackage{longtable}
\usepackage{mathtools}
\usepackage{amsthm}
\usepackage{enumerate}
\usepackage{fancyhdr}
\pagestyle{fancy}


%%%%%%%%%%%%%%% theorem style definitions %%%%%%%%%%%%%%%%%%%%%%
\newtheorem{theorem}{Theorem}[section]
%\newtheorem{corollary}{Corollary}[theorem]
%\newtheorem{lemma}[theorem]{Lemma}
\newtheorem*{remark}{Remark}
%\newtheorem{example}{Example}[section]
\newtheorem*{definition}{Definition}


%%%%%%%%%%%%%%%% custom function definitions %%%%%%%%%%%%%%%%%%%%%%%%%
%%%%%%%%%%% as class progresses, this section may be enhanced %%%%%%%%

\def\multiset#1#2{\ensuremath{\left(\kern-.3em\left(\genfrac{}{}{0pt}{}{#1}{#2}\right)
		\kern-.3em\right)}} %%% multiset notation
\newcommand\rf[2]{{#1}^{\overline{#2}}} %%%% rising factorial \rf{x}{m}
\newcommand\ff[2]{{#1}^{\underline{#2}}} %%%%% falling factorial \ff{x}{m}
\newcommand{\Perm}[2]{{}^{#1}\!P_{#2}} % permutation
\newcommand{\half}{\ensuremath{\frac{1}{2}}}
\newcommand{\braces}[1]{\left\{#1\right\}}
\newcommand{\set}[1]{\braces{#1}}
\newcommand{\snb}[2]{\ensuremath{\left(\kern-.3em\left(\genfrac{}{}{0pt}{}{#1}{#2}\right)\kern-.3em\right)}}
\newcommand{\fallingfactorial}[1]{^{\underline{#1}}}
\newcommand{\fallfac}[2]{{#1}^{\underline{#2}}}
\newcommand{\stirlingone}[2]{\genfrac[]{0pt}{1}{#1}{#2}}
\newcommand{\stiri}[2]{\stirlingone{#1}{#2}}
\newcommand{\dstirlingone}[2]{\genfrac[]{0pt}{0}{#1}{#2}}
\newcommand{\dstiri}[2]{\dstirlingone{#1}{#2}}
\newcommand{\stirlingtwo}[2]{\genfrac\{\}{0pt}{1}{#1}{#2}}
\newcommand{\stirii}[2]{\stirlingtwo{#1}{#2}}
\newcommand{\dstirlingtwo}[2]{\genfrac\{\}{0pt}{0}{#1}{#2}}
\newcommand{\dstirii}[2]{\dstirlingtwo{#1}{#2}}
\newcommand{\ol}[1]{\overline{#1}}
%%%%%%%
% book stuff
%%%%%%%%%

\usepackage[twoside,outer=1.5in,inner=2in,bottom=1.5in,top=1.5in,marginpar=2in]{geometry} 
\usepackage{scrextend}
\usepackage{palatino}
\usepackage{multicol}
\usepackage{float}
\usepackage[hidelinks]{hyperref}
\usepackage[nameinlink, capitalise, noabbrev]{cleveref}
%\usepackage{background}
\usepackage{graphicx}
\usepackage{listliketab}

\usepackage{lipsum}
\usepackage{caption}
\usepackage{marginnote}

\reversemarginpar


\usepackage[dvipsnames]{xcolor}
\usepackage{amsthm}
\usepackage{shadethm}

\setlength{\parindent}{0pt}

\newcommand*\Note{%
	\marginnote[\textcolor{blue}{\raggedright{\LARGE ?}\ Note }]{}%
}
\newcommand*\Warn{%
	\marginnote[\textcolor{red}{\raggedright{\LARGE !}\ Warning }]{}%
}
\reversemarginpar

\newcommand{\N}{\mathbb{N}}
\newcommand{\Z}{\mathbb{Z}}
\newcommand{\I}{\mathbb{I}}
\newcommand{\R}{\mathbb{R}}
\newcommand{\Q}{\mathbb{Q}}
\renewcommand{\qed}{\hfill$\blacksquare$}
\let\newproof\proof
\renewenvironment{proof}{\begin{addmargin}[1em]{0em}\begin{newproof}}{\end{newproof}\end{addmargin}\qed}
% \newcommand{\expl}[1]{\text{\hfill[#1]}$}


\newenvironment{lemma}[2][Lemma]{\begin{trivlist}
		\item[\hskip \labelsep {\bfseries #1}\hskip \labelsep {\bfseries #2.}]}{\end{trivlist}}
\newenvironment{problem}[2][Problem]{\begin{trivlist}
		\item[\hskip \labelsep {\bfseries #1}\hskip \labelsep {\bfseries #2.}]}{\end{trivlist}}
\newenvironment{example}[2][Example]{\begin{trivlist}
		\item[\hskip \labelsep {\bfseries #1}\hskip \labelsep{\bfseries #2.}]}{\end{trivlist}}
\newenvironment{solution}{{\noindent \bfseries Solution\hskip 2ex}}{\qed}
\newenvironment{exercise}[2][Exercise]{\begin{trivlist}
		\item[\hskip \labelsep {\bfseries #1}\hskip \labelsep {\bfseries #2.}]}{\end{trivlist}}
\newenvironment{reflection}[2][Reflection]{\begin{trivlist}
		\item[\hskip \labelsep {\bfseries #1}\hskip \labelsep {\bfseries #2.}]}{\end{trivlist}}
\newenvironment{proposition}[2][Proposition]{\begin{trivlist}
		\item[\hskip \labelsep {\bfseries #1}\hskip \labelsep {\bfseries #2.}]}{\end{trivlist}}
\newenvironment{corollary}[2][Corollary]{\begin{trivlist}
		\item[\hskip \labelsep {\bfseries #1}\hskip \labelsep {\bfseries #2.}]}{\end{trivlist}}

%%%% add package inclusions here, if any

%%%%% Define your custom functions and macros here, if any

%%%%%%%%%%%%%%%%%%%% Define the variables below %%%%%%%%%%%%%%%%%%%%%%%%%%%%%%
\def\titlestring{The Lecture Title}
\def\scribestring{Your Name}
\def\datestring{Day, Mon, Date Year}


%%%%%%%%%%%%%%%%%%% Page Headers -- Do Not Change %%%%%%%%%%%%%%%%%%%%%%%%%%%
\lhead{\titlestring}
\rhead{Page \thepage}
\cfoot{Concrete Math -- White -- TJHSST}
\renewcommand{\headrulewidth}{0.4pt}
\renewcommand{\footrulewidth}{0.4pt}

%%%%%%%%%%%%%%%%% Begin the document %%%%%%%%%%%%%%%%%%%%%%%%%
\begin{document}
\thispagestyle{plain}  %% no headers on this page

%%%% Do not change these lines %%%%%
\noindent
{\LARGE \textbf{\titlestring}}\\\\
%
{\Large Scribe: \scribestring}\\ \\
{\textbf{Date}: \datestring}


%%%%%%%%%%%%%%%%%%%%%%%%%% YOUR CONTENT GOES HERE %%%%%%%%%%%%%%%%%%%%%%%%%%
\noindent

\section{Undetermined Coefficients}
Given $a_n + A a_{n-1} + Ba_{n-2} = 0$, we can solve this homogeneous recurrence relation with a shorter method (with undetermined coefficients). We will find the solution to this recurrence using a 

Notice that given this relation, we can notice that the following sum is still zero: 
\[
	\sum_{n=0}^\infty (a_n + Aa_{n-1} + Ba_{n-2})x^n = 0
\]
We can now simplify this in terms of the generating function $g(x)$ for $a_n$: 
\[
 \implies (g(x) - a_0 - a_1 x) + Ax(g(x) - a_0) + Bx^2 g(x) = 0
\]
We collect like terms: 
\[
	g(x) ( 1 + Ax + Bx^2) = (a_1 + Aa_0)x + a_0 = Cx + D
\]
This implies that we can write this generating function as 
\[
	g(x) = \frac{Cx+D}{1 + Ax + Bx^2}
\]
We can now use a partial fraction decomposition to decompose $1 + Ax + Bx^2$ into a sum of reciprocals of (distinct) linear terms $1-r_1x$, $1-r_2x$:
\[
	g(x) = \frac{Cx+D}{1 + Ax + Bx^2} = \frac{E}{1-r_1x} + \frac{F}{1-r_2x}
\]
where $(1-r_1x)(1-r_2x) = 1 + Ax + Bx^2$. This decomposition directly leads to the explicit form
\[
	a_n = Er_1^n + F r_2^n
\]
Notice, however, that $(x-r_1)(x-r_2) = x^2 + Ax + B$, so instead we can use the following method to determine these exponents directly: 

\textbf{Given $a_n + A a_{n-1} + Ba_{n-2} = 0$:}
\begin{enumerate}
\item Solve $x^2 + Ax + B = 0 = (x-r_1)(x-r_2)$. 
\item Write $a_n = C_1r_1^n + C_2r_2^n$. 
\item Use initial conditions to solve for $C_1$, $C_2$. 
\end{enumerate}

\begin{problem}{1}
Let $a_n = 7a_{n-1} -10a_{n-2}$, with $a_0 = 0, a_1 = 7$. Find an explicit formula for $a_n$. 
\end{problem}
\begin{solution}
	We write this recurrence relation as 
	\[
		a_n - 7a_{n-1} + 10a_{n-2} = 0
	\]
	We can solve the quadratic $x^2 - 7x + 10 = 0$, which gives the roots $5$ and $2$. Therefore, 
	\[
		a_n = C_12^n + C_25^n
	\]
	Plugging in our initial conditions gives
	\[
		0 = C_1 + C_2 \quad 7 = 2C_1 + 5C_2
	\]	
	giving $C_1 = -\frac{7}{3}$, $C_2 = \frac{7}{3}$. Therefore, 
	\[
		a_n = \frac{7}{3} (5^n - 2^n)
	\]
\end{solution}

What if we have now that $r_1 = r_2$? Recall that we had the form for our generating function above, but when we do the partial fraction decomposition, we must now have terms of the form
\[
	g(x) = \frac{Cx+D}{1 + Ax + Bx^2} = \frac{Cx+D}{(1-rx)^2} =  \frac{E}{1-rx} + \frac{F}{(1-rx)^2}
\]
This yields the explicit form 
\[
	a_n = Er^n + F(n+1)r^n = E'r^n + F'nr^n
\]
for some $E', F'$. We can now do the same process as above if the roots have multiplicity greater than 1. For example, if a root $r$ has multiplicity 3, we will have terms in $a_n$ that look like 
\[
	a_n = Er^n + Fn r^n + Gn^2 r^n.
\]
This takes a similar form for roots of greater multiplicity. 
\begin{problem}{2}
	Find an explicit form for $a_n$: $a_n + 6a_{n-1} + 9a_{n-2} = 0$, $a_1 =1$, $a_2 = 2$. 
\end{problem}
\begin{solution}
\[
	a_n = -\frac{8}{9}(-3)^n + \frac{5}{9}n(-3)^n 
\]
\end{solution}

We will revisit the recurrence relation $a_n = 2a_{n-1} + n - 1$, and look at doing this relatively judiciously. By using the generating function, we have
\[
	g(x) = 1 + 2xg(x) + \sum_{n=1}^\infty nx^n - \sum_{n=1}^\infty x^n 
\]
\begin{align*}
	\rightarrow g(x)(1-2x) &= 1 + \sum_{n=1}^\infty nx^n - \sum_{n=1}^\infty x^n \\
	&= 1 + \frac{x}{(1-x)^2} - \left( \frac{1}{1-x} - 1 \right) \\
	&= 2 + \frac{x}{(1-x)^2} - \frac{1}{1-x} \\
	&= 2 + \frac{2x-1}{(1-x)^2}
\end{align*}
\[
	\implies g(x) = \frac{2}{1-2x} - \frac{1}{(1-x)^2}
\]
which gives us the desired result much more quickly. 










\end{document}
