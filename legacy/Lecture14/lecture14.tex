%%%%%%% Lecture Notes Style for Concrete Mathematics %%%%%%%%%%%%%
%%%%%%% Taught by Patrick White, TJHSST %%%%%%%%%%%%%%%%%%%%%%%%%%

\documentclass[11pt,twosided]{article}
%%%%%%%%%%%%%%%%%%%%%%%%%%%%%%%%%%%%%%%%%%%%%%%%%%%%%%%
%%%% HEADER FILE for CONCRETE MATH LECTURE NOTES %%%%%%
%%%%%%%%%%%%%%%%%%%%%%%%%%%%%%%%%%%%%%%%%%%%%%%%%%%%%%%

%%%%%%%%%%%%%%%%%%%%%%%%%%%%%%%%%%%%%%%%%%%%%%%%%%%%%%%%%%%%%%%%%%%
%% This file is included at the top of every lecture notes file %%%
%% It should ONLY be changed by the instructor %%
%%%%%%%%%%%%%%%%%%%%%%%%%%%%%%%%%%%%%%%%%%%%%%%%%%%%%%%%%%%%%%%%%%%

%%%%%%%%%%%%%%%%%%%%%% package inclusions %%%%%%%%%%%%%%%%%%%%
%\usepackage[
%top=2cm,
%bottom=2cm,
%left=3cm,
%right=2cm,
%headheight=17pt, % as per the warning by fancyhdr
%includehead,includefoot,
%heightrounded, % to avoid spurious underfull messages
%]{geometry} 


\usepackage{amsmath,amssymb}
\usepackage{mathtools}
\usepackage{amsthm}

\usepackage{fancyhdr}
\pagestyle{fancy}


%%%%%%%%%%%%%%% theorem style definitions %%%%%%%%%%%%%%%%%%%%%%
\newtheorem{theorem}{Theorem}[section]
%\newtheorem{corollary}{Corollary}[theorem]
%\newtheorem{lemma}[theorem]{Lemma}
\newtheorem*{remark}{Remark}
%\newtheorem{example}{Example}[section]
\newtheorem*{definition}{Definition}


%%%%%%%%%%%%%%%% custom function definitions %%%%%%%%%%%%%%%%%%%%%%%%%
%%%%%%%%%%% as class progresses, this section may be enhanced %%%%%%%%

\def\multiset#1#2{\ensuremath{\left(\kern-.3em\left(\genfrac{}{}{0pt}{}{#1}{#2}\right)
		\kern-.3em\right)}} %%% multiset notation
\newcommand\rf[2]{{#1}^{\overline{#2}}} %%%% rising factorial \rf{x}{m}
\newcommand\ff[2]{{#1}^{\underline{#2}}} %%%%% falling factorial \ff{x}{m}


%%%%%%%
% book stuff
%%%%%%%%%

\usepackage[twoside,outer=1.5in,inner=2in,bottom=1.5in,top=1.5in,marginpar=2in]{geometry} 
\usepackage{amsmath,amsthm,amssymb,scrextend}
\usepackage{fancyhdr}
\usepackage{palatino}
\usepackage{multicol}
%\usepackage{background}
\usepackage{graphicx}
\usepackage{listliketab}

\usepackage{lipsum}
\usepackage{caption}
\usepackage{marginnote}

\reversemarginpar


\usepackage[dvipsnames]{xcolor}
\usepackage{amsthm}
\usepackage{shadethm}

\newcommand*\Note{%
	\marginnote[\textcolor{blue}{\raggedright{\LARGE ?}\ Note }]{}%
}
\newcommand*\Warn{%
	\marginnote[\textcolor{red}{\raggedright{\LARGE !}\ Warning }]{}%
}
\reversemarginpar

\newcommand{\N}{\mathbb{N}}
\newcommand{\Z}{\mathbb{Z}}
\newcommand{\I}{\mathbb{I}}
\newcommand{\R}{\mathbb{R}}
\newcommand{\Q}{\mathbb{Q}}
\renewcommand{\qed}{\hfill$\blacksquare$}
\let\newproof\proof
\renewenvironment{proof}{\begin{addmargin}[1em]{0em}\begin{newproof}}{\end{newproof}\end{addmargin}\qed}
% \newcommand{\expl}[1]{\text{\hfill[#1]}$}


\newenvironment{lemma}[2][Lemma]{\begin{trivlist}
		\item[\hskip \labelsep {\bfseries #1}\hskip \labelsep {\bfseries #2.}]}{\end{trivlist}}
\newenvironment{problem}[2][Problem]{\begin{trivlist}
		\item[\hskip \labelsep {\bfseries #1}\hskip \labelsep {\bfseries #2.}]}{\end{trivlist}}
\newenvironment{example}[2][Example]{\begin{trivlist}
		\item[\hskip \labelsep {\bfseries #1}\hskip \labelsep{\bfseries #2.}]}{\end{trivlist}}
\newenvironment{solution}{{\noindent \bfseries Solution\hskip 2ex}}{\qed}
\newenvironment{exercise}[2][Exercise]{\begin{trivlist}
		\item[\hskip \labelsep {\bfseries #1}\hskip \labelsep {\bfseries #2.}]}{\end{trivlist}}
\newenvironment{reflection}[2][Reflection]{\begin{trivlist}
		\item[\hskip \labelsep {\bfseries #1}\hskip \labelsep {\bfseries #2.}]}{\end{trivlist}}
\newenvironment{proposition}[2][Proposition]{\begin{trivlist}
		\item[\hskip \labelsep {\bfseries #1}\hskip \labelsep {\bfseries #2.}]}{\end{trivlist}}
\newenvironment{corollary}[2][Corollary]{\begin{trivlist}
		\item[\hskip \labelsep {\bfseries #1}\hskip \labelsep {\bfseries #2.}]}{\end{trivlist}}



%%%% add package inclusions here, if any
\usepackage{blubase}
%%%%% Define your custom functions and macros here, if any

%%%%%%%%%%%%%%%%%%%% Define the variables below %%%%%%%%%%%%%%%%%%%%%%%%%%%%%%
\def\titlestring{Application of Generating Functions}
\def\scribestring{Your Name}
\def\datestring{Day, Mon, Date Year}


%%%%%%%%%%%%%%%%%%% Page Headers -- Do Not Change %%%%%%%%%%%%%%%%%%%%%%%%%%%
\lhead{\titlestring}
\rhead{Page \thepage}
\cfoot{Concrete Math -- White -- TJHSST}
\renewcommand{\headrulewidth}{0.4pt}
\renewcommand{\footrulewidth}{0.4pt}

%%%%%%%%%%%%%%%%% Begin the document %%%%%%%%%%%%%%%%%%%%%%%%%
\begin{document}
\thispagestyle{plain}  %% no headers on this page

%%%% Do not change these lines %%%%%
\noindent
{\LARGE \textbf{\titlestring}}\\\\
%
{\Large Scribe: \scribestring}\\ \\
{\textbf{Date}: \datestring}


%%%%%%%%%%%%%%%%%%%%%%%%%% YOUR CONTENT GOES HERE %%%%%%%%%%%%%%%%%%%%%%%%%%
\noindent

\section{From the Review: Stirling}
\begin{problem}{1}
	Prove 
	\[
		\sum_{j=0}^n \stirii{n}{k} \stiri{n}{k} = \delta_{nk}
	\]
\end{problem}
\begin{proof}
We will prove this using coordinate transformations from the polynomial basis to the falling factorials, and vice versa. Note that:
\[
	x^n = \sum_{j=0}^n \stirii{n}{k} \ffact{x}{j}
\]
and 
\[
	\ffact{x}{j} = \sum_{k=0}^j \stiri{j}{k} x^k 
\]
Note that we can allow this last sum to go from $0$ to $n$, which is fine because we are only adding zeros. This gives us: 
\[
	x^n = \sum_{j=0}^n \stirii{n}{k}\sum_{k=0}^n \stiri{j}{k} x^k 
\]
Interchanging the sums freely, we can see that: 
\[
	x^n = \sum_{k=0}^n \sum_{j=0}^n \stirii{n}{k} \stiri{j}{k} x^k 
\]
Notice that we \textbf{must} have that $x^k = x^n$ for the internal sum coefficient to be non-zero - otherwise, the sum is always zero. Therefore, we must have that: 
\[
	\sum_{j=0}^n \stirii{n}{k} \stiri{j}{k}  = \delta_{nk}
\]
\end{proof}

\section{Applications of Generating Functions}
\begin{problem}{2}
A food-counting problem - what is the number of ways to fill a bag with $n$ pieces of fruit given:
\begin{itemize}
\item The number of apples is even. 
\item The number of bananas is a multiple of 5.
\item The number of oranges is $\leq$ 4. 
\item The number of pears is $\leq$ 1. 
\end{itemize}
\end{problem}

\begin{solution}
We will solve this with a generating function. Notice that we can express each of these conditions into a generating function: 
\[
	(1 + x^2 + x^4 + \ldots ) (1 + x^5 + x^10 + \ldots ) (1+x+x^2+x^3+x^4)(1+x)
\]
Notice that we can manipulate these as follows: 
\[
	= \frac{1}{1-x^2} \frac{1}{1-x^5} \frac{1-x^5}{1-x} (1+x) = \frac{1}{(1-x)^2}
\]
\end{solution}

\begin{problem}
	Find the number of ways to make change for a dollar (in pennies, nickels, dimes, quarters, or half-dollars). 
\end{problem}
\begin{solution}
This is the coefficient of $x^{100}$ in 
\[
	= \frac{1}{(1-x)(1-x^5)(1-x^{10})(1-x^{25})(1-x^{50})}
\]
which happens to be 293. 
\end{solution}
\begin{problem}
	Find the number of ways to make change for a dollar if we have access to any denomination. 
\end{problem}
\begin{solution}
This gives us the generating function for integer partitions: 
\[
	g(x) = \frac{1}{(1-x)}\frac{1}{(1-x^2)}\frac{1}{(1-x^3)} \ldots = \Pi_{k>0} \frac{1}{1-x^k} = \sum_{r>0} P(r) x^r
\]
where $P(r)$ is the number of integer partitions of $r$. This does not have a particularly pretty closed form. 
\end{solution}

\begin{theorem}
The number of partitions of $n$ into distinct parts (using every number exactly once) is equal to the number of partitions of $n$ into odd parts (using only numbers that are odd). 
\end{theorem}

\begin{proof}
We will do this using generating functions. The generating function to partition a number into odd numbers is: 
\[
	\frac{1}{1-x} \frac{1}{1-x^3} \frac{1}{1-x^5} \frac{1}{1-x^7} \cdots
\]
The generating function for the number of ways to partition a number into distinct numbers is: 
\[
	(1+x)(1+x^2)(1+x^3)(1+x^4) \cdots 
\]
If we manipulate the second generating function as follows: 
\[
 = \frac{1-x^2}{1-x} \frac{1-x^4}{1-x^2} \frac{1-x^6}{1-x^3} \frac{1-x^8}{1-x^4} \cdots 
\] 
we will see that all the terms of the form $(1-x^k)$ where $k$ is even will cancel with the similar terms in the denominator, yielding the desired. 

\end{proof}

\section{Solving Recurrence Relations}
We will apply generating functions to find explicit formulae for recurrence relations. 
\begin{problem}
	Given the recurrence relation $a_n = a_{n-1} + 8a_{n-2} - 12a_{n-3}$, and the terms $a_0 = 2, a_1 = 3, a_2 = 19$, find an explicit formula for $a_n$ and a generating function for the sequence. 
\end{problem}
\begin{solution}
We will first construct the generating function: 
\[
	g(x) = \sum_{n=0}^\infty  a_n x^n = 2 + 3x + 19x^2 + \sum_{n=3}^\infty a_n x^n 
\]	
Applying the recurrence relation: 
\begin{align*}
	g(x) &= 2 + 3x + 19x^2 + \sum_{n=3}^\infty (a_{n-1} + 8a_{n-2} - 12a_{n-3}) x^n \\
	&= 2 + 3x + 19x^2 + x \sum_{n=3}^\infty a_{n-1}x^{n-1} + 8x^2 \sum_{n=3}^\infty a_{n-2}x^{n-2} - 12x^3\sum_{n=3}^\infty a_{n-3} x^{n-3} \\
	&= 2 + 3x + 19x^2 + x \sum_{n=2}^\infty a_{n}x^{n} + 8x^2 \sum_{n=1}^\infty a_{n}x^{n} - 12x^3\sum_{n=0}^\infty a_{n} x^{n} \\
	&= 2 + 3x + 19x^2 + x(g(x) - 2 - 3x) + 8x^2(g(x) - 2) + 12x^3 g(x) 
\end{align*}
This gives us, after some rearranging: 
\[
	g(x) = \frac{2+x}{1-x-8x+12x^3}
\]
We now have to do partial fractions to do this fully - breaking it down into \[
	g(x) = \frac{A}{1+3x} + \frac{B}{1-2x} + \frac{C}{(1-2x)^2}
\]
Using cover-up, we obtain $A = \frac{3}{5}, B = \frac{2}{5}, C = 1$. When we plug this in and expand using series: 
\begin{align*}
	g(x) &= \frac{\frac{3}{5}}{1+3x} + \frac{\frac{2}{5}}{1-2x} + \frac{1}{(1-2x)^2}\\
	&= \frac{3}{5} \sum_{n=0}^\infty (-3x)^n  + \frac{2}{5} \sum_{n=0}^\infty (2x)^n + \sum_{n=0}^\infty \snb{2}{n} (2x)^n\\
	&= \frac{3}{5} \sum_{n=0}^\infty (-3x)^n  + \frac{2}{5} \sum_{n=0}^\infty (2x)^n + \sum_{n=0}^\infty (n+1) (2x)^n
\end{align*}
We can now just read out the explicit formula: 
\[
	a_n = \frac{3}{5}(-3)^n  + \frac{2}{5}(2)^n + (n+1)(2)^n\\
\]	
	
\end{solution}


\begin{problem}
	Find a generating function and explicit formula for the sequence $a_0 = 1, a_1 = 9, a_n = 6a_{n-1} - 9a_{n-2}$. 
\end{problem}

\begin{solution}
We find the generating function first: 
\begin{align*}
	g(x) &= \sum_{n=0}^\infty a_n x^n \\
	&= 1 + 9x + \sum_{n=2}^\infty (6a_{n-1} - 9a_{n-2}) x^n \\
	&= \frac{1+3x}{1-6x+9x^2} \\
	&= \frac{-1}{1-3x} + \frac{2}{(1-3x)^2}
\end{align*}
This directly gives us the explicit formula \[
	a_n = -3^n + 2(n+1) 3^n = (2n+1)3^n 
\]
\end{solution}

\begin{problem}
	Find a generating function and explicit formula for the sequence $a_0 = 0, a_1 = 6, a_n = -3a_{n-1} + 10a_{n-2} + 3\cdot 2^n $. 
\end{problem}

\begin{solution}
Again, we find the generating function: 
\begin{align*}
	g(x) &= \sum_{n=0}^\infty a_n x^n \\
	&= 0 + 6x + \sum_{n=2}^\infty (-3a_{n-1} + 10a_{n-2} + 3\cdot 2^n) x^n \\
	&= 6x -3x g(x) + 10x^2 g(x) + \frac{3}{1-2x} - 3 - 6x \\
	&= \frac{6x}{(1-2x)^2(1+5x)}\\
	&= \frac{-\frac{12}{49}}{1-2x} + \frac{\frac{6}{7}}{(1-2x)^2} + \frac{-\frac{30}{49}}{1+5x}
\end{align*}

In the end, we have:
\[
	a_n = -\frac{12}{49} 2^n + \frac{6}{7}(n+1)2^n -\frac{30}{49} (-5)^n 
\]

\end{solution}
\end{document}
