%%%%%%% Lecture Notes Style for Concrete Mathematics %%%%%%%%%%%%%
%%%%%%% Taught by Patrick White, TJHSST %%%%%%%%%%%%%%%%%%%%%%%%%%

\documentclass[11pt,twosided]{article}
%%%%%%%%%%%%%%%%%%%%%%%%%%%%%%%%%%%%%%%%%%%%%%%%%%%%%%%
%%%% HEADER FILE for CONCRETE MATH LECTURE NOTES %%%%%%
%%%%%%%%%%%%%%%%%%%%%%%%%%%%%%%%%%%%%%%%%%%%%%%%%%%%%%%

%%%%%%%%%%%%%%%%%%%%%%%%%%%%%%%%%%%%%%%%%%%%%%%%%%%%%%%%%%%%%%%%%%%
%% This file is included at the top of every lecture notes file %%%
%% It should ONLY be changed by the instructor %%
%%%%%%%%%%%%%%%%%%%%%%%%%%%%%%%%%%%%%%%%%%%%%%%%%%%%%%%%%%%%%%%%%%%

%%%%%%%%%%%%%%%%%%%%%% package inclusions %%%%%%%%%%%%%%%%%%%%
%\usepackage[
%top=2cm,
%bottom=2cm,
%left=3cm,
%right=2cm,
%headheight=17pt, % as per the warning by fancyhdr
%includehead,includefoot,
%heightrounded, % to avoid spurious underfull messages
%]{geometry} 


\usepackage{amsmath,amssymb}
\usepackage{mathtools}
\usepackage{amsthm}

\usepackage{fancyhdr}
\pagestyle{fancy}


%%%%%%%%%%%%%%% theorem style definitions %%%%%%%%%%%%%%%%%%%%%%
\newtheorem{theorem}{Theorem}[section]
%\newtheorem{corollary}{Corollary}[theorem]
%\newtheorem{lemma}[theorem]{Lemma}
\newtheorem*{remark}{Remark}
%\newtheorem{example}{Example}[section]
\newtheorem*{definition}{Definition}


%%%%%%%%%%%%%%%% custom function definitions %%%%%%%%%%%%%%%%%%%%%%%%%
%%%%%%%%%%% as class progresses, this section may be enhanced %%%%%%%%

\def\multiset#1#2{\ensuremath{\left(\kern-.3em\left(\genfrac{}{}{0pt}{}{#1}{#2}\right)
		\kern-.3em\right)}} %%% multiset notation
\newcommand\rf[2]{{#1}^{\overline{#2}}} %%%% rising factorial \rf{x}{m}
\newcommand\ff[2]{{#1}^{\underline{#2}}} %%%%% falling factorial \ff{x}{m}


%%%%%%%
% book stuff
%%%%%%%%%

\usepackage[twoside,outer=1.5in,inner=2in,bottom=1.5in,top=1.5in,marginpar=2in]{geometry} 
\usepackage{amsmath,amsthm,amssymb,scrextend}
\usepackage{fancyhdr}
\usepackage{palatino}
\usepackage{multicol}
%\usepackage{background}
\usepackage{graphicx}
\usepackage{listliketab}

\usepackage{lipsum}
\usepackage{caption}
\usepackage{marginnote}

\reversemarginpar


\usepackage[dvipsnames]{xcolor}
\usepackage{amsthm}
\usepackage{shadethm}

\newcommand*\Note{%
	\marginnote[\textcolor{blue}{\raggedright{\LARGE ?}\ Note }]{}%
}
\newcommand*\Warn{%
	\marginnote[\textcolor{red}{\raggedright{\LARGE !}\ Warning }]{}%
}
\reversemarginpar

\newcommand{\N}{\mathbb{N}}
\newcommand{\Z}{\mathbb{Z}}
\newcommand{\I}{\mathbb{I}}
\newcommand{\R}{\mathbb{R}}
\newcommand{\Q}{\mathbb{Q}}
\renewcommand{\qed}{\hfill$\blacksquare$}
\let\newproof\proof
\renewenvironment{proof}{\begin{addmargin}[1em]{0em}\begin{newproof}}{\end{newproof}\end{addmargin}\qed}
% \newcommand{\expl}[1]{\text{\hfill[#1]}$}


\newenvironment{lemma}[2][Lemma]{\begin{trivlist}
		\item[\hskip \labelsep {\bfseries #1}\hskip \labelsep {\bfseries #2.}]}{\end{trivlist}}
\newenvironment{problem}[2][Problem]{\begin{trivlist}
		\item[\hskip \labelsep {\bfseries #1}\hskip \labelsep {\bfseries #2.}]}{\end{trivlist}}
\newenvironment{example}[2][Example]{\begin{trivlist}
		\item[\hskip \labelsep {\bfseries #1}\hskip \labelsep{\bfseries #2.}]}{\end{trivlist}}
\newenvironment{solution}{{\noindent \bfseries Solution\hskip 2ex}}{\qed}
\newenvironment{exercise}[2][Exercise]{\begin{trivlist}
		\item[\hskip \labelsep {\bfseries #1}\hskip \labelsep {\bfseries #2.}]}{\end{trivlist}}
\newenvironment{reflection}[2][Reflection]{\begin{trivlist}
		\item[\hskip \labelsep {\bfseries #1}\hskip \labelsep {\bfseries #2.}]}{\end{trivlist}}
\newenvironment{proposition}[2][Proposition]{\begin{trivlist}
		\item[\hskip \labelsep {\bfseries #1}\hskip \labelsep {\bfseries #2.}]}{\end{trivlist}}
\newenvironment{corollary}[2][Corollary]{\begin{trivlist}
		\item[\hskip \labelsep {\bfseries #1}\hskip \labelsep {\bfseries #2.}]}{\end{trivlist}}



%%%% add package inclusions here, if any
\usepackage{blubase}
%%%%% Define your custom functions and macros here, if any

%%%%%%%%%%%%%%%%%%%% Define the variables below %%%%%%%%%%%%%%%%%%%%%%%%%%%%%%
\def\titlestring{The Lecture Title}
\def\scribestring{Your Name}
\def\datestring{Day, Mon, Date Year}


%%%%%%%%%%%%%%%%%%% Page Headers -- Do Not Change %%%%%%%%%%%%%%%%%%%%%%%%%%%
\lhead{\titlestring}
\rhead{Page \thepage}
\cfoot{Concrete Math -- White -- TJHSST}
\renewcommand{\headrulewidth}{0.4pt}
\renewcommand{\footrulewidth}{0.4pt}

%%%%%%%%%%%%%%%%% Begin the document %%%%%%%%%%%%%%%%%%%%%%%%%
\begin{document}
\thispagestyle{plain}  %% no headers on this page

%%%% Do not change these lines %%%%%
\noindent
{\LARGE \textbf{\titlestring}}\\\\
%
{\Large Scribe: \scribestring}\\ \\
{\textbf{Date}: \datestring}


%%%%%%%%%%%%%%%%%%%%%%%%%% YOUR CONTENT GOES HERE %%%%%%%%%%%%%%%%%%%%%%%%%%
\noindent

\section{Partitions and Stirling Numbers}
\begin{problem}{1}
How many ways can a set of 4 elements be partitioned into 2 non-empty sets? 
\end{problem}
\begin{proof}[Solution:]
casework 
\end{proof}

Defintion: the number of ways to partition a set of $n$ elements into $k$ non-empty subsets is the \textbf{Stirling Number of the second kind} $S(n, k)$ or %binomial with curly brackets

Let's construct a few base cases: 
\begin{align*}
	&S(1, 1) = 1 \\
	&S(n, 1) = 1 \\
	&S(n, n) = 1 \\
	&S(0, 0) = 1 \text{\quad (defined as such)} \\
	&S(n, 2) = 
\end{align*}
The first four of these are fairly trivial to see, but the last two are not. 
\begin{problem}{Claim:}
$S(n, 2) = 2^{n-1} - 1$ 
\end{problem}
\begin{proof}
Consider $A = \{ 0, 1 \}^n$. This will be the characteristic function of membership in the two sets $S_0$, $S_1$. Note that $|A| = 2^n$, but $1^n$ and $0^n$ are not valid partitions (since they have an empty set). So we have $|A| = 2^n - 2$ valid mappings. However, since we are double counting because the identity of the subsets are irrelevant, we divide by 2 to yield $2^{n-1} - 1$. 
\end{proof}

\begin{problem}{Claim: }
$S(n, n-1) = \binom{n}{2}$. 
\end{problem}
\begin{proof}
Choose two elements from the $n$ that will be in the same set, and each of the rest of the elements must be in its own set. 
\end{proof}

With these base cases in mind, we can look at constructing a recursive formula for $S(n, k)$. We can start by looking at the partitions of an $n-1$-element set into $k-1$ elements - and the $n$th element must go into a new set by itself. Alternatively, we can consider $S(n-1, k)$, or the partitions of an $n-1$-element set into $k$-elements, and we must pick which one of the sets the last element must go into in $k$ ways. This gives us the recursive formula
\[
	S(n, k) = S(n-1, k-1) + kS(n-1, k)
\]
We can now begin computing values of the Stirling numbers of the second kind: 
\begin{center}
\begin{tabular}{c | c c c c c c c}
B & 1 & 2 & 3 & 4 & 5 & 6 & 7 \\ \hline 
labeled & 1 & 0 & 0 & 0 & 0 & 0 & 0 \\
unlabeled & & & \\
labeled & & & \\
unlabeled & & & \\

\end{tabular}
\end{center}
$1 - - - - - -$
$1 1 - - - - -$
$1 3 1 - - - -$
$1 7 6 1 - - -$
$1 15 25 10 1 - -$
$1 31 90 65 15 1 -$
$1 63 301 350 140 21 1$ etc. 

\section{Revisiting Balls and Urns}
\begin{center}
\begin{tabular}{c c c c c c}
B & U & no restrictions & $\leq 1$ per urn & $\geq 1$ per urn \\ \hline 
labeled & labeled & & & \\
unlabeled & labeled & & & \\
labeled & unlabeled & & & \\
unlabeled & unlabeled & & & \\

\end{tabular}
$L -> L$ 
$U -> L$, leq 1 - choose B of the U urns (U B)
$U -> L$, no restrictions - multiset problem, ((U, B))
$U -> L$, $\geq$ 1 - fill the urns with balls, and then same as before - ((U, B-U))
$L -> U$, geq 1 - literally the Stirling numbers of the second kind
$L -> U$, leq 1 - either 1 or 0 $[ B \leq U]$ only 1 if $B \leq U$. 
$L -> U$, no restrictions - sum of Stirling numbers $S(B, n)$ from $n = 1 to U$, and if we go off the triangle it is 0
$L -> L$, geq 1 - partition the elements, and then label the subsets $U! S(B, U)$. 
\end{center}
stuff 


\end{document}
