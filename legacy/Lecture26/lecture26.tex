%%%%%%% Lecture Notes Style for Concrete Mathematics %%%%%%%%%%%%%
%%%%%%% Taught by Patrick White, TJHSST %%%%%%%%%%%%%%%%%%%%%%%%%%

\documentclass[11pt,twosided]{article}
%%%%%%%%%%%%%%%%%%%%%%%%%%%%%%%%%%%%%%%%%%%%%%%%%%%%%%%
%%%% HEADER FILE for CONCRETE MATH LECTURE NOTES %%%%%%
%%%%%%%%%%%%%%%%%%%%%%%%%%%%%%%%%%%%%%%%%%%%%%%%%%%%%%%

%%%%%%%%%%%%%%%%%%%%%%%%%%%%%%%%%%%%%%%%%%%%%%%%%%%%%%%%%%%%%%%%%%%
%% This file is included at the top of every lecture notes file %%%
%% It should ONLY be changed by the instructor %%
%%%%%%%%%%%%%%%%%%%%%%%%%%%%%%%%%%%%%%%%%%%%%%%%%%%%%%%%%%%%%%%%%%%

%%%%%%%%%%%%%%%%%%%%%% package inclusions %%%%%%%%%%%%%%%%%%%%
%\usepackage[
%top=2cm,
%bottom=2cm,
%left=3cm,
%right=2cm,
%headheight=17pt, % as per the warning by fancyhdr
%includehead,includefoot,
%heightrounded, % to avoid spurious underfull messages
%]{geometry} 


\usepackage{amsmath,amssymb,amsfonts}
\usepackage{longtable}
\usepackage{mathtools}
\usepackage{amsthm}
\usepackage{enumerate}
\usepackage{fancyhdr}
\pagestyle{fancy}


%%%%%%%%%%%%%%% theorem style definitions %%%%%%%%%%%%%%%%%%%%%%
\newtheorem{theorem}{Theorem}[section]
%\newtheorem{corollary}{Corollary}[theorem]
%\newtheorem{lemma}[theorem]{Lemma}
\newtheorem*{remark}{Remark}
%\newtheorem{example}{Example}[section]
\newtheorem*{definition}{Definition}


%%%%%%%%%%%%%%%% custom function definitions %%%%%%%%%%%%%%%%%%%%%%%%%
%%%%%%%%%%% as class progresses, this section may be enhanced %%%%%%%%

\def\multiset#1#2{\ensuremath{\left(\kern-.3em\left(\genfrac{}{}{0pt}{}{#1}{#2}\right)
		\kern-.3em\right)}} %%% multiset notation
\newcommand\rf[2]{{#1}^{\overline{#2}}} %%%% rising factorial \rf{x}{m}
\newcommand\ff[2]{{#1}^{\underline{#2}}} %%%%% falling factorial \ff{x}{m}
\newcommand{\Perm}[2]{{}^{#1}\!P_{#2}} % permutation
\newcommand{\half}{\ensuremath{\frac{1}{2}}}
\newcommand{\braces}[1]{\left\{#1\right\}}
\newcommand{\set}[1]{\braces{#1}}
\newcommand{\snb}[2]{\ensuremath{\left(\kern-.3em\left(\genfrac{}{}{0pt}{}{#1}{#2}\right)\kern-.3em\right)}}
\newcommand{\fallingfactorial}[1]{^{\underline{#1}}}
\newcommand{\fallfac}[2]{{#1}^{\underline{#2}}}
\newcommand{\stirlingone}[2]{\genfrac[]{0pt}{1}{#1}{#2}}
\newcommand{\stiri}[2]{\stirlingone{#1}{#2}}
\newcommand{\dstirlingone}[2]{\genfrac[]{0pt}{0}{#1}{#2}}
\newcommand{\dstiri}[2]{\dstirlingone{#1}{#2}}
\newcommand{\stirlingtwo}[2]{\genfrac\{\}{0pt}{1}{#1}{#2}}
\newcommand{\stirii}[2]{\stirlingtwo{#1}{#2}}
\newcommand{\dstirlingtwo}[2]{\genfrac\{\}{0pt}{0}{#1}{#2}}
\newcommand{\dstirii}[2]{\dstirlingtwo{#1}{#2}}
\newcommand{\ol}[1]{\overline{#1}}
%%%%%%%
% book stuff
%%%%%%%%%

\usepackage[twoside,outer=1.5in,inner=2in,bottom=1.5in,top=1.5in,marginpar=2in]{geometry} 
\usepackage{scrextend}
\usepackage{palatino}
\usepackage{multicol}
\usepackage{float}
\usepackage[hidelinks]{hyperref}
\usepackage[nameinlink, capitalise, noabbrev]{cleveref}
%\usepackage{background}
\usepackage{graphicx}
\usepackage{listliketab}

\usepackage{lipsum}
\usepackage{caption}
\usepackage{marginnote}

\reversemarginpar


\usepackage[dvipsnames]{xcolor}
\usepackage{amsthm}
\usepackage{shadethm}

\setlength{\parindent}{0pt}

\newcommand*\Note{%
	\marginnote[\textcolor{blue}{\raggedright{\LARGE ?}\ Note }]{}%
}
\newcommand*\Warn{%
	\marginnote[\textcolor{red}{\raggedright{\LARGE !}\ Warning }]{}%
}
\reversemarginpar

\newcommand{\N}{\mathbb{N}}
\newcommand{\Z}{\mathbb{Z}}
\newcommand{\I}{\mathbb{I}}
\newcommand{\R}{\mathbb{R}}
\newcommand{\Q}{\mathbb{Q}}
\renewcommand{\qed}{\hfill$\blacksquare$}
\let\newproof\proof
\renewenvironment{proof}{\begin{addmargin}[1em]{0em}\begin{newproof}}{\end{newproof}\end{addmargin}\qed}
% \newcommand{\expl}[1]{\text{\hfill[#1]}$}


\newenvironment{lemma}[2][Lemma]{\begin{trivlist}
		\item[\hskip \labelsep {\bfseries #1}\hskip \labelsep {\bfseries #2.}]}{\end{trivlist}}
\newenvironment{problem}[2][Problem]{\begin{trivlist}
		\item[\hskip \labelsep {\bfseries #1}\hskip \labelsep {\bfseries #2.}]}{\end{trivlist}}
\newenvironment{example}[2][Example]{\begin{trivlist}
		\item[\hskip \labelsep {\bfseries #1}\hskip \labelsep{\bfseries #2.}]}{\end{trivlist}}
\newenvironment{solution}{{\noindent \bfseries Solution\hskip 2ex}}{\qed}
\newenvironment{exercise}[2][Exercise]{\begin{trivlist}
		\item[\hskip \labelsep {\bfseries #1}\hskip \labelsep {\bfseries #2.}]}{\end{trivlist}}
\newenvironment{reflection}[2][Reflection]{\begin{trivlist}
		\item[\hskip \labelsep {\bfseries #1}\hskip \labelsep {\bfseries #2.}]}{\end{trivlist}}
\newenvironment{proposition}[2][Proposition]{\begin{trivlist}
		\item[\hskip \labelsep {\bfseries #1}\hskip \labelsep {\bfseries #2.}]}{\end{trivlist}}
\newenvironment{corollary}[2][Corollary]{\begin{trivlist}
		\item[\hskip \labelsep {\bfseries #1}\hskip \labelsep {\bfseries #2.}]}{\end{trivlist}}

%%%% add package inclusions here, if any
\usepackage{blubase}
%%%%% Define your custom functions and macros here, if any

%%%%%%%%%%%%%%%%%%%% Define the variables below %%%%%%%%%%%%%%%%%%%%%%%%%%%%%%
\def\titlestring{The Lecture Title}
\def\scribestring{Your Name}
\def\datestring{Day, Mon, Date Year}


%%%%%%%%%%%%%%%%%%% Page Headers -- Do Not Change %%%%%%%%%%%%%%%%%%%%%%%%%%%
\lhead{\titlestring}
\rhead{Page \thepage}
\cfoot{Concrete Math -- White -- TJHSST}
\renewcommand{\headrulewidth}{0.4pt}
\renewcommand{\footrulewidth}{0.4pt}

%%%%%%%%%%%%%%%%% Begin the document %%%%%%%%%%%%%%%%%%%%%%%%%
\begin{document}
\thispagestyle{plain}  %% no headers on this page

%%%% Do not change these lines %\noindent
{\LARGE \textbf{\titlestring}}\\\\
%
{\Large Scribe: \scribestring}\\ \\
{\textbf{Date}: \datestring}


%%%%%%%%%%%%%%%%%%%%%%%%%% YOUR CONTENT GOES HERE %%%%%%%%%%%%%%%%%%%%%%%%%%
\noindent

\section{Burnside's Lemma}
Given a set $S$ and a permutation group $G$ acting on $S$, the number of equivalence classes of $S$ is given by 
\[
\frac{1}{|G|} \sum_{\pi \in G} |fix(\pi)|
\]
where $fix(i)$ is the set of elements fixed by $\pi$. 

For example, let's look at the square symmetry group $D_4$, and $S$ be the 2-colors of the vertices of a square. $S$ thus consists of elements that look like

% insert colorings here

Recall that $G$ is the group $\{ e, r_1, r_2, r_3, v, h, l, r\}$, where the index on $r$ represents the number of 90-degree clockwise rotations. For every permutation in $G$, we construct a table that counts the number of elements in $S$ that are left unchanged by each permutation. We start filling in as follows: 
\begin{center}
\begin{tabular}{c|c}
$\pi \in G$ & Elements fixed  \\ \hline 
$e$ & 16  \\
$r_1$ & 2 \\
$r_2$ & 4 \\
$r_3$ & 2 \\
$v$ & 4 \\
$h$ & 4 \\
$l$ & 8 \\
$r$ & 8 \\ \hline
Total & 48
\end{tabular}
\end{center}
By Burnside's, then, we have $\frac{1}{8} \cdot 48 = 6$. This is fairly easy to compute explicitly, but we can obviously also apply this to many more complicated problems.

Note that we may also interpret Burnside's as saying that the average size of $fix(\pi)$ is the number of equivalence classes, which somewhat leads to a proof of this (coming soon...)

Let's look at another example: 
\begin{problem}
Consider a bracelet with 5 beads, with each bead possibly being one of four colors (red, yellow, blue, white). Rotations of the bracelet are equivalent, but flipping the bracelet over gives a new coloring. How many colorings are there now? 
\end{problem}

\begin{solution}
We construct a similar table: 
\begin{center}
\begin{tabular}{c|c}
$\pi \in G$ & Elements fixed  \\ \hline 
$e$ & $4^5$ \\
$r$ & 4 \\
$r^2$ & 4 \\
$r^3$ & 4 \\
$r^4$ & 4 \\ \hline
Total & 1040
\end{tabular}
\end{center}
By Burnside's, we get 208 distinct colorings. 
\end{solution}

\begin{problem}
Same problem as above, but now reflections are now the same. 
\end{problem}

\begin{solution}
Add on this table: 
\begin{center}
\begin{tabular}{c|c}
$\pi \in G$ & Elements fixed  \\ \hline 
$f_1$ & 64 \\
$f_2$ & 64 \\
$f_3$ & 64 \\
$f_4$ & 64 \\
$f_5$ & 64 \\ \hline
Total & 1360
\end{tabular}
\end{center}
By Burnside's, we get 136 distinct colorings. 
\end{solution}

\begin{problem}
How many elements are in the group of rotations of a cube? How many 2-colorings are there of the faces? Of the vertices? 
\end{problem}

\begin{solution}
Notice that we have 24 total rotations - rotations can put one of the six faces on top, and choosing one of the four adjacent faces to be the "front" face gives all possible rotations. 

To break them down, we can look at what their axes are. 
\begin{itemize}
\item There is one identity rotation. The cycle structure looks like $(U)(D)(F)(B)(L)(R)$. 
\item An axis through the face of a cube gives six different rotations by 90 degrees (cycle structure $(U)(D)(FLBR)$), and three different rotations by 180 degrees (cycle structure $(U)(D)(FB)(LR)$). 
\item An axis through a pair of opposite edges gives six different rotations (cycle structure $(UB)(DF)(LR)$). 
\item An axis through a pair of opposite vertices gives 8 different rotations, 2 for each space diagonal (cycle structure $(ULF)(BRD)$).
\end{itemize}

We now analyze how many colorings are fixed by these different types of rotations for each type. We notice that we can essentially assign every cycle to one color - so this gives 64 for the identity, 8 for 90-degree face rotations, 16 for 180-degree face rotations, 8 for edge rotations, 4 for vertex rotations, which gives a total of $64 \times 1 + 8 \times 6 + 16 \times 3 + 8 \times 6 + 8 \times 4 = 240$. By Burnside's, we have 10 possible colorings. 
\end{solution}









\end{document}
