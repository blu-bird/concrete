%%%%%%% Lecture Notes Style for Concrete Mathematics %%%%%%%%%%%%%
%%%%%%% Taught by Patrick White, TJHSST %%%%%%%%%%%%%%%%%%%%%%%%%%

\documentclass[11pt,twosided]{article}
%%%%%%%%%%%%%%%%%%%%%%%%%%%%%%%%%%%%%%%%%%%%%%%%%%%%%%%
%%%% HEADER FILE for CONCRETE MATH LECTURE NOTES %%%%%%
%%%%%%%%%%%%%%%%%%%%%%%%%%%%%%%%%%%%%%%%%%%%%%%%%%%%%%%

%%%%%%%%%%%%%%%%%%%%%%%%%%%%%%%%%%%%%%%%%%%%%%%%%%%%%%%%%%%%%%%%%%%
%% This file is included at the top of every lecture notes file %%%
%% It should ONLY be changed by the instructor %%
%%%%%%%%%%%%%%%%%%%%%%%%%%%%%%%%%%%%%%%%%%%%%%%%%%%%%%%%%%%%%%%%%%%

%%%%%%%%%%%%%%%%%%%%%% package inclusions %%%%%%%%%%%%%%%%%%%%
%\usepackage[
%top=2cm,
%bottom=2cm,
%left=3cm,
%right=2cm,
%headheight=17pt, % as per the warning by fancyhdr
%includehead,includefoot,
%heightrounded, % to avoid spurious underfull messages
%]{geometry} 


\usepackage{amsmath,amssymb}
\usepackage{mathtools}
\usepackage{amsthm}

\usepackage{fancyhdr}
\pagestyle{fancy}


%%%%%%%%%%%%%%% theorem style definitions %%%%%%%%%%%%%%%%%%%%%%
\newtheorem{theorem}{Theorem}[section]
%\newtheorem{corollary}{Corollary}[theorem]
%\newtheorem{lemma}[theorem]{Lemma}
\newtheorem*{remark}{Remark}
%\newtheorem{example}{Example}[section]
\newtheorem*{definition}{Definition}


%%%%%%%%%%%%%%%% custom function definitions %%%%%%%%%%%%%%%%%%%%%%%%%
%%%%%%%%%%% as class progresses, this section may be enhanced %%%%%%%%

\def\multiset#1#2{\ensuremath{\left(\kern-.3em\left(\genfrac{}{}{0pt}{}{#1}{#2}\right)
		\kern-.3em\right)}} %%% multiset notation
\newcommand\rf[2]{{#1}^{\overline{#2}}} %%%% rising factorial \rf{x}{m}
\newcommand\ff[2]{{#1}^{\underline{#2}}} %%%%% falling factorial \ff{x}{m}


%%%%%%%
% book stuff
%%%%%%%%%

\usepackage[twoside,outer=1.5in,inner=2in,bottom=1.5in,top=1.5in,marginpar=2in]{geometry} 
\usepackage{amsmath,amsthm,amssymb,scrextend}
\usepackage{fancyhdr}
\usepackage{palatino}
\usepackage{multicol}
%\usepackage{background}
\usepackage{graphicx}
\usepackage{listliketab}

\usepackage{lipsum}
\usepackage{caption}
\usepackage{marginnote}

\reversemarginpar


\usepackage[dvipsnames]{xcolor}
\usepackage{amsthm}
\usepackage{shadethm}

\newcommand*\Note{%
	\marginnote[\textcolor{blue}{\raggedright{\LARGE ?}\ Note }]{}%
}
\newcommand*\Warn{%
	\marginnote[\textcolor{red}{\raggedright{\LARGE !}\ Warning }]{}%
}
\reversemarginpar

\newcommand{\N}{\mathbb{N}}
\newcommand{\Z}{\mathbb{Z}}
\newcommand{\I}{\mathbb{I}}
\newcommand{\R}{\mathbb{R}}
\newcommand{\Q}{\mathbb{Q}}
\renewcommand{\qed}{\hfill$\blacksquare$}
\let\newproof\proof
\renewenvironment{proof}{\begin{addmargin}[1em]{0em}\begin{newproof}}{\end{newproof}\end{addmargin}\qed}
% \newcommand{\expl}[1]{\text{\hfill[#1]}$}


\newenvironment{lemma}[2][Lemma]{\begin{trivlist}
		\item[\hskip \labelsep {\bfseries #1}\hskip \labelsep {\bfseries #2.}]}{\end{trivlist}}
\newenvironment{problem}[2][Problem]{\begin{trivlist}
		\item[\hskip \labelsep {\bfseries #1}\hskip \labelsep {\bfseries #2.}]}{\end{trivlist}}
\newenvironment{example}[2][Example]{\begin{trivlist}
		\item[\hskip \labelsep {\bfseries #1}\hskip \labelsep{\bfseries #2.}]}{\end{trivlist}}
\newenvironment{solution}{{\noindent \bfseries Solution\hskip 2ex}}{\qed}
\newenvironment{exercise}[2][Exercise]{\begin{trivlist}
		\item[\hskip \labelsep {\bfseries #1}\hskip \labelsep {\bfseries #2.}]}{\end{trivlist}}
\newenvironment{reflection}[2][Reflection]{\begin{trivlist}
		\item[\hskip \labelsep {\bfseries #1}\hskip \labelsep {\bfseries #2.}]}{\end{trivlist}}
\newenvironment{proposition}[2][Proposition]{\begin{trivlist}
		\item[\hskip \labelsep {\bfseries #1}\hskip \labelsep {\bfseries #2.}]}{\end{trivlist}}
\newenvironment{corollary}[2][Corollary]{\begin{trivlist}
		\item[\hskip \labelsep {\bfseries #1}\hskip \labelsep {\bfseries #2.}]}{\end{trivlist}}



%%%% add package inclusions here, if any
\usepackage{blubase}
%%%%% Define your custom functions and macros here, if any

%%%%%%%%%%%%%%%%%%%% Define the variables below %%%%%%%%%%%%%%%%%%%%%%%%%%%%%%
\def\titlestring{The Lecture Title}
\def\scribestring{Your Name}
\def\datestring{Day, Mon, Date Year}


%%%%%%%%%%%%%%%%%%% Page Headers -- Do Not Change %%%%%%%%%%%%%%%%%%%%%%%%%%%
\lhead{\titlestring}
\rhead{Page \thepage}
\cfoot{Concrete Math -- White -- TJHSST}
\renewcommand{\headrulewidth}{0.4pt}
\renewcommand{\footrulewidth}{0.4pt}

%%%%%%%%%%%%%%%%% Begin the document %%%%%%%%%%%%%%%%%%%%%%%%%
\begin{document}
\thispagestyle{plain}  %% no headers on this page

%%%% Do not change these lines %%%%%
\noindent
{\LARGE \textbf{\titlestring}}\\\\
%
{\Large Scribe: \scribestring}\\ \\
{\textbf{Date}: \datestring}


%%%%%%%%%%%%%%%%%%%%%%%%%% YOUR CONTENT GOES HERE %%%%%%%%%%%%%%%%%%%%%%%%%%
\noindent

\section{Manipulating Generating Functions}
Examples of turning sequences into functions: 
\begin{enumerate}
\item $<1, 1, 1, 1, \ldots > \, = 1 + x + x^2 + x^3 + \ldots = \frac{1}{1-x}$
\item $<2, 2, 2, 2, \ldots > \, = 2 + 2x + 2x^2 + 2x^3 + \ldots = \frac{2}{1-x}$
\item  $<1, -1, 1, -1, \ldots > \,= 1 - x + x^2 - x^3 + \ldots = \frac{1}{1+x}$
\item $<1, 0, 1, 0, 1, \ldots > \,= 1 + x^2 + x^4 + \ldots = \frac{1}{1-x^2}$
\item $<0, 1, 0, 1, 0, \ldots >\, = x + x^3 + x^5 + \ldots = \frac{x}{1-x^2}$
\item $<1, 2, 4, 8, 16, \ldots > \, = 1 + 2x + 4x^2 + 8x^3 + \ldots = \frac{1}{1-2x}$
\item $<1, \frac{1}{2}, \frac{1}{6}, \frac{1}{24}, \ldots> \, = 1 + \frac{1}{2} x + \frac{1}{6} x^2 + \frac{1}{24}x^3 + \ldots = \frac{e^x - 1}{x}$
\item $<1, 2, 3, 4, \ldots> = 1 + 2x + 3x^2 + 4x^3 + \ldots = \dv{}{x}(1 + x + x^2 + x^3 + \ldots) = \dv{}{x}\left(\frac{1}{1-x}\right) = \frac{1}{(1-x)^2} $
\item $<2, 6, 12, 20, 30, \ldots> = 2 + 6x + 12x^2 + 20x^3 + \ldots = \dv{}{x}(1 + 2x + 3x^2 + 4x^3 + \ldots) = \dv{}{x}\left(\frac{1}{(1-x)^2}\right) = \frac{2}{(1-x)^3} $ \\
Notice this is the sequence $\{ (k+1)(k+2) \}$.
\item $<0, 1, 4, 9, 16, \ldots> = x + 4x^2 + 9x^3 + 16x^4 + \ldots = \frac{2}{(1-x)^3} -3\frac{1}{(1-x)^2} + \frac{1}{1-x} = \frac{x+ x^2}{1-x^3}$\\
Notice that $k^2 = 1(k+2)(k+1) - 3(k+1) + 1$, so we can just take a linear combination of the sequences we know. 
\end{enumerate}

\section{Multiplying Generating Functions}
Suppose we have
\[
	A(x) = a_0 + a_1x + a_2x^2 + \ldots = \sum_{k=0}^\infty a_kx^k 
\]
\[
	B(x) = b_0 + b_1x + b_2x^2 + \ldots = \sum_{k=0}^\infty b_kx^k 
\]
and $C(x) = A(x) B(x)$. How would we find $c_n$? Notice that we can only generate terms of $x^n$ if we consider $a_0b_n, a_1b_{n-1}$, etc. Therefore, the coefficients are the Cauchy product of the coefficients: 
\[
	c_n = \sum_{k=0}^n a_k b_{n-k}
\]
so 
\[
	C(x) = stuff.
\]
Let's now consider a more specific example - if we let $B(x) = 1 + x + x^2 + x^3 + \ldots$, we will get $C(x) = A(x)B(x)$ as
\[
	C(x) = a_0x + (a_0+a_1)x + (a_0+a_1+a_2)x^2 + \ldots + \sum_{i=0}^k x^k + \ldots.
\]
Notice that leads us to conclude the following theorem: 
\begin{theorem}
If $A(x)$ is the ordinary generating function for the sequence $\{a_0, a_1, \ldots \}$, then $\frac{A(x)}{1-x}$ is the generating function for its partial sums. 
\end{theorem}

Let's see some applications of this now. Notice that we can very easily find the generating function for $<1, 2, 3, 4, \ldots>$ by directly applying this result to $A(x) = \frac{1}{1-x}$: 
\[
	\frac{1}{(1-x)^2} = 1 + 2x + 3x^2 + \ldots
\]
Notice the similarity we see with the generating function for the multiset coefficients: 
\[
	\frac{1}{(1-x)^2} = \sum_{k=0}^\infty \snb{2}{k}x^k = \sum_{k=0}^\infty \binom{2+k-1}{k}x^k = \sum_{k=0}^\infty \binom{k+1}{k}x^k = \sum_{k=0}^\infty (k+1)x^k 
\]
We can now get the generating function for the triangular numbers $D(x)$ if we take partial sums again: 
\begin{align*}
D(x) = B(x) C(x) &= \frac{1}{(1-x)^3} \\
&= \sum_{k=0}^\infty \snb{3}{k}x^k \\
&= \binom{3+k-1}{k}x^k \\
&= \sum_{k=0}^\infty \binom{k+2}{2}x^k \\
&= \sum_{k=0}^\infty \frac{(k+1)(k+2)}{2}x^k = \sum_{k=1}^\infty \frac{k(k+1)}{2}x^{k-1}
\end{align*}

Once more, to get the tetrahedral numbers 
\[
	E(x) = 1 + 4x + 10x^2 + 20x^3 + 35x^4 + \ldots
\]
\begin{align*}
E(x) = B(x) \cdot x D(x) &= \frac{x}{(1-x)^4} \\
&= \sum_{k=0}^\infty \snb{4}{k}x^{k+1} \\
&= \binom{4+k-1}{k}x^k \\
&= \sum_{k=0}^\infty \binom{k+3}{3}x^{k+1} \\
&= \sum_{k=0}^\infty \frac{(k+1)(k+2)(k+3)}{6}x^{k+1} = \sum_{k=1}^\infty \frac{k(k+1)(k+2))}{6}x^{k}
\end{align*}

From here, it seems that
\[
	\sum_{k=1}^n \rf{k}{m} = \frac{\rf{n}{m+1}}{m+1}
\]
just by considering summing the first $n$ terms of the triangular sequence and comparing it to the coefficient of the tetrahedral sequence, and then generalizing the statement to arbitrary $n$. This is a "finite calculus" version of the power rule for integration. 

Let's use this to find $\sum_{k=1}^n k^2$. Notice that we can write $k^2 = k(k+1) - k$, so we can just say
\begin{align*}
\sum_{k=1}^n k^2 &= \sum_{k=1}^n k(k+1) - \sum_{k=1}^n k\\
&= \frac{n(n+1)(n+2)}{3} - \frac{n(n+1)}{2} = \frac{n(n+1)(2n+1)}{6}
\end{align*}
which is the correct formula that we have proved before. 

Suppose we do this again, using the sum we have just shown: 
\begin{align*}
\sum_{k=1}^n k^3 &= \sum_{k=1}^n k(k+1)(k+2) - 3\sum_{k=1}^n k^2 - 2 \sum_{k=1}^n k \\
&= \frac{\rf{n}{4}}{4} - 3\frac{n(n+1)(2n+1)}{6} - 2\frac{n(n+1)}{2} \\
&= \frac{n^2(n+1)^2}{4}
\end{align*}








\end{document}
