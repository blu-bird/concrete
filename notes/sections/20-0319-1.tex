\label{20-0319-1}
%Do proof
\begin{problem} 1 Prove $\sum_{j = 0}^n \stirii{n}{k}\stiri j k = \delta(n,k)$ \end{problem}

\begin{solution}
We are given:

1. $x^n = \sum_{j = 0}^n \stirii{n}{j} x^{\underline{j}}$
% \nextline

2. $x^{\underline{j}} = \sum_{k = 0}^j \stiri j k x^k = \sum_{k = 0}^n \stiri j k x^k$ 
\newline
For this second part, we can increase the upper bound because all values for which $n > j$ are just equal to 0.
\newline 
Plugging Equation 2 into Equation 1, we get...
\newline
$x^n$ = $\sum_{j = 0}^n \stirii{n}{j} \sum_{k = 0}^n \stiri j k x^k$
\newline
Because j is independent of the inner sum, we can actually switch the order and move both sums to encapsulate the entire expression.
\newline
$x^n$ = $\sum_{j = 0}^n \sum_{k = 0}^n \stirii{n}{j} \stiri j k x^k$
\newline
The sum basically represents adding the rows of a matrix vs adding the columns of a matrix, which basically proves the statement and we get...
$\sum_{j = 0}^n \stirii{n}{k} \stiri j k = \delta(n,k)$ 
\end{solution}