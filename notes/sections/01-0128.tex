\label{01-0128}
Consider the following common variety of counting problem: 
\begin{problem}[Classic]{Problem}
Suppose you have $b$ balls and $u$ urns. How many ways are there to put the balls into the urns, if: 
\begin{center}
	\begin{minipage}[t]{0.24\linewidth}
		the balls are:
		\begin{enumerate}
		\item labeled 
		\item unlabeled 
		\end{enumerate}
	\end{minipage}
	\begin{minipage}[t]{0.24\linewidth}
		the urns are:
		\begin{enumerate}
		\item labeled 
		\item unlabeled 
		\end{enumerate}
	\end{minipage}
	\begin{minipage}[t]{0.5\linewidth}
		and there can be: 
		\begin{enumerate}
			\item \textbf{no} restrictions on the balls in the urns
			\item \textbf{at most} one ball per urn
			\item \textbf{at least} one ball per urn
			\end{enumerate}
	\end{minipage}
\end{center}
\end{problem}
This problem comes in 12 possible varieties! What happens when we have: 
\begin{itemize}
	\item 7 balls and 3 urns? 
	\item 4 balls and 8 urns? 
\end{itemize}
In time, we will be able to answer all 12 variants of each of these questions. 

\begin{remark}
	This series of twelve problems for any $b$ and $u$ is called the 
	\href{https://en.wikipedia.org/wiki/Twelvefold_way}{\textbf{twelvefold way}},
	posed by Gian-Carlo Rota. 
\end{remark}

\newpage
% \end{document}