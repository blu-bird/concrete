\subsection{Permutations and Transpositions}

Any permutation can be written as the product of transpositions, 
or the permutation of two elements (ie. (1 2), which can become (2 1) by swapping
the values). The order of the numbers transpositions does not matter. If your cyclic
permutation is made of more than 2 numbers, then permute the first number
 with every other number in the permutation. \\

For example, given (1 2 3 4), the product of transpositions will be (12)(13)(14).
Given, (425), the transpositions will be (42)(45).\\

Any permutation can be considered odd or even based on the number of
transpositions involved in the permutation. If there are an odd number of
transpositions, then the permutation is odd. If there is an even number,
then the permutation is even.

Below is a table filled in with examples using the information from
above:
\begin{center}
    \begin{tabular}{ccc}
        $\pi$ & Product of Transpositions & Type/Sign/Parity\\
        (123)(45) & (12) (13) (45)  & odd\\
(1234) & (12) (13) (14) & odd\\
(13) (425) (67)&  (13) (67) (42) (45) & even\\
(12345) & (12) (13) (14) (15)&  even
    \end{tabular}
\end{center}

