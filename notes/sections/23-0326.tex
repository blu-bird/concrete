% %%%%%%% Lecture Notes Style for Concrete Mathematics %%%%%%%%%%%%%
% %%%%%%% Taught by Patrick White, TJHSST %%%%%%%%%%%%%%%%%%%%%%%%%%

% \documentclass[11pt,twosided]{article}
% %%%%%%%%%%%%%%%%%%%%%%%%%%%%%%%%%%%%%%%%%%%%%%%%%%%%%%%
%%%% HEADER FILE for CONCRETE MATH LECTURE NOTES %%%%%%
%%%%%%%%%%%%%%%%%%%%%%%%%%%%%%%%%%%%%%%%%%%%%%%%%%%%%%%

%%%%%%%%%%%%%%%%%%%%%%%%%%%%%%%%%%%%%%%%%%%%%%%%%%%%%%%%%%%%%%%%%%%
%% This file is included at the top of every lecture notes file %%%
%% It should ONLY be changed by the instructor %%
%%%%%%%%%%%%%%%%%%%%%%%%%%%%%%%%%%%%%%%%%%%%%%%%%%%%%%%%%%%%%%%%%%%

%%%%%%%%%%%%%%%%%%%%%% package inclusions %%%%%%%%%%%%%%%%%%%%
%\usepackage[
%top=2cm,
%bottom=2cm,
%left=3cm,
%right=2cm,
%headheight=17pt, % as per the warning by fancyhdr
%includehead,includefoot,
%heightrounded, % to avoid spurious underfull messages
%]{geometry} 


\usepackage{amsmath,amssymb,amsfonts}
\usepackage{longtable}
\usepackage{mathtools}
\usepackage{amsthm}
\usepackage{enumerate}
\usepackage{fancyhdr}
\pagestyle{fancy}


%%%%%%%%%%%%%%% theorem style definitions %%%%%%%%%%%%%%%%%%%%%%
\newtheorem{theorem}{Theorem}[section]
%\newtheorem{corollary}{Corollary}[theorem]
%\newtheorem{lemma}[theorem]{Lemma}
\newtheorem*{remark}{Remark}
%\newtheorem{example}{Example}[section]
\newtheorem*{definition}{Definition}


%%%%%%%%%%%%%%%% custom function definitions %%%%%%%%%%%%%%%%%%%%%%%%%
%%%%%%%%%%% as class progresses, this section may be enhanced %%%%%%%%

\def\multiset#1#2{\ensuremath{\left(\kern-.3em\left(\genfrac{}{}{0pt}{}{#1}{#2}\right)
		\kern-.3em\right)}} %%% multiset notation
\newcommand\rf[2]{{#1}^{\overline{#2}}} %%%% rising factorial \rf{x}{m}
\newcommand\ff[2]{{#1}^{\underline{#2}}} %%%%% falling factorial \ff{x}{m}
\newcommand{\Perm}[2]{{}^{#1}\!P_{#2}} % permutation
\newcommand{\half}{\ensuremath{\frac{1}{2}}}
\newcommand{\braces}[1]{\left\{#1\right\}}
\newcommand{\set}[1]{\braces{#1}}
\newcommand{\snb}[2]{\ensuremath{\left(\kern-.3em\left(\genfrac{}{}{0pt}{}{#1}{#2}\right)\kern-.3em\right)}}
\newcommand{\fallingfactorial}[1]{^{\underline{#1}}}
\newcommand{\fallfac}[2]{{#1}^{\underline{#2}}}
\newcommand{\stirlingone}[2]{\genfrac[]{0pt}{1}{#1}{#2}}
\newcommand{\stiri}[2]{\stirlingone{#1}{#2}}
\newcommand{\dstirlingone}[2]{\genfrac[]{0pt}{0}{#1}{#2}}
\newcommand{\dstiri}[2]{\dstirlingone{#1}{#2}}
\newcommand{\stirlingtwo}[2]{\genfrac\{\}{0pt}{1}{#1}{#2}}
\newcommand{\stirii}[2]{\stirlingtwo{#1}{#2}}
\newcommand{\dstirlingtwo}[2]{\genfrac\{\}{0pt}{0}{#1}{#2}}
\newcommand{\dstirii}[2]{\dstirlingtwo{#1}{#2}}
\newcommand{\ol}[1]{\overline{#1}}
%%%%%%%
% book stuff
%%%%%%%%%

\usepackage[twoside,outer=1.5in,inner=2in,bottom=1.5in,top=1.5in,marginpar=2in]{geometry} 
\usepackage{scrextend}
\usepackage{palatino}
\usepackage{multicol}
\usepackage{float}
\usepackage[hidelinks]{hyperref}
\usepackage[nameinlink, capitalise, noabbrev]{cleveref}
%\usepackage{background}
\usepackage{graphicx}
\usepackage{listliketab}

\usepackage{lipsum}
\usepackage{caption}
\usepackage{marginnote}

\reversemarginpar


\usepackage[dvipsnames]{xcolor}
\usepackage{amsthm}
\usepackage{shadethm}

\setlength{\parindent}{0pt}

\newcommand*\Note{%
	\marginnote[\textcolor{blue}{\raggedright{\LARGE ?}\ Note }]{}%
}
\newcommand*\Warn{%
	\marginnote[\textcolor{red}{\raggedright{\LARGE !}\ Warning }]{}%
}
\reversemarginpar

\newcommand{\N}{\mathbb{N}}
\newcommand{\Z}{\mathbb{Z}}
\newcommand{\I}{\mathbb{I}}
\newcommand{\R}{\mathbb{R}}
\newcommand{\Q}{\mathbb{Q}}
\renewcommand{\qed}{\hfill$\blacksquare$}
\let\newproof\proof
\renewenvironment{proof}{\begin{addmargin}[1em]{0em}\begin{newproof}}{\end{newproof}\end{addmargin}\qed}
% \newcommand{\expl}[1]{\text{\hfill[#1]}$}


\newenvironment{lemma}[2][Lemma]{\begin{trivlist}
		\item[\hskip \labelsep {\bfseries #1}\hskip \labelsep {\bfseries #2.}]}{\end{trivlist}}
\newenvironment{problem}[2][Problem]{\begin{trivlist}
		\item[\hskip \labelsep {\bfseries #1}\hskip \labelsep {\bfseries #2.}]}{\end{trivlist}}
\newenvironment{example}[2][Example]{\begin{trivlist}
		\item[\hskip \labelsep {\bfseries #1}\hskip \labelsep{\bfseries #2.}]}{\end{trivlist}}
\newenvironment{solution}{{\noindent \bfseries Solution\hskip 2ex}}{\qed}
\newenvironment{exercise}[2][Exercise]{\begin{trivlist}
		\item[\hskip \labelsep {\bfseries #1}\hskip \labelsep {\bfseries #2.}]}{\end{trivlist}}
\newenvironment{reflection}[2][Reflection]{\begin{trivlist}
		\item[\hskip \labelsep {\bfseries #1}\hskip \labelsep {\bfseries #2.}]}{\end{trivlist}}
\newenvironment{proposition}[2][Proposition]{\begin{trivlist}
		\item[\hskip \labelsep {\bfseries #1}\hskip \labelsep {\bfseries #2.}]}{\end{trivlist}}
\newenvironment{corollary}[2][Corollary]{\begin{trivlist}
		\item[\hskip \labelsep {\bfseries #1}\hskip \labelsep {\bfseries #2.}]}{\end{trivlist}}

% %%%% add package inclusions here, if any

% %%%%% Define your custom functions and macros here, if any

% %%%%%%%%%%%%%%%%%%%% Define the variables below %%%%%%%%%%%%%%%%%%%%%%%%%%%%%%
% \def\titlestring{Indeterminate Coefficients + Inhomogeneity}
% \def\scribestring{Yoseph Mak}
% \def\datestring{26 Mar 2019}
% \def\multiset#1#2{\ensuremath{\left(\kern-.3em\left(\genfrac{}{}{0pt}{}{#1}{#2}\right)\kern-.3em\right)}}
% \setlength{\parindent}{0pt}
% \setlength{\parskip}{1em}

% %%%%%%%%%%%%%%%%%%% Page Headers -- Do Not Change %%%%%%%%%%%%%%%%%%%%%%%%%%%
% \lhead{\titlestring}
% \rhead{Page \thepage}
% \cfoot{Concrete Math -- White -- TJHSST}
% \renewcommand{\headrulewidth}{0.4pt}
% \renewcommand{\footrulewidth}{0.4pt}

% %%%%%%%%%%%%%%%%% Begin the document %%%%%%%%%%%%%%%%%%%%%%%%%
% \begin{document}
% \thispagestyle{plain}  %% no headers on this page

% %%%% Do not change these lines %%%%%
% \noindent
% {\LARGE \textbf{\titlestring}}\\\\
% %
% {\Large Scribe: \scribestring}\\ \\
% {\textbf{Date}: \datestring}


%%%%%%%%%%%%%%%%%%%%%%%%%% YOUR CONTENT GOES HERE %%%%%%%%%%%%%%%%%%%%%%%%%%
\noindent

\subsection{Recursive Problems with Inhomogeneous Terms}

\begin{problem}1
    We draw $n$ lines in the plane. How many regions can we create?
\end{problem}

\begin{solution}
    We can solve this with recursion. Obviously, $a_0=1,$ but consider the $a_{n-1}$ case. If we add another line, consider every place it can intersect. There are $n-1$ lines to intersect with, and if we have $n-1$ arbitrary lines, putting a line through them adds $n$ regions to the total (imagine that the lines are all parallel since their intersections have nothing to do with the new line). Thus, we write $a_n=a_{n-1}+n$ with $a_0=1,$ giving us $$a_n=a_0+\sum_{k=1}^{\infty}k=1+\frac{k (k+1)}{2}=\boxed{\multiset{3}{n-1}+1}.$$
\end{solution}

\begin{problem}2
    How many sequences of length $n$ using $0-3$ contain an even number of zeroes?
\end{problem}

\begin{solution}
    Since we're using sequences, we need to consider even and odd zeros. Construct two sequences $O_n$ and $E_n$ where $O_0=0$ and $E_0=1.$ Then, we have $E_{n+1}=3E_n+O_n.$ We also have $O_n+E_n=4^n$ since every sequence consists of an odd or even number of zeroes and there are four choices per digit in the sequence.

    Thus, we write $E_{n+1}=3E_n+(4^n-E_n)=2E_n+4^n.$ If we instead write this as $E_n=2E_{n-1}+4^{n-1},$ we can solve for our generating function as follows: $$(1-2x)g(x)=1+\frac{x}{1-4x}=\frac{1-3x}{1-4x}$$
$$\rightarrow g(x)=\frac{1-3x}{(1-2x)(1-4x)}=\frac{1/2}{1-2x}+\frac{1/2}{1-4x}$$
so we have $E_n=\boxed{\frac{2^n+4^n}{2}}=2^{2n-1}+2^{n-1}.$ 
\end{solution}


\subsection{Extra Terms (Inhomogeneous Terms)}

\textbf{Definition.} \textit{Say we have $\sum_{i=0}^{n}c_ia_i=f$ for some function $f.$ Then $f$ is the \textbf{forcing function} or \textbf{inhomogeneity} of the recursion.}

\textbf{Definition.} \textit{If we have $\sum_{i=0}^{n}c_ia_i=f$ as before, then the solution to the equation $\sum_{i=0}^{n}c_ia_i=0$ is called the \textbf{homogeneous} solution to the equation while the solution to the original equation is called the \textbf{particular} solution. Note that neither uses the initial conditions.}

We'll need both solutions to solve recurrences in general since the real solution will be a linear combination of what we have because initial conditions are annoying. Specifically, our particular solution cannot change, but we can always add a multiple of the homogeneous solution to get our final answer based on the initial equations. If $p(n)$ is our particular solution and $h(n)$ is our homogeneous, the solution will always be of the form $p(n)+c\cdot h(n)$ for some constant c.

\begin{problem}3
    Given $a_n+2a_{n-1}=n+3, a_0=3,$ solve for $a_n.$
\end{problem}

\begin{solution}
    The homogeneous solution is $(-2)^n$ (or a nonzero constant multiple of that) because our characteristic equation is $x^2+2x=0$ and $0$ clearly doesn't work.

    Let's look for the particular solution now. Consider that the inhomogeneity is linear. We agree that $a_n=Bn+D$ by the \textbf{Method of Undetermined Coefficients,} which is essentially a way of "guessing" what the solution will look like. We have $$a_n+2a_{n-1}=Bn+D+2(B(n-1)+D)=3nB+3D-2B=n+3,$$ so we have $B=\frac{1}{3}$ and $3D-2B=3D-\frac{2}{3}=3\rightarrow D=\frac{3+2/3}{3}=\frac{11}{9}.$ Thus, our particular solution is $a_n=\frac{1}{3}n+\frac{11}{9}$.
    
    Finally, let's combine the two. At $n=0,$ our particular says $\frac{11}{9},$ but our homogeneous says $1.$ Thus, we have $\frac{11}{9}+c\cdot 1=3$ since the solution is of the form $p(n)+c\cdot h(n).$ Thus, $c=\frac{16}{9}.$ This gives us the solution
    $$a_n=\boxed{\frac{1}{3}n+\frac{11}{9}+\frac{16}{9}(-2)^n}.$$
\end{solution}

Let's now look at the original two problems again.

\begin{problem}{2 (revisit)}
    Let $a_n=4^{n-1}+2a_{n-1}.$ Solve for $a_n.$
\end{problem}

\begin{solution}
    The homogeneous is clearly $2^n,$ but the particular is less clear-cut. Writing $a_n=c4^n,$ we have $c(4^n-2\cdot 4^{n-1})=4^{n-1}$ which gives $c=\frac{1}{2}.$ If we then write $a_n=\frac{1}{2}4^n+d\cdot 2^n$ and use $a_1=3,$ we have $a_1=2+d\cdot 2=3\rightarrow d=\frac{1}{2},$ so we get $a_n=\frac{4^n+2^n}{2}$ as before.
\end{solution}

\begin{problem}{1 (revisit)}
    Let $a_n=a_{n-1}+n.$ Solve for $a_n.$
\end{problem}

\begin{solution}
    The homogeneous for $a_n=a_{n-1}+n$ is literally $1$. However, to find the particular here, the Method of Undetermined Coefficients says that we must use a quadratic since a linear term will cancel if we set $a_n-a_{n-1}=n.$ (In more specific terms, if we write $a_n=Bn+D,$ we find that $a_n-a_{n-1}=Bn+D-B(n-1)+D=B,$ leaving us with no information.)

    Thus, write $a_n=An^2+Bn+C.$ Using $a_0=1, a_1=2, a_2=4,$ we have $C=1, A+B+C=2,$ and $4A+2B+C=4.$ This gives $A=\frac{1}{2}, B=\frac{1}{2}, C=1.$ Thus, our final solution is $a_n=\frac{1}{2}n^2+\frac{1}{2}n+1+d,$ where $d$ is the coefficient on the homogeneous solution. Taking $a_0=1$ again, we get $d=0,$ so the answer is just the particular: $a_n=\boxed{\frac{1}{2}n^2+\frac{1}{2}n+1}$ which equals our original answer.
    
\end{solution}
