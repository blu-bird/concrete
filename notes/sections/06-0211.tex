\label{06-0211}
\subsection{Partitions and Stirling Numbers}
\begin{problem}{1}
How many ways can a set of 4 elements be partitioned into 2 non-empty sets? 
\end{problem}
\begin{proof}[Solution:]
We perform casework. Here are the ways: 
\[
    1 \mid 2 \, 3 \, 4 \quad 
    2 \mid 1 \, 3 \, 4 \quad 
    3 \mid 1 \, 2 \, 4 \quad 
    4 \mid 1\, 2 \, 3 \quad 
\]
\[
    1 \, 2 \mid 3 \, 4 \quad 
    1 \, 3 \mid 2 \, 4 \quad 
    1 \, 4 \mid 2 \, 3
\]
\end{proof}

\begin{definition}
    The number of ways to partition a set of $n$ elements into $k$ non-empty subsets is the \textbf{Stirling Number of the second kind} $S(n, k)$ 
    or $\dstirii{n}{k}$.
\end{definition}

Let's construct a few base cases: 
\begin{align*}
	&S(1, 1) = 1 \\
	&S(n, 1) = 1 \\
	&S(n, n) = 1 \\
	&S(0, 0) = 1 \text{\quad (defined as such)} \\
	&S(n, 2) = 2^{n-1} - 1 \\
    &S(n, n-1) = \binom n 2
\end{align*}
The first four of these are fairly trivial to see, but the last two are not. 
\begin{proposition}{A}
$S(n, 2) = 2^{n-1} - 1$ 
\end{proposition}
\begin{proof}
Consider $A = \{ 0, 1 \}^n$. This will be the characteristic function of membership in the two sets $S_0$, $S_1$. Note that $|A| = 2^n$, but $1^n$ and $0^n$ are not valid partitions (since they have an empty set). So we have $|A| = 2^n - 2$ valid mappings. However, since we are double counting because the identity of the subsets are irrelevant, we divide by 2 to yield $2^{n-1} - 1$. 
\end{proof}
\begin{proposition}{B}
$S(n, n-1) = \binom{n}{2}$. 
\end{proposition}
\begin{proof}
Choose two elements from the $n$ that will be in the same set, and each of the rest of the elements must be in its own set. 
\end{proof}

With these base cases in mind, we can look at constructing a recursive formula for $S(n, k)$. We can start by looking at the partitions of an $n-1$-element set into $k-1$ elements - and the $n$th element must go into a new set by itself. Alternatively, we can consider $S(n-1, k)$, or the partitions of an $n-1$-element set into $k$-elements, and we must pick which one of the sets the last element must go into in $k$ ways. This gives us the recursive formula
\[
	S(n, k) = S(n-1, k-1) + kS(n-1, k)
\]
We can now begin computing values of the Stirling numbers of the second kind: 
\begin{center}
\begin{tabular}{c | c c c c c c c}
$n \backslash k$ & 1 & 2 & 3 & 4 & 5 & 6 \\ \hline 
1 & 1 & 0 & 0 & 0 & 0 & 0 \\
2 & 1 & 1 & 0 & 0 & 0 & 0 \\
3 & 1 & 3 & 1 & 0 & 0 & 0\\
4 & 1 & 7 & 6 & 1 & 0 & 0\\
5 & 1 & 15& 25& 10& 1 & 0\\
6 & 1 & 31& 90& 65& 15& 1\\
\end{tabular}
\end{center}

\subsection{Revisiting Balls and Urns}
We actually now have most of the information we need to understand the 
twelvefold way! If we have $b$ balls and $u$ urns, here are the number of ways 
to put the balls in the urns (subject to either no restrictions, or putting 
at least 1 in each/at most 1 in each):
\begin{center}
\begin{tabular}{c c c c c c}
B & U & $\varnothing$ & $\leq 1$ per & $\geq 1$ per \\ \hline 
labeled & labeled & $u^b$ & $u\fallingfactorial{b}$ & $u! \stirii{b}{u}$ \\
unlabeled & labeled & $\snb u b$ & $\binom u b$ & $\snb u {b-u}$ \\
labeled & unlabeled & $\sum_{n=1}^u \stirii b n$ & $[b \leq u]$ & $\stirii b u$ \\
unlabeled & unlabeled & ? & $[b \leq u]$ & ?  \\
\end{tabular}
\end{center}

A few notes on some of these cases: 
\begin{itemize}
    \item unlabeled balls into labeled urns, $\geq 1$ per: fill the urns with at 
    least one ball each first, then do the multisets. This also has the name of 
    a ``composition of $b$ into $u$ parts.''
    \item $[P]$ means 1 if $P$ is true, and 0 otherwise. This appears for 
    labeled balls unto unlabeled urns, $\leq 1$ per urn -- either we can 
    do it, or there are no ways to do it. Same with unlabeled balls into unlabeled urns. 
\end{itemize}
