\label{03-0131}
\begin{problem}{4}
    How many divisors does the number 496 have? What about the number 360?
\end{problem}

\begin{solution}
    Consider the prime factorizations of these numbers: 
    \[
        496 = 31 \cdot 2^4  \qquad 360 = 5 \cdot 3^2 \cdot 2^3
    \]
    A divisor of these numbers is created by choosing a possible number of 
    prime factors that divide the original number. For example, a divisor
    of 496 can have either 0 or 1 factors of 31, and anywhere from 0 to 4 
    factors of 2. Then the number of divisors of 496 is $(1 + 1) (4 + 1) = 10$,
    and similarly, the number of divisors of 360 is $(1 + 1)(2 + 1)(3 + 1) = 24$, 
    following from the Multiplication Theorem.
\end{solution}

\begin{problem}{5}
    What are the sum of the divisors of 496 and 360? The ``sum of all divisors''
    function is denoted $\sigma(n)$. 
\end{problem}
\begin{solution}
    Using a modification of the Multiplication Theorem, note that the 
    multiplication operation encodes the idea of ``all possible ways of 
    combining two things.'' As such, if we consider the sum of all of the 
    different prime powers that can appear in any divisor and multiply them
    together, we will create a term with every divisor, added together. For
    instance, the sum of the divisors of 496 is:
    \[
        \sigma(496) = (1 + 31)(1 + 2 + 4  + 8 + 16) = 32 \dot 31 = 992
    \]
    and similarly 
    \[
        \sigma(360) = (1 + 5)(1 + 3 + 9)(1 + 2 + 4 + 8) = 6 \cdot 13 \cdot 15 = 1170.
    \]
\end{solution}

\begin{remark}
    If we consider the sum of all \textit{proper} divisors of a positive 
    integer $n$ ($\sigma(n) - n$), we can classify integers into three categories
    based on how $\sigma(n) - n$ compares to $n$: 
    \begin{itemize}
        \item If $\sigma(n) - n < n$, then $n$ is \textit{deficient}. 
        \item If $\sigma(n) - n > n$, then $n$ is \textit{abundant}.
        \item If $\sigma(n) - n = n$, then $n$ is \textit{perfect}.
    \end{itemize}
    In the example above, note that 360 is abundant and 496 is perfect. 

    Finding perfect numbers is actually an incredibly difficult problem in 
    number theory -- in fact, nobody even knows if there exists perfect
    numbers that are odd! In the even case, though, Euler showed that 
    perfect numbers are of the form $2^{p-1}(2^p - 1)$ if $2^p - 1$ is prime 
    (which requires $p$ itself to be prime also). Primes of the form $2^p - 1$
    are called \textbf{Mersenne primes}, and it is not even known if there
    are infinitely many of these primes! Only 51 Mersenne primes are known 
    to exist as of 2023, so we only know of 51 perfect numbers. 

\end{remark}