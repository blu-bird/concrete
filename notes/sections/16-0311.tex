%%%%%%% Lecture Notes Style for Concrete Mathematics %%%%%%%%%%%%%
%%%%%%% Taught by Patrick White, TJHSST %%%%%%%%%%%%%%%%%%%%%%%%%%

\documentclass[11pt,twosided]{article}
%%%%%%%%%%%%%%%%%%%%%%%%%%%%%%%%%%%%%%%%%%%%%%%%%%%%%%%
%%%% HEADER FILE for CONCRETE MATH LECTURE NOTES %%%%%%
%%%%%%%%%%%%%%%%%%%%%%%%%%%%%%%%%%%%%%%%%%%%%%%%%%%%%%%

%%%%%%%%%%%%%%%%%%%%%%%%%%%%%%%%%%%%%%%%%%%%%%%%%%%%%%%%%%%%%%%%%%%
%% This file is included at the top of every lecture notes file %%%
%% It should ONLY be changed by the instructor %%
%%%%%%%%%%%%%%%%%%%%%%%%%%%%%%%%%%%%%%%%%%%%%%%%%%%%%%%%%%%%%%%%%%%

%%%%%%%%%%%%%%%%%%%%%% package inclusions %%%%%%%%%%%%%%%%%%%%
%\usepackage[
%top=2cm,
%bottom=2cm,
%left=3cm,
%right=2cm,
%headheight=17pt, % as per the warning by fancyhdr
%includehead,includefoot,
%heightrounded, % to avoid spurious underfull messages
%]{geometry} 


\usepackage{amsmath,amssymb,amsfonts}
\usepackage{longtable}
\usepackage{mathtools}
\usepackage{amsthm}
\usepackage{enumerate}
\usepackage{fancyhdr}
\pagestyle{fancy}


%%%%%%%%%%%%%%% theorem style definitions %%%%%%%%%%%%%%%%%%%%%%
\newtheorem{theorem}{Theorem}[section]
%\newtheorem{corollary}{Corollary}[theorem]
%\newtheorem{lemma}[theorem]{Lemma}
\newtheorem*{remark}{Remark}
%\newtheorem{example}{Example}[section]
\newtheorem*{definition}{Definition}


%%%%%%%%%%%%%%%% custom function definitions %%%%%%%%%%%%%%%%%%%%%%%%%
%%%%%%%%%%% as class progresses, this section may be enhanced %%%%%%%%

\def\multiset#1#2{\ensuremath{\left(\kern-.3em\left(\genfrac{}{}{0pt}{}{#1}{#2}\right)
		\kern-.3em\right)}} %%% multiset notation
\newcommand\rf[2]{{#1}^{\overline{#2}}} %%%% rising factorial \rf{x}{m}
\newcommand\ff[2]{{#1}^{\underline{#2}}} %%%%% falling factorial \ff{x}{m}
\newcommand{\Perm}[2]{{}^{#1}\!P_{#2}} % permutation
\newcommand{\half}{\ensuremath{\frac{1}{2}}}
\newcommand{\braces}[1]{\left\{#1\right\}}
\newcommand{\set}[1]{\braces{#1}}
\newcommand{\snb}[2]{\ensuremath{\left(\kern-.3em\left(\genfrac{}{}{0pt}{}{#1}{#2}\right)\kern-.3em\right)}}
\newcommand{\fallingfactorial}[1]{^{\underline{#1}}}
\newcommand{\fallfac}[2]{{#1}^{\underline{#2}}}
\newcommand{\stirlingone}[2]{\genfrac[]{0pt}{1}{#1}{#2}}
\newcommand{\stiri}[2]{\stirlingone{#1}{#2}}
\newcommand{\dstirlingone}[2]{\genfrac[]{0pt}{0}{#1}{#2}}
\newcommand{\dstiri}[2]{\dstirlingone{#1}{#2}}
\newcommand{\stirlingtwo}[2]{\genfrac\{\}{0pt}{1}{#1}{#2}}
\newcommand{\stirii}[2]{\stirlingtwo{#1}{#2}}
\newcommand{\dstirlingtwo}[2]{\genfrac\{\}{0pt}{0}{#1}{#2}}
\newcommand{\dstirii}[2]{\dstirlingtwo{#1}{#2}}
\newcommand{\ol}[1]{\overline{#1}}
%%%%%%%
% book stuff
%%%%%%%%%

\usepackage[twoside,outer=1.5in,inner=2in,bottom=1.5in,top=1.5in,marginpar=2in]{geometry} 
\usepackage{scrextend}
\usepackage{palatino}
\usepackage{multicol}
\usepackage{float}
\usepackage[hidelinks]{hyperref}
\usepackage[nameinlink, capitalise, noabbrev]{cleveref}
%\usepackage{background}
\usepackage{graphicx}
\usepackage{listliketab}

\usepackage{lipsum}
\usepackage{caption}
\usepackage{marginnote}

\reversemarginpar


\usepackage[dvipsnames]{xcolor}
\usepackage{amsthm}
\usepackage{shadethm}

\setlength{\parindent}{0pt}

\newcommand*\Note{%
	\marginnote[\textcolor{blue}{\raggedright{\LARGE ?}\ Note }]{}%
}
\newcommand*\Warn{%
	\marginnote[\textcolor{red}{\raggedright{\LARGE !}\ Warning }]{}%
}
\reversemarginpar

\newcommand{\N}{\mathbb{N}}
\newcommand{\Z}{\mathbb{Z}}
\newcommand{\I}{\mathbb{I}}
\newcommand{\R}{\mathbb{R}}
\newcommand{\Q}{\mathbb{Q}}
\renewcommand{\qed}{\hfill$\blacksquare$}
\let\newproof\proof
\renewenvironment{proof}{\begin{addmargin}[1em]{0em}\begin{newproof}}{\end{newproof}\end{addmargin}\qed}
% \newcommand{\expl}[1]{\text{\hfill[#1]}$}


\newenvironment{lemma}[2][Lemma]{\begin{trivlist}
		\item[\hskip \labelsep {\bfseries #1}\hskip \labelsep {\bfseries #2.}]}{\end{trivlist}}
\newenvironment{problem}[2][Problem]{\begin{trivlist}
		\item[\hskip \labelsep {\bfseries #1}\hskip \labelsep {\bfseries #2.}]}{\end{trivlist}}
\newenvironment{example}[2][Example]{\begin{trivlist}
		\item[\hskip \labelsep {\bfseries #1}\hskip \labelsep{\bfseries #2.}]}{\end{trivlist}}
\newenvironment{solution}{{\noindent \bfseries Solution\hskip 2ex}}{\qed}
\newenvironment{exercise}[2][Exercise]{\begin{trivlist}
		\item[\hskip \labelsep {\bfseries #1}\hskip \labelsep {\bfseries #2.}]}{\end{trivlist}}
\newenvironment{reflection}[2][Reflection]{\begin{trivlist}
		\item[\hskip \labelsep {\bfseries #1}\hskip \labelsep {\bfseries #2.}]}{\end{trivlist}}
\newenvironment{proposition}[2][Proposition]{\begin{trivlist}
		\item[\hskip \labelsep {\bfseries #1}\hskip \labelsep {\bfseries #2.}]}{\end{trivlist}}
\newenvironment{corollary}[2][Corollary]{\begin{trivlist}
		\item[\hskip \labelsep {\bfseries #1}\hskip \labelsep {\bfseries #2.}]}{\end{trivlist}}

%%%% add package inclusions here, if any
\usepackage{longtable}
\usepackage{mathtools}

%%%%% Define your custom functions and macros here, if any

%%%%%%%%%%%%%%%%%%%% Define the variables below %%%%%%%%%%%%%%%%%%%%%%%%%%%%%%
\def\titlestring{Generating Functions}
\def\scribestring{Frank Ding \\ \\ Annotated by Gwendolyn Jacobson and Caroline Jin}
\def\datestring{Tue, March 13, 2018 \\ Annotated following Mon, March 11, 2019 lecture.}


%%%%%%%%%%%%%%%%%%% Page Headers -- Do Not Change %%%%%%%%%%%%%%%%%%%%%%%%%%%
\lhead{\titlestring}
\rhead{Page \thepage}
\cfoot{Concrete Math -- White -- TJHSST}
\renewcommand{\headrulewidth}{0.4pt}
\renewcommand{\footrulewidth}{0.4pt}

%%%%%%%%%%%%%%%%% Begin the document %%%%%%%%%%%%%%%%%%%%%%%%%
\begin{document}
\thispagestyle{plain}  %% no headers on this page

%%%% Do not change these lines %%%%%
\noindent
{\LARGE \textbf{\titlestring}}\\\\
%
{\Large Scribe: \scribestring}\\ \\
{\textbf{Date}: \datestring}


%%%%%%%%%%%%%%%%%%%%%%%%%% YOUR CONTENT GOES HERE %%%%%%%%%%%%%%%%%%%%%%%%%%
\noindent

\section{Fundamental Generating Functions}
Recall the generating function for a sequence $a_0, a_1, a_2, \dots, a_n, \dots$
\[
f(x) = \sum_{i=0}^{\infty} a_i x^i
\]
The goal is to find a formula for an arbitrary $a_i$. \\
\begin{longtable}{p{0.2in}p{2.2in}p{3.6in}}
    a.  &   $a_0, a_1, \dots = 1, 1, 1, \dots$ \begin{multline*} \text{Maclaurin series for $\frac{1}{1-x}.$} \end{multline*} &
        $\begin{aligned}[t]
            f(x)    &=  1 + x + x^2 + \dots \\ \\
            &= \sum x^i \\ \\
            x \cdot f(x)  &=  x + x^2 + x^3 + \dots \\
            f(x) \cdot (1-x) &= 1 \\
            f(x)    &=  \frac{1}{1-x}
        \end{aligned}$ \\~\\
    b.  &   $a_0, a_1, \dots = 2, 2, 2, \dots$ \begin{multline*} \text{...OR recognize $\frac{2}{1-x}$.}\end{multline*}&
        $\begin{aligned}[t]
            f(x)    &=  2 + 2x + 2x^2 + \dots \\
            x \cdot f(x)  &=  2x + 2x^2 + 2x^3 + \dots \\
            f(x) \cdot (1-x) &= 2 \\
            f(x)    &=  \frac{2}{1-x}
        \end{aligned}$ \\~\\
        
    c.  &   $a_0, a_1, \dots = 1, -1, 1, -1, \dots$ \begin{multline*} \text{i.e. $\frac{1}{1-(-x)}$} \end{multline*}&
        $\begin{aligned}[t]
            f(x)    &=  1 - x + x^2 - x^3 + \dots \\
                    &= \sum x^i (-1)^{i+1} \\
            x \cdot f(x) &= x - x^2 + x^3 - \dots \\
            f(x) + x \cdot f(x) &= 1 \\
            f(x)(1+x) &= 1 \\
            f(x) &=  \frac{1}{1+x}
        \end{aligned}$ \\~\\
    d.  &   $a_0, a_1, \dots = 1, 0, 1, 0, \dots$ &
        $\begin{aligned}[t]
            f(x)    &=  1 + x^2 + x^4 + \dots \\
            &= \sum x^{2i} \\
            x^2 \cdot f(x) &= x^2 + x^4 + x^6 + \dots \\
            f(x) \cdot (1 - x^2) &= 1 \\
            f(x)    &=  \frac{1}{1-x^2}\\
        \end{aligned}$ \\~\\
        & OR you can add parts a. and c. and then divide by 2\\ \\
    e.  &   $a_0, a_1, \dots = 0, 1, 0, 1, \dots$ &
        $\begin{aligned}[t]
            f(x)    &=  x + x^3 + x^5 + \dots \\
            &= \sum x^{2i+1} \\
            x^2 \cdot f(x) &= x^3 + x^5 + x^7 + \dots \\
            f(x) - x^2 \cdot f(x) &= x \\
            f(x) &= \frac{x}{1-x^2}
        \end{aligned}$ \\~\\
        & OR you can subtract part a. by part c. and then divide by 2\\ \\
    f.  &   $a_0, a_1, \dots = 1, 2, 4, 8, 16, \dots$ &
        $\begin{aligned}[t]
            f(x)    &=  1 + 2x + 4x^2 + 8x^3 + \dots \\
            &= \sum (2^i)(x^i) \\
            2x \cdot f(x) &=  2x + 4x^2 + 8x^3 + \dots \\
    f(x) \cdot (1-2x) &= 1 \\
            f(x)    &=  \frac{1}{1-2x}
        \end{aligned}$ \\~\\
        & OR recognize similarity to $\frac{1}{1-x}$ Maclaurin series, and simply substitute $2x$ for $x$.\\ \\
    g.  &   $a_0, a_1, \dots = 1, \frac{1}{2}, \frac{1}{6}, \frac{1}{24}, \dots$ \begin{multline*} \text{Similar to $e^x$ Maclaurin series.} \\ \\\text{Account for shift in constant} \\ \text{and $x$.} \end{multline*} &
        $\begin{aligned}[t]
            e^x &= 1 + x + \frac{x^2}{2} + \frac {x^3}{6} + \frac{x^4}{24} + \dots \\
            &= \sum \frac{x^n}{n!} \\
            e^x - 1 &= x + \frac{x^2}{2} + \frac {x^3}{6} + \frac{x^4}{24} + \dots \\
            \frac{e^x - 1}{x} &= 1 + \frac{x}{2} + \frac {x^2}{6} + \frac{x^3}{24} + \dots \\
        \end{aligned}$ \\~\\
    h.  &   $a_0, a_1, \dots = 1, 2, 3, 4, \dots$ \begin{multline*} \text{Focus on } \textit{powers} \text{ of } x. \\ \rightarrow \text{Use first derivative!} \end{multline*}&
        $\begin{aligned}[t]
            f(x)    &=  1 + 2x + 3x^2 + \dots \\
            \frac{1}{1-x} &= 1 + x + x^2 + \dots \\
            \frac{d}{dx} \left( \frac{1}{1-x} \right) &= 1 + 2x + 3x^2 + \dots \\
            &= f(x) = \frac{1}{(1-x)^2}
        \end{aligned}$ \\~\\
    i.  &   $a_0, a_1, \dots = 2, 6, 12, 20, 30, \dots$ \begin{multline*} n(n-1) \\ \rightarrow \text{second derivative!} \end{multline*} &
        $\begin{aligned}[t]
            \frac{1}{1-x} &= \sum x^k \\
            \frac{1}{(1-x)^2} &= \sum_{k=0}^{\infty} k x^{k-1} = \sum_{k=0}^{\infty} (k+1)x^k \\
                &= 1 + 2x + 3x^2 + \dots \\
            \frac{2}{(1-x)^3} &= \sum_{k=0}^{\infty} k(k-1)x^{k-2} = \sum_{k=0}^{\infty} (k+2)(k+1)x^k \\
                &= 2 + 6x + 12x^2 + \dots
        \end{aligned}$ \\~\\
    j.  &   $a_0, a_1, \dots = 0, 1, 4, 9, 16, \dots$ &
        $\begin{aligned}[t]
            k^2     &=  \underline{1}(k+2)(k+1) + \underline{-3}(k+1) + \underline{1} \\
            \sum_{k=0}^{\infty} k^2 x^k &= \frac{2}{(1-x)^3} - \frac{3}{(1-x)^2} + \frac{1}{1-x} \\
                &= \frac{x + x^2}{(1-x)^3}
        \end{aligned}$\\
        &Note: the values $(k+2)(k+1)$ refer to the $\big(\binom{2}{k}\big)$ and so on
\end{longtable}

\par\noindent\rule{\textwidth}{0.4pt}

\paragraph{Example.}
	How many ways can you fill a bag with $n$ fruits given the following conditions:
	\begin{itemize}
	    \item the number of apples is even
	    \item the number of bananas is a multiple of 5
	    \item the number of oranges is $\leq 4$
	    \item the number of pears is $\leq 1$
	\end{itemize}

\paragraph{Solution.} Use a generating function:
\begin{align*}
    f(x)    &=  
    \underbrace{(1 + x^2 + x^4 + \dots)}_\text{apples} + 
    \underbrace{(1 + x^5 + x^{10} + \dots)}_\text{bananas} + 
    \underbrace{(1 + x + x^2 + x^3 + x^4)}_\text{oranges} + 
    \underbrace{(1 + x)}_\text{pears}\\
            &= \left(\frac{1}{1-x^2}\right) \left(\frac{1}{1-x^5}\right)
            \left(\frac{1-x^5}{1-x}\right)
            \left(\frac{1+x}{1}\right) = \frac{1}{(1-x)^2} \\
            &= \text{fundamental generating function (h)}
            =  \sum_{k=0}^{\infty} (k+1)x^k
\end{align*}
$\therefore \text{There are \boxed{$n+1$} ways to pack $n$ fruits.}$

\end{document}
