% partitions

\section{Interlude: Partition Numbers}

\begin{definition}
    Define the $n$th partition number, $p(n)$, be the number of ways to write $n$ as a
sum of positive integers.
\end{definition}

As an example,
\begin{align*}
    5 &= 5 \\
&= 4 + 1 \\
&= 3 + 2 \\
&= 3 + 1 + 1\\
&= 2 + 2 + 1\\
&= 2 + 1 + 1 + 1\\
&= 1 + 1 + 1 + 1 + 1\\
p(5) &= 7
\end{align*}
\begin{definition}
    Let $p_k(n)$ count the number of ways to partition $n$ into exactly $k$ parts. 
\end{definition}
\begin{theorem}
    \[
        p(n) = \frac 1{\pi \sqrt 2} \sum_{k=1}^\infty A_k(n) \frac d{dn} 
        \left(\frac 1 {\sqrt{n - 1/24}} \sinh \left[\frac \pi k \sqrt{\frac 23 \left( n - \frac 1 {24} \right)} \right] \right) 
    \]
    where 
    \[
        A_k = \sum_{\substack{\gcd(m, k) = 1 \\ m < k}} \exp \left( i \pi \left( S(m, k) - \frac{2mn}k \right) \right)
    \]
\end{theorem}
\begin{proof}
    Trivial and left as an exercise to the reader.
\end{proof}

\begin{theorem}
    \[
        p_k(n) \geq \frac 1{k!} \binom{n-1}{k-1}
    \]
\end{theorem}
\begin{proof}
    Let's look for solutions to 
    \[ x_1 + x_2 + \dots + x_k = n \]
    By stars and bars, we there are $\snb k {n-k} = \binom{n-1}{k-1}$ solutions. 
On the other hand, each partition of $k$ elements is associated with at most $k!$
compositions. Thus, 
\[ p_k(n) \geq \frac 1{k!} \binom{n-1}{k-1}\]
\end{proof}

A \textit{Ferrers graph} or \textit{Young tableau} is a way of drawing partitions graphically.
Each row in the tableau contains as many dots as the the magnitude of the corresponding part in the partition. 
Let the \textit{transpose} of a partition $\lambda$ into $\lambda^T$ be the tableau that
you get by swapping rows with columns. A partition $\lambda$ is \textit{self-conjugate} if $\lambda = \lambda^T$
\begin{theorem}
    The number of self-conjugate partitions of $n$ is equal to the number of partitions of $n$ with distinct odd-sized parts.
\end{theorem}
\begin{proof}
    Any self-conjugate partition has symmetry about the main diagonal. If
we select the top row and left-most column together to be one part, this gives
an odd sized part. The remainder of the tableaux is still self-conjugate and we
recursively get more odd-sized parts. Notice to reverse the proof, we need to
have the odd-sized parts be distinct sizes, or else they won't stack correctly.
\end{proof}
\begin{theorem}
    The number of partitions of $n$ into even-sized parts is equal to the number
of partitions of $n$ where each part has even multiplicity.
\end{theorem}
\begin{proof}
    Take the transpose.
\end{proof}
\begin{theorem}
    \[ p_4(n) = p_4(3n)\]
\end{theorem}
\begin{proof}
    Let $\lambda$ be a partition of $n$ with 4 parts. Then construct $\lambda^*$
    by replacing each part $\lambda_i$ with $n-\lambda_i$. This gives a partition of $3n$ into 4 parts. 
\end{proof}

Trivially, the above proof generalizes to yield $p_k(n) = p_k((k - 1)n)$.
\begin{theorem}
    $p_k(n) = \sum_{i=1}^k p_i(n-k)$
\end{theorem}
\begin{proof}
    Consider pre-seeding each of the $k$ parts with a 1. Then we have to partition $n-k$, 
    allowing for empty parts; thus we have to sum over the allowed
number of non-empty parts. This corresponds to excising the left-most column
in the Young tableuax.
\end{proof}