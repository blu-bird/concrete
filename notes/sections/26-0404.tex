% calkin-wilf
\label{26-0404}
\section{Interlude: Calkin-Wilf Tree}

\subsection{Hyperbinaries}
\begin{example}1
    What is the 23rd positive rational number?
\end{example}

\begin{solution}
    We need an enumeration of the rational numbers! 
\end{solution}

We introduce the hyperbinary numbers, where we write binary numbers, but
allow for multiple representation of numbers by also letting us use the number 2.
For example,
\begin{align*}
    4 &= 100 \\ &= 20 \\ &= 12
\end{align*}
Let us count the number of ways we can write integer $n$ in hyperbinary, $b(n)$. If
we sum $b(n)$, restarting every time we hit a 1, we get powers of 3 because it’s just
showing the number of strings of 3 different digits. To find a recurrence for $b(n)$,
let's casework on the last digit:
\begin{itemize}
    \item The only way to make an odd number is to take a number and append a 1:
    $b(2n + 1) = b(n)$
    \item The only ways to make an even number is to take a number and append a 0
    (which doubles it) or a 2 (which doubles it and adds 2): $b(2n+2) = b(n+1)+b(n)$
\end{itemize}
Donald Knuth called this recurrence $fusc(n)$ (fusc for ``obfuscate''). Let's return to
our original problem of finding the 23rd rational positive number. Define the $n$th
rational number to be $b(n)/b(n + 1)$. This list contains every rational in reduced
form exactly once.

Let's define the \textit{Calkin-Wilf tree} to be a complete binary tree that has as node $n$ the
rational $b_n/b_{n+1}$. Notice that each node that contains $b_n/b_{n+1}$ has as its left child
$b_{2n+1}/b_{2n+2}$ and right child $b_{2n+2}/b_{2n+3}$. Notice that applying the recurrence that
we had earlier yields relations
\begin{align*}
    \frac{b_{2n+1}}{b_{2n+2}} &= \frac{b_n}{b_n + b_{n+1}}\\
    \frac{b_{2n+2}}{b_{2n+3}} &= \frac{b_n + b_{n+1}}{b_{n+1}}
\end{align*}

Thus the parent of any node containing $x / y$ is either $x / (y-x)$ or $(x-y) / x$ depend-
ing on whether it is a left or right child. However, only one of these numbers is
positive, so we can deduce the parent of any node.

We know make the following claims:
\begin{theorem}
    $\frac x y \in T, \frac xy \neq \frac 1 1 \Longrightarrow \gcd(x, y) = 1$
\end{theorem}
\begin{proof}
    Assume that $\gcd(x, y) = k > 1$ for at least one $\frac x y \in T$. 
    Let $\frac{k x'}{k y'}$ be the highest node in the tree. However, 
    by our algorithm for finding the parent of any node, we know the parent
    must have greatest common divisor at least equal to $k$ because it is 
    $\frac{kx'}{k(y' - x')}$ or $\frac{k(x' - y')}{ky'}$. This contradicts our assmption.
\end{proof}

\begin{theorem}
    $\frac x y \in \Q^+, \gcd(x, y)= 1 \Longrightarrow \frac x y \in T$
\end{theorem}
\begin{proof}
    Let $S$ be the set of all rationals not in the tree. Let $y$ be the smallest denominator in $S$
    and let $x$ be minimal numerator of a number in $S$ with denominator
$y$. If $x > y$, then $\frac{x-y} y$ could not be in the tree because it would have child 
$\frac x y$ which, by assumption, isn't in the tree. But $\frac{x-y}y$ can't be in $S$ because
the way we choose our $x$ and $y$ Otherwise, if $x < y$, consider instead $\frac x{y-x}$. 

\end{proof}
\begin{theorem}
    Every rational that appears in the tree appears at most once. 
\end{theorem}
\begin{proof}
    Using the same rules as the previous proof, we find the ”smallest” x/y
that appears twice in the tree. However, it must have its parent appear twice,
which would be yet ``smaller''.
\end{proof}

\begin{theorem}
   The Calkin-Wilf tree contains every reduced positive rational exactly once.
\end{theorem}

\subsubsection*{A nice theorem}
\begin{definition}
    A \emph{runlength encoding} takes a number applies the ”look-and-say” rule: see
the length of each block of repeated digit, then write down its length and the digit. 
$444411002 \rightarrow 44212012$
\end{definition}
\begin{theorem}
    Consider, starting from the least significant digit of the binary representation of any positive integer $n$, the length of each run. Letting this be the continued fraction
representation of a rational, this gives the $n$th rational under the ordering given by the
Calkin-Wilf tree.
\end{theorem}
