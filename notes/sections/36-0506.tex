% \title{
% Orbit-Stabilizer Theorem
% }

% Scribe: Fred Zhang

% Date: Monday, May 6, 2019

\subsection{Orbits and Stabilizer}

Given $s \in S$ and $G$ a permutation group acting on $S$.

Definition. $\operatorname{orbit}(s)=\{\pi s \mid \pi \in G\}$

Definition. stabilizer $(s)=\{\pi \in G \mid \pi s=s\}$

Theorem 1.1. Orbit-Stabilizer Theorem: For all $s \in S$ and $G$ acting on $S$ $|\operatorname{orbit}(s)| \cdot \mid$ stabilizer $(s)|=| G \mid$

stabilizer(s) is a subgroup of $G$, so by Lagrange's theorem, $\mid$ stabilizer $(s)|||G|$

Lemma 1.1. If $\pi=s$ and $\pi \in \operatorname{stabilizer}(s)$, then $\pi^{-1} s=s$.

Proof.

$$
\begin{aligned}
s & =\pi s \\
\pi^{-1} s & =\pi^{-1} \pi s \\
\pi^{-1} s & =s
\end{aligned}
$$

So now the orbit of $S$ forms cosets of stabilizer $(s)$ (there is a 1-1 corespondence). So $|\operatorname{orbit}(s)|=\#$ of cosets of stabilizer(s). Applying Lagrange's theorme gives the orbit-stabilizer theorem.