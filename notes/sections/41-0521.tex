\label{41-0521}
\section{Turing Machines}
Turing Machines are a fundamental model of computation postulated to be
able to compute all computable problems.

\subsection{Basics}
\begin{definition}
    An \emph{alphabet} is just a set of ``characters,'' or ``symbols.''
\end{definition}
\begin{definition}
    A \emph{word} is just a string of characters from the alphabet.
\end{definition}
\begin{definition}
    A \emph{language $\mathcal L$} is a subset of the set of all words
\end{definition}
\begin{example}1
    All $\set{0, 1}^*$ strings with an even number of 0s form a language.
\end{example}
\begin{example}2
    All sentences in $[a - z]^*$ including ``the'' form a language.
\end{example}
\begin{example}3
    All $a, b, c, n$ such that $a^n + b^n = c^n, n > 2$ form a language.
\end{example}

\subsection{Push Down Automata}
Push Down Automata are a type of automata that also contain something
called the ``stack,'' which behaves like the stack we know from programming.

Consider the language $\mathcal L = ww^R$, or the palindromes in $\set{0, 1}^*$.
\begin{itemize}
    \item Read a character from input
    \item Push that character onto the ``stack''
    \item Repeat OR go to the next step
    \item Read a character and pop off the stack
    \item Compare the next character to the one popped off the stack
    \item Repeat previous two steps
    \item If the stack is empty when we run out of input, we end in ACCEPT
    state
\end{itemize}

This specific instance of a Push-Down Automata model is \textbf{Nondeterministic},
 which means that it somehow always takes the path to an “ACCEPT”
state if there is one.

\subsection{Turing Machines}
Turing Machines are even more robust.
\begin{definition}
    A Turing Machine has access to:
    \begin{itemize}
        \item Input, 
        \item state, 
        \item an infinite ``tape,'' which can be read or written to.
    \end{itemize}
    And thus has the ability to perform these operations:
    \begin{itemize}
        \item  Read input OR tape
        \item Write to tape
        \item Move left OR right
    \end{itemize}
\end{definition}

\begin{example}1
    Add $x + y$ in unary, i.e. $7 + 3 = 10$ in unary is
$$1111111 + 111 = 1111111111.$$
We model the input as
$$1111111\#111$$
and the code(notated with the state on the left before the colon and the state moved
to as after the semicolon,)
$$0 : 1 \to write(1), right; 0 $$
$$0 : \# \to nothing; 0$$
\end{example}

\begin{example}2
    Multiplication in Unary
We can take the input for multiplying 1111 and 111 to be 1111\#111 Using
this, we can get our process to be this:
\begin{itemize}
    \item Copy input to tape
    \item Read the first 1 in the tape, and move right until the next blank
    \item Copy the 1, and then change the first 1 to a 0
    \item Back up to the 0
    \item Repeat copying the 1 and pasting and then converting to a 0 until we
    hit the next string of digits.
    \item Change that 1 to a 0
    \item Change all 0s before back to a 1
    \item Goto step 3
    \item End at the string ``0\#1''
\end{itemize}
\end{example}

\begin{example}3
    Convert binary to unary
$$001101 \to \ ?$$
Code:

$s_0: 0 \to$ nothing; $s_0$,

$s_0 : 1 \to$ write 1; $s_1$,

$s_1 : 0 \to$ double output string; $s_1$,

$s_1 : 1 \to$ double output string, write 1; $s_1$

double $x$:

\qquad write $x\#x$

\qquad add, as we previously reviewed
\end{example}

\begin{example}4
    Dividing binary strings (integer division, not floating.)
$$1001101//110011$$
Procedure
\begin{itemize}
    \item Convert to unary
    \item Subtract the $a-b$ (overwrite ones with zeros)
    \item $Stop when b<a$
\end{itemize}
\end{example}

Q: Does a multitape Turing Machine have more power than a single tape
Turing Machine? 

A: We can interweave the contents of both tapes onto one
tape, and then Record the position of each multitape's position on the tape,
like so:
$$A_{-1}B_{-1}A_0B_0A_1B_1A_2B_2\dots\#P_A\#P_B.$$
Because a multitape is simulatable in a single tape Turing Machine, it is not
more powerful.

\subsection{Non-determinism}
$\mathcal L = $ \underline{composite numbers} in binary 

We can determine membership either deterministically or non-deterministically:

\subsubsection*{Deterministic Implementation}
Simulate 4 Tapes: $x$, $D$, $S_1 = x\#D$, and $S_2 : x \mod D$ And we can use the
code:\\

Increment $D$ \\
\quad if $x \mod D \equiv 0$, ACCEPT \\
\quad Else $D$++ \\
If $D=x$, REJECT \\

This is $pO(divide)$, where $p$ is the smallest prime factor of $x$.
This is an exponential time algorithm in binary length of input.

\subsubsection*{Non-deterministic Implementation}
Write $D$ OR $x$, \\
\qquad Accept if $D\mid x$ and $D < x$.\\

This is $O(divide) = O(n)$

\subsubsection{The P-NP problem}
The question is, then: Are non-deterministic Turing Machines necessarily
faster than their deterministic counterparts? 
In particular, is the set of languages computable by a non-deterministic Turing machine 
larger than the set computable by a deterministic Turing machine? If we
restrict each of these Turing machines to polynomial time, this is the famous
open problem $P = NP$.
