% \title{
% Structures and Exponential Generating Functions
% }

% \author{
% Scribe: Sohom Paul
% }

% Date: Tuesday, April 2, 2019

\section{Exponential Generating Functions}

Definition. A structure is a particular organization of the elements of a set or sets.

For example, some structures include a set, a non-empty set, an even-sized set, a permutation, an increasing permutation, or a function from $A$ into $B$

Definition. The exponential generating function for a sequence $f_{0}, f_{1}, f_{2}, f_{3} \ldots$ is

$$
F(x)=\sum_{k=0}^{\infty} f_{n} \frac{x^{n}}{n !}
$$

Definition. $[n]$ is the set $1,2, \ldots, n$, orthe set of $n$ elements.

Theorem 1.1. If $f_{n}$ counts the number of $F$-structures on $[n]$, then

$$
F(x)=\sum_{n=0}^{\infty} f_{n} \frac{x^{n}}{n !}
$$

is the exponential generating function for the structure $F$.

Let's look at some examples of exponential generating functions.

- $\frac{1}{1-x}$ is the exponential generating function of $n$ !

- $e^{a x}$ is the exponential generating function of $a^{n}$

Exponential generating functions add as expected, which corresponds combinatorially to $O R$, or disjoint union. Multiplication of exponential generating function is more involved. Let $A(x)=\sum_{p} a_{p} \frac{x^{p}}{p !}$ and $B(x)=\sum_{q} a_{q} b_{q} \frac{x^{q}}{q !}$. Then

$$
\begin{aligned}
A(x) B(x) & =\sum_{p, q} a_{p} b_{q} \frac{x^{p+q}}{p ! q !} \\
& =\sum_{n} \sum_{p} \frac{a_{p} b_{n-p} n !}{p !(n-p) !} \frac{x^{n}}{n !} \\
& =\sum_{n} \sum_{p}\left(\begin{array}{l}
n \\
p
\end{array}\right) a_{p} b_{n-p} \frac{x^{n}}{n !}
\end{aligned}
$$

Thus, if $C(x)=A(x) B(x)=\sum C_{n} \frac{x^{n}}{n !}$, then

$$
C_{n}=\sum_{p}\left(\begin{array}{l}
n \\
p
\end{array}\right) a_{p} b_{n-p}
$$

Recall that the Bell numbers obey recurrence

$$
B_{n+1}=\sum\left(\begin{array}{l}
n \\
k
\end{array}\right) B_{n-k}
$$

Let $b(x)=\sum B_{n} \frac{x^{n}}{n !}$ and let $a(x)=e^{x}$, the exponential generating function for $(1,1,1, \ldots)$. Plugging into the recurrence,

$$
\begin{aligned}
\sum_{n=0}^{\infty} B_{n+1} \frac{x^{n}}{n !} & =\sum_{n=0}^{\infty} \sum_{k=0}^{\infty}\left(\begin{array}{l}
n \\
k
\end{array}\right) B_{n-k} \frac{x^{n}}{n !} \\
b^{\prime}(x) & =b(x) e^{x} \\
b(x)=e^{e^{x}-1} &
\end{aligned}
$$

where we use the initial condition $b(0)=1$. Now, notice

$$
b(x)=\frac{1}{e} \sum_{k=0}^{\infty} \frac{\left(e^{x}\right)^{k}}{k !}=\frac{1}{e} \sum_{k=0}^{\infty} \sum_{n=0}^{\infty} \frac{1}{k !} \frac{(k x)^{n}}{n !}
$$

We can pull out our coefficient

$$
B_{n}=\frac{1}{e} \sum_{k=0}^{\infty} \frac{k^{n}}{k !}
$$

\subsection{Exponential Generating Functions for Structures}

Let's consider some structures on sets. Consider some trivial structures:

- a set on $[n]$ is $f_{n}=1$, so $F(x)=e^{x}$

- a non-empty set on $[n]$ is $f_{n}=1-\delta_{0 n}$, so $F(x)=e^{x}-1$

- an empty set on $[n]$ is $f_{n}=\delta_{0 n}$, so $F(x)=1$

- a singleton set on $[n]$ is $f_{n}=\delta_{1 n}$, so $F(x)=x$

- a 2-element set on $[n]$ is $f_{n}=\delta_{2 n}$, so $F(x)=\frac{1}{2} x^{2}$

- an even length set on $[n]$ has $F(x)=\cosh x$

$$
\text { Concrete Math - White - TJHSST }
$$

- an odd length set on $[n]$ has $F(x)=\sinh x$

Theorem 2.1. Addition Property: If $G$ and $H$ are exponential generating functions of structures, then $F=G+H$ is the exponential generating functions of the union of the structures

To see the above theorem in action, notice that the exponential generating function for empty sets, 1 , added to the exponential generating function for non-empty set, $e^{x}-1$, gives the exponential generating function for all sets, $e^{x}$

Theorem 2.2. Multiplication Property: Let $g$ and $h$ be structures on sets and let fbe a gh structure. For set $A$, if we partition $A$ into disjoint sets $A_{1} \cup A_{2}$, put all elements in $A_{1}$ into a $g$-structure and $A_{2}$ into a $h$-structure, then the exponential generating function of $f$ is $F(x)=G(X) H(x)$

Example 1. Let us count subsets of $A=[n]$. Partition $A$ into a set and another set (the complement). The number of ways to do this is given by the coefficient

$$
f(x)=g(x) h(x)=e^{x} e^{x}=e^{2 x}=\sum 2^{n} \frac{x^{n}}{n !}
$$

Thus a set of size $n$ has $2^{n}$ different subsets.

Example 2. Let us count non-empty subsets of $A=[n]$. Analogous to above, partition $A$ in $g$-structure representing non-empty sets and $h$-structure representing sets. This gives

$$
f(x)=g(x) h(x)=\left(e^{x}-1\right)\left(e^{x}\right)=\sum_{n=0}^{\infty}\left(2^{n}-1\right) \frac{x^{n}}{n !}
$$

Thus a set of size $n$ has $2^{n}-1$ non-empty subsets.

Example 3. Let us count functions from $[n] \rightarrow[k]$. Let $A=A_{1} \cup A_{2} \cup \ldots A_{k}$ where $A_{i}$ is the preimage of $i$ in $[n]$, so

$$
f(x)=\left(e^{x}\right)^{k}=\sum_{n=0}^{\infty} k^{n} \frac{x^{n}}{n !}
$$

Thus we have $k^{n}$ functions from $[n] \rightarrow[k]$

Example 4. Let us count surjective functions from $[n] \rightarrow[k]$. By analogy to above problem, we have to let each of sets $A_{i}$ be nonempty, so we have

$$
f(x)=\left(e^{x}-1\right)^{k}=\sum k ! S(n, k) \frac{x^{n}}{n !}
$$

where we appeal to the 12-fold table. Thus, the exponential generating function of the Stirling numbers of the second kind is

$$
\sum_{n=0}^{\infty} S(n, k) \frac{x^{n}}{n !}=\frac{\left(e^{x}-1\right)^{k}}{k !}
$$

$$
\text { Concrete Math - White - TJHSST }
$$

Let's continue the derivation of the exponential generating function of the Stirling numbers of the second kind.

$$
\begin{aligned}
\sum_{n=0}^{\infty} S(n, k) \frac{x^{n}}{n !} & =\frac{1}{k !}\left(e^{x}-1\right)^{k} \\
& =\frac{1}{k !} \sum_{t=0}^{k}\left(\begin{array}{l}
k \\
t
\end{array}\right) e^{t x}(-1)^{k-t} \\
& =\frac{1}{k !} \sum_{t=0}^{k}\left(\begin{array}{l}
k \\
t
\end{array}\right) \sum_{n=0}^{\infty} \frac{(t x)^{n}}{n !}(-1)^{k-t} \\
& =\frac{1}{k !} \sum_{n=0}^{\infty}\left(\sum_{t=0}^{k}\left(\begin{array}{l}
k \\
t
\end{array}\right)(-1)^{k-t} t^{n}\right) \frac{x^{n}}{n !}
\end{aligned}
$$

This gives

$$
S(n, k)=\frac{1}{k !} \sum_{t=0}^{k}\left(\begin{array}{l}
k \\
t
\end{array}\right)(-1)^{k-t} t^{n}
$$

which we had derived earlier by using Principle of Inclusion and Exclusion

Theorem 2.3. Composition Property: Let $g, h$ be structures on sets and let $f=g \circ h$, the structure by

- partitioning $A$ into an arbitrary number of blocks $A_{1} \cup A_{2} \cup \ldots A_{k}$

- giving each $A_{i}$ an h-structure

- and giving the partition itself a g-structure

The exponential generating function obeys $F(x)=G(H(X))$

Example 5. Let us count the number of ways to partition a set. Each of our partitions of $A$ must be non-empty sets, but the structure for the partitions is merely a set, so our exponential generating function is $e^{e^{x}-1}$. This gives again our exponential generating function for the Bell numbers.

Example 6. Let us count the number of ways to partition a set into subsets with at least two elements. Our exponential generating function is $e^{e^{x}-1-x}$

Example 7. Let us count the number of ways to partition a set into subsets with even number of elements. The exponential generating function is $e^{\cosh x}$