% \title{
% Structures and Exponential Generating Functions
% }

% \author{
% Scribe: Sohom Paul
% }

% Date: Tuesday, April 2, 2019
\label{25-0402}
\section{Exponential Generating Functions}

\begin{definition}
    A \emph{structure} is a particular organization of the elements
     of a set or sets.
\end{definition}
For example, some structures include a set, a non-empty set, 
an even-sized set, a permutation, an increasing permutation, 
or a function from $A$ into $B$
\begin{definition}
    The exponential generating function for a sequence 
    $f_{0}, f_{1}, f_{2}, f_{3} \ldots$ is
\[
    F(x)=\sum_{k=0}^{\infty} f_{n} \frac{x^{n}}{n !}
\]
\end{definition}
\begin{definition}
    $[n]$ is the set $1,2, \ldots, n$, or the set of $n$ elements.
\end{definition}
\begin{theorem}
    If $f_{n}$ counts the number of $F$-structures on $[n]$, then
\[
    F(x)=\sum_{n=0}^{\infty} f_{n} \frac{x^{n}}{n !}
\]
is the exponential generating function for the structure $F$.
\end{theorem}

Let's look at some examples of exponential generating functions.
\begin{itemize}
    \item $\frac{1}{1-x}$ is the exponential generating function of $n!$
    \item $e^{a x}$ is the exponential generating function of $a^{n}$
\end{itemize}
Exponential generating functions add as expected, which corresponds 
combinatorially to \emph{OR}, or disjoint union. Multiplication of
exponential generating function is more involved. 
Let $A(x)=\sum_{p} a_{p} \frac{x^{p}}{p!}$ and $B(x)=\sum_{q} b_{q} \frac{x^{q}}{q!}$. 
Then
\begin{align*}
    A(x) B(x) & =\sum_{p, q} a_{p} b_{q} \frac{x^{p+q}}{p ! q !} \\
& =\sum_{n} \sum_{p} \frac{a_{p} b_{n-p} n !}{p !(n-p) !} \frac{x^{n}}{n !} \\
& =\sum_{n} \sum_{p} \binom n p a_p b_{n-p} \frac{x^n}{n!}
\end{align*}
Thus, if $C(x)=A(x) B(x)=\sum C_{n} \frac{x^{n}}{n !}$, then
\[
    C_n = \sum_p \binom n p a_p b_{n-p}
\]
Recall that the Bell numbers obey recurrence
\[
    B_{n+1} = \sum \binom n k B_{n-k}
\]
Let $b(x)=\sum B_{n} \frac{x^{n}}{n !}$ and let $a(x)=e^{x}$, 
the exponential generating function for $(1,1,1, \ldots)$. 
Plugging into the recurrence,
\begin{align*}
    \sum_{n=0}^{\infty} B_{n+1} \frac{x^{n}}{n !} & =\sum_{n=0}^{\infty} \sum_{k=0}^{\infty} \binom n k B_{n-k} \frac{x^n}{n!} \\
    b'(x) &= b(x) e^x \\ 
    b(x) = e^{e^x - 1} &
\end{align*}
where we use the initial condition $b(0)=1$. Now, notice
\[
    b(x)=\frac{1}{e} \sum_{k=0}^{\infty} \frac{\left(e^{x}\right)^{k}}{k !}=\frac{1}{e} \sum_{k=0}^{\infty} \sum_{n=0}^{\infty} \frac{1}{k !} \frac{(k x)^{n}}{n !}
\]
We can pull out our coefficient
\[
B_{n}=\frac{1}{e} \sum_{k=0}^{\infty} \frac{k^{n}}{k !}
\]

\subsection{Exponential Generating Functions for Structures}
Let's consider some structures on sets. Consider some trivial structures:
\begin{itemize}
    \item a set on $[n]$ is $f_{n}=1$, so $F(x)=e^{x}$
    \item a non-empty set on $[n]$ is $f_{n}=1-\delta_{0 n}$, so $F(x)=e^{x}-1$
    \item an empty set on $[n]$ is $f_{n}=\delta_{0 n}$, so $F(x)=1$
    \item a singleton set on $[n]$ is $f_{n}=\delta_{1 n}$, so $F(x)=x$
    \item a 2-element set on $[n]$ is $f_{n}=\delta_{2 n}$, so $F(x)=\frac{1}{2} x^{2}$
    \item an even length set on $[n]$ has $F(x)=\cosh x$
    \item an odd length set on $[n]$ has $F(x)=\sinh x$
\end{itemize}

\begin{theorem}
    Addition Property: If $G$ and $H$ are exponential generating functions of structures, then $F=G+H$ is the exponential generating functions of the union of the structures
\end{theorem}

To see the above theorem in action, notice that the exponential generating function for empty sets, 1 , added to the exponential generating function for non-empty set, $e^{x}-1$, gives the exponential generating function for all sets, $e^{x}$

\begin{theorem}
    Multiplication Property: Let $g$ and $h$ be structures on sets and let $f$ be a $gh$-structure. For set $A$, if we partition $A$ into disjoint sets $A_{1} \cup A_{2}$, put all elements in $A_{1}$ into a $g$-structure and $A_{2}$ into a $h$-structure, then the exponential generating function of $f$ is $F(x)=G(X) H(x)$
\end{theorem}

\begin{example}1
    Let us count subsets of $A=[n]$. Partition $A$ into a set and another set (the complement). The number of ways to do this is given by the coefficient
\[
    f(x)=g(x) h(x)=e^{x} e^{x}=e^{2 x}=\sum 2^{n} \frac{x^{n}}{n !}
\]
Thus a set of size $n$ has $2^{n}$ different subsets.
\end{example}

\begin{example}2
    Let us count non-empty subsets of $A=[n]$. Analogous to above, partition $A$ in $g$-structure representing non-empty sets and $h$-structure representing sets. This gives
\[
    f(x)=g(x) h(x)=\left(e^{x}-1\right)\left(e^{x}\right)=\sum_{n=0}^{\infty}\left(2^{n}-1\right) \frac{x^{n}}{n !}
\]
Thus a set of size $n$ has $2^{n}-1$ non-empty subsets.
\end{example}

\begin{example}3
    Let us count functions from $[n] \rightarrow[k]$. Let $A=A_{1} \cup A_{2} \cup \ldots A_{k}$ where $A_{i}$ is the preimage of $i$ in $[n]$, so
\[
f(x)=\left(e^{x}\right)^{k}=\sum_{n=0}^{\infty} k^{n} \frac{x^{n}}{n !}
\]
Thus we have $k^{n}$ functions from $[n] \rightarrow[k]$
\end{example}

\begin{example}4
    Let us count surjective functions from $[n] \rightarrow[k]$. By analogy to above problem, we have to let each of sets $A_{i}$ be nonempty, so we have
\[
    f(x)=\left(e^{x}-1\right)^{k}=\sum k ! S(n, k) \frac{x^{n}}{n !}
\]
where we appeal to the 12-fold table. Thus, the exponential generating function of the Stirling numbers of the second kind is
\[
    \sum_{n=0}^{\infty} S(n, k) \frac{x^{n}}{n !}=\frac{\left(e^{x}-1\right)^{k}}{k !}
\]
\end{example}
Let's continue the derivation of the exponential generating function of the Stirling numbers of the second kind.
\begin{align*}
    \sum_{n=0}^{\infty} S(n, k) \frac{x^{n}}{n !}
     & =\frac{1}{k !}\left(e^{x}-1\right)^{k} \\
    & =\frac{1}{k !} \sum_{t=0}^{k} \binom k t e^{tx} (-1)^{k-t} \\
    & =\frac{1}{k !} \sum_{t=0}^{k} \binom k t \sum_{n=0}^{\infty} \frac{(t x)^{n}}{n !}(-1)^{k-t} \\
    & = =\frac{1}{k !} \sum_{n=0}^{\infty}\left(\sum_{t=0}^{k} \binom k t (-1)^{k-t} t^{n}\right) \frac{x^{n}}{n !}
\end{align*}
This gives
\[
    S(n, k)=\frac{1}{k !} \sum_{t=0}^{k} \binom k t (-1)^{k-t} t^{n}
\]
which we had derived earlier by using Principle of Inclusion and Exclusion
\begin{theorem}
    Composition Property: Let $g, h$ be structures on sets and let $f=g \circ h$, the structure by
\begin{itemize}
    \item partitioning $A$ into an arbitrary number of blocks $A_{1} \cup A_{2} \cup \ldots A_{k}$
    \item giving each $A_{i}$ an $h$-structure
    \item and giving the partition itself a $g$-structure
\end{itemize}
The exponential generating function obeys $F(x)=G(H(X))$
\end{theorem}

\begin{example}5
    Let us count the number of ways to partition a set. Each of our partitions of $A$ must be non-empty sets, but the structure for the partitions is merely a set, so our exponential generating function is $e^{e^{x}-1}$. This gives again our exponential generating function for the Bell numbers.
\end{example}

\begin{example}6
    Let us count the number of ways to partition a set into subsets with at least two elements. Our exponential generating function is $e^{e^{x}-1-x}$
\end{example}

\begin{example}7
    Let us count the number of ways to partition a set into subsets with even number of elements. The exponential generating function is $e^{\cosh x}$
\end{example}
